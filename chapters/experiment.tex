% experiment.tex: Chapter describing the experiment

\chapter{The CMS Experiment}
\label{experiment_chapter}

\section{The Large Hadron Collider}
\label{lhc_section}

The Large Hadron Collider (LHC) is the worlds highest energy and largest
particle accelerator with a maximum design luminosity of 14\TeV and a radius of
2804\meters \cite{bruning2004}. It collides protons on protons. In 2012, when
LHC most recently produced collisions, it had a center of mass energy of 7\TeV;
when it turns back on in 2015 after upgrades it will run at either 13\TeV or
14\TeV.

The LHC is located near Geneva, Switzerland, although parts of the accelerator
(including \pointfive, where the CMS detector is located) are in France.

A number of smaller accelerators are used together in series to accelerate
protons to the energies necessary to be injected into the LHC. The first step
is a linear accelerator, \linactwo, which accelerates protons from rest to to
50\MeV. These protons are then injected into a chain of three circular
accelerators, each injecting into the next. The first of these accelerators is
the Proton Synchrotron Booster (PSB) which accelerates the protons to 1.4\GeV.
The second is the Proton Synchrotron (PS) which accelerates the protons to
26\GeV. The third and final accelerator in the Super Proton Synchrotron (SPS)
which accelerates the protons to 450\GeV and injects directly into the LHC.

Bunches of protons are accelerated using this system and injected into the LHC
to form two, counter-rotating beams. When the desired number of bunches have
been injected into the LHC, the LHC accelerates them to 4\TeV. When the beams
have reached their nominal energy, they are focused and brought into collision
at four different points on the ring where the various experiments (\ALICE,
\ATLAS, \LHCB, and CMS) are located.
