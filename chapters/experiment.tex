% experiment.tex: Chapter describing the experiment

\chapter{The CMS Experiment}
\label{experiment_chapter}

\section{The Large Hadron Collider}
\label{lhc_section}

\TODO{LHC Graphic of accelerator chain}

The Large Hadron Collider (LHC) is the worlds highest energy and largest
particle accelerator with a maximum design luminosity of 14\TeV and a radius of
2804\meters \cite{bruning2004}. It collides protons on protons. In 2012, when
LHC most recently produced collisions, it had a center of mass energy of 7\TeV;
when it turns back on in 2015 after upgrades it will run at either 13\TeV or
14\TeV.

The LHC is located near Geneva, Switzerland, although parts of the accelerator
(including \pointfive, where the CMS detector is located) are in France.

A number of smaller accelerators are used together in series to accelerate
protons to the energies necessary to be injected into the LHC. The first step
is a linear accelerator, \linactwo, which accelerates protons from rest to to
50\MeV. These protons are then injected into a chain of three circular
accelerators, each injecting into the next. The first of these accelerators is
the Proton Synchrotron Booster (PSB) which accelerates the protons to 1.4\GeV.
The second is the Proton Synchrotron (PS) which accelerates the protons to
26\GeV. The third and final accelerator in the Super Proton Synchrotron (SPS)
which accelerates the protons to 450\GeV and injects directly into the LHC.

Bunches of protons are accelerated using this system and injected into the LHC
to form two, counter-rotating beams. When the desired number of bunches have
been injected into the LHC, the LHC accelerates them to 4\TeV. When the beams
have reached their nominal energy, they are focused and brought into collision
at four different points on the ring where the various experiments (\ALICE,
\ATLAS, \LHCB, and CMS) are located.

The beams are steered around the accelerator ring by a series of
superconducting, dipole magnets. When running at a center of mass energy of
8\TeV, these magnets operate at roughly 7.5\Tesla. There are also quadrapole
and some higher order magnets around the ring used to focus the beams. The
bunches of protons are accelerated by 16 superconducting radio frequency
cavities. These cavities accelerate slower protons while slowing faster ones,
thereby keeping the proton bunches compact in both real and momentum space.
There is room for 2808 bunches in LHC, although in 2012 there were only
\TODO{how many bunches?}.

The LHC is also the highest luminosity collider in the world. The instantaneous
luminosity is given by
\begin{equation}
    \luminosity=fn\frac{N^{2}}{\sigma}
\end{equation}
where $f$ is the frequency of interaction (which is fixed by the LHC's
circumference), $n$ is the number of bunches in a beam, $N$ is the number of
protons per bunch (with the $N^{2}$ coming from the assumption that there are
the same number of protons in the two colliding bunches), and $\sigma$ is the
area profile of the beams. Although a higher luminosity means more particles
are produced and more data can be collected, the maximum luminosity is limited
by several practical factors. The first factor is cost; a higher luminosity
general requires a more expensive machine as the ring is either made larger or
the technology needed to run the machine is made more complex. The second is
the challenge that higher luminosities present to the analyzers. The luminosity
can be increased by increasing the number of protons in a bunch or by squeezing
the bunches more tightly, but eventually the probability of getting multiple
proton-proton interactions per bunch crossing becomes large, leading to a
phenomenon known as pileup. These extra interactions add additional particles
to the detector and can make it difficult to separate interesting events from
uninteresting background. The luminosity can also be increased by increasing
the number of bunches in the machine, but this decreases the time between the
collisions and leads to a phenomenon known as out of time pileup which can also
obscure interesting events. In 2012, the optimal luminosity was achieved by
running with bunches spaced by 50\ns instead of the design nominal bunch
spacing of 25\ns. The decision to use this bunch spacing was driven by
\linactwo which had trouble injecting high population bunches.

\section{The Compact Muon Solenoid}
\label{cms_section}

\TODO{CMS cutaway detector picture \label{cms_cutaway_fig}}

The Compact Muon Solenoid (CMS) is one of two general purpose particle
detectors built on the LHC ring \cite{cms_tdr_1}\cite{cms_tdr_2}. CMS is
designed to detect the very high energy, sub-atomic particles that are produced
in the LHC's proton-proton collisions. CMS is designed to have a high
acceptance and efficiency for these collisions. Here acceptance indicates the
area in both physical space as well as the area in energy and momentum space in
which the detector can detect particles. Efficiency is the probability that a
particle in CMS's acceptance region is properly measured. In order to have a
high acceptance and efficiency, CMS must be both large---to cover a large area
of physical space---and dense---to cover a large area in momentum and energy
space. CMS is \TODO{21\meters in length, 15\meters in diameter, and weighs
14\kilotonne}.

CMS is built as a series of nested, finite cylinders, where each cylinder is a
separate subdetector. The beams enter the detector along the axis of the
cylinder. The collision point is in the center. There are a pair of endcaps on
either side of the cylinder to increase the acceptance of the detector. The
endcaps and central region of the cylinder (call the barrel) overlap to prevent
particles from escaping undetected through the crack. A cutaway of the detector
is shown in \FIG~\ref{cms_cutaway_fig}.

The coordinate system used by CMS is as follows: the origin is the nominal
interaction point at the center of CMS, the \xaxis is defined to point to the
center of the LHC ring, the \yaxis is defined as vertically up, and the \zaxis
points counter-clockwise and tangent to the LHC ring such that it forms a
right-handed coordinate system with the \xaxis and \yaxis. CMS uses a
cylindrical coordinate system with coordinates ($\eta$, $\phi$). The azimuthal
angel $\phi$ is in the \xyplane while $\eta$ is the pseudorapidity defined by
\begin{equation}
    \eta = -\logn \tan \frac{\theta}{2}
\end{equation}

where $\theta$ is the polar angle with respects to the \zaxis.
