\chapter{Unfolding}
\label{chatper:unfolding}

Particle detectors are incredibly sophisticated machines designed to make
precise measurements of the various decay products created during the
collisions. However, there are physical limitations that prevent a perfect
measurement from being made. The finite energy, momentum, and position
resolution of the various subdetectors impose limitations on the final
measurements. In order to allow our measurement to be compared to theoretical
predictions, we correct for these detector effects with a process known as
unfolding.

We unfold in two steps: first the data are unfolded to correct for bin
migration, second we correct for the imperfect efficiency of the trigger and
reconstruction. The data are unfolded against the three definitions of
generator level electrons: born, dressed, and bare as described in
\SEC~\ref{sec:electron_dressing}. Results in this chapter are shown with
dressed electrons, born and bare electrons are given in \APP\TODO{Born, Bare
Appendix}.

\section{Bin Migration}
\label{sec:bin_migration}

The finite resolution of CMS's angular position measurements leads to a finite
resolution of the reconstructed \phistar distribution. Events which at the
generator level would have ended up in a certain \phistar bin may instead
migrate to one of the neighboring bins.

MC events are used to create a response matrix that describes how generator
level events are reconstructed in CMS. Through unfolding this matrix is
inverted, allowing us to transform the data to remove the undesired bin
migration. This matrix is shown in \FIG~\ref{fig:bin_migration_matrix}. The
amount of bin migration, as measured by the off-diagonal elements, is 5.5\%.

\TODO{Bin migration plot that is the input to \RooUnfold
\label{fig:bin_migration_matrix}.}

\RooUnfold is used to perform the unfolding \cite{adye_2011}. It implements
Bayes' theorem as described in \cite{dagostini_1995}, and iteratively applies
it to invert the response matrix. \RooUnfold uses a limited number of
iterations in order to terminate the algorithm before finding the true (but
highly unstable) inverse. Four iterations are used in this analysis.

\subsection{Closure Tests}

This unfolding procedure was tested in several ways using the two signal MC
samples discussed in \SEC~\ref{ssec:monte_carlo}. First, the reconstructed
\phistar distributions from the \MADGRAPH and \POWHEG sample were unfolded
using their own generator level quantities and compared to their generator
level \phistar distributions. The results of this test are shown
\FIG~\ref{fig:unfolded_vs_true_in_mc}.

\TODO{Figure 8 from the AN \label{fig:unfolded_vs_true_in_mc}}

Second, each MC signal sample was also divided into two independent halves.
The reconstructed \phistar distribution in each half was unfolded using the
generator level quantities from the other half. The two halves were then
compared to each other. The results of this test are shown in
\FIG~\ref{fig:half_vs_half_unfolding}.

\TODO{Figure 9 from the AN \label{fig:half_vs_half_unfolding}}

Third and finally, the reconstructed \phistar distribution in the \MADGRAPH
MC was unfolded with the generator level quantities from the \POWHEG sample and
vice versa. The results of this test are shown in
\FIG~\ref{fig:cross_mc_unfolding}.

\TODO{Figure 10 from the AN \label{fig:cross_mc_unfolding}}

In most cases the results are consistent within the assigned statistical error
bars. However, in the case of the \MADGRAPH sample unfolded with \POWHEG, there
is disagreement between in the low \phistar bins. This is due \RooUnfold under
estimating the uncertainties, as discussed in
\SEC~\ref{ssec:unfolding_statistical_uncertainties}.

\subsection{Statistical Uncertainties}
\label{ssec:unfolding_statistical_uncertainties}

The uncertainties shown in \FIGS~\ref{fig:unfolded_vs_true_in_mc},
\ref{fig:half_vs_half_unfolding}, and \ref{fig:cross_mc_unfolding} are the
statistical uncertainties returned by \RooUnfold. These uncertainties only take
the number of events in the reconstructed distribution into account. The
uncertainties are correlated by the off-diagonal elements. While \RooUnfold can
return the full covariance matrix, we instead use a simpler approximation
provided by \RooUnfold where the uncertainties are based only on the diagonal
elements. We tested the effects of this simplification on the uncertainties
using two tests discussed below.

In the first test, each bin in the \phistar distribution was regenerated from a
Gaussian with the value of the original bin as the mean and the value of the
statistical uncertainty on the bin from the original distribution as the
standard deviation. Using this process, \num{500} \phistar distributions were
generated, giving us \num{501} total distributions as the original distribution
was also used. These distributions were then unfolded using the full \MADGRAPH
sample. The results are shown in \FIG~\ref{fig:toy_unfolding_results} where the
\TODO{triangle points} are the median value of the 501 distributions, and the
error bars show the extent of the value in the central 68.2\% of the
distributions. These are compared to the errors provided by \RooUnfold. Both
methods lead to identical results.

\TODO{Figure 11 from the AN \label{fig:toy_unfolding_results}}

In a second test, \num{5000} and \num{50000} randomly selected events from the
\POWHEG sample were used instead of the data, and the same 501 distributions
were constructed in the same manner as above. The results of this test are
shown in \FIG~\ref{fig:toy_powheg_unfolding_results}. As expected, when fewer
events are used the uncertainties reported by \RooUnfold are larger.

\TODO{Figure 12 from the AN \label{fig:toy_powheg_unfolding_results}}

While \RooUnfold properly handles the statistical uncertainty due to the number
of events in the data being unfolded, it does not account for statistical
uncertainty due to the number of generator level events used to create the
bin migration matrix. This leads to an underestimation of the total uncertainty
which is especially pronounced in cases where the number of events in the
MC sample is smaller than the number of data events, as is the case in this
analysis.

A toy MC method is used to propagate the uncertainty from the bin migration
matrix to the unfolded distribution. In this method, \num{500} new bin
migration matrices are generated by randomly fluctuating each bin of the
original matrix generated using \num{5000} events from the \POWHEG sample. Bins
with \num{0} events are left at \num{0}. Bins with a small number of
events---where the number of events in the bin divided by the statistical
uncertainty is less than \num{5}---are fluctuated using a Poisson distribution
with the number of events in the original bin as the most probable value. All
other bins are fluctuated using a Gaussian with mean equal to the number of
events in the original bin and standard deviation equal to the statistical
uncertainty on the original bin. The reconstructed \phistar distribution from
the full \MADGRAPH sample is then unfolded with each of the \num{501} bin
migration matrices. \TODO{The results are shown in
\FIG~\ref{fig:5000_propegation_unfolding}. The top left plot shows the
uncertainties from \RooUnfold as compared to the uncertainties derived using
the toy MC method. The lower left plot shows the extent of the \phistar bin
values generated by the ensemble of tow MC bin migration matrices. The top
right shows the pull of the bins in the top left plot. The bottom right plot
shows the original bin migration matrix.}

\TODO{Figure 13 from the AN \label{fig:5000_propegation_unfolding}}

At large \phistar values the uncertainty calculated with the toy MC method goes
to \num{0}. This is because there are no off-diagonal bins in this region of
the original bin migration matrix and so no off-diagonal bins are allowed to
appear due to our fluctuations. This problem is greatly reduced as we use more
generator level events to construct the matrix, as can be seen in
\FIGS~\ref{fig:50000_propegation_unfolding} and
\ref{fig:full_propegation_unfolding}.

\TODO{Figure 14 from the AN \label{fig:50000_propegation_unfolding}}
\TODO{Figure 15 from the AN \label{fig:full_propegation_unfolding}}

The total statistical uncertainty due to the bin migration unfolding is the
sum of the uncertainty reported by \RooUnfold due to the number of events in
the data and the uncertainty calculated using the toy MC method in quadrature.

\subsection{Systematic Uncertainties}
\label{ssec:unfolding_systematic_uncertainties}

The unfolding to correct for bin migration is dependent on the way in which the
bin migration is simulated in MC. Differences in the \phistar distribution
between MC and data can lead to systematic uncertainties in the unfolded data.
Two such potential differences between the MC and the data are considered. The
first potential difference is in the shape of the \phistar distribution, and
the second is in the resolution of the \phistar distribution.

One of the advantages of the Bayesian unfolding method used in this analysis is
that it, unlike a simpler bin-by-bin correction, is theoretically insensitive
to the distribution of \phistar at the generator level. In fact, \DAgostini
recommends using a flat generated distribution as one easy method of ensuring
that there are enough generated events in each bin \cite{dagostini_1995}.
\TODO{Cite twice in same chapter?} We use this recommendation as the basis for
a test of the systematic uncertainty. The \phistar distribution at the
generator level in the \MADGRAPH sample is inserted into a histogram with bin
widths of $\phistar = 0.011$. Each bin is weighted with a weight equal to the
inverse number of events in the bin so that the distribution is flattened. The
full \POWHEG sample is then unfolded using this \MADGRAPH distribution.
\TODO{The response matrix from this modified \MADGRAPH distribution is show in
  \FIG~\ref{fig:flat_unfolding}. The ratio of the unfolded \POWHEG distribution
over the generated distribution is shown as well.} This ratio uses only the
uncertainties provided by \RooUnfold and so underestimates the total
uncertainty. No deviation from \num{1} is seen, and so no systematic
uncertainty is assigned for the shape of the generator level \phistar
distribution.

\TODO{Figure 16 from the AN \label{fig:flat_unfolding}}

Differences in the resolution can arises due to differences in how the
detector is simulated and how the detector responds in reality. Various
corrections are applied to the MC in order to make it more closely match the
data; these are discussed in \SEC~\ref{sec:scale_factors}. The uncertainties
from these corrections are discussed in \CHP~\ref{chapter:uncertainties}.

\section{Efficiency Correction}

Not every event which should be detected by CMS is. Some events are lost at
each stage of event selection and reconstruction. These lost events must be
corrected for in order to accurately compare our results to theory. We
therefore apply corrections for the reconstruction efficiency, the electron ID
efficiency, and the trigger efficiency.

The efficiency corrections are applied after the bin migration corrections. The
correction factors are derived for each \phistar bin using the \Ztoee \MADGRAPH
sample discussed in \SEC~\ref{ssec:monte_carlo}. Two sets of events are
selected. The first set, the ``acceptance set'', is created by applying the
acceptance definition discussed in \SEC~\ref{sec:acceptance} to the generator
level quantities in the MC. The second set, the ``final selection set'', is
created by selecting MC events by applying the full analysis selection to the
reconstructed level quantities in the MC. The efficiency in each \phistar bin
is calculated by counting the number of events in the ``final selection set''
in a bin, and dividing by the number of events in the same bin in the
``acceptance set''. This gives us an average efficiency composed of all the
efficiencies of the events in the bin.

The main advantage of using this average efficiency instead of correcting each
event's efficiency individually is that any correlations between the various
efficiencies are automatically taken into account in this process.

\TODO{Add plots from AN when they're finalized.}
