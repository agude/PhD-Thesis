\chapter{Monte Carlo}
\label{chatper:monte_carlo}

\section{Tag and Probe}

Tag and Probe (\TnP) is a minimally biased method of calculating the efficiency
of some process that is applied. \TnP takes advantage of the well known mass
and narrow width of the \Z boson to select a set of electrons for which very
few selection requirements have been applied. This is done by finding one
high-quality electron, the tag, and another minimally biased object, the probe,
that could be an electron. The invariant mass of these objects is computed and
if it is near the \Z mass peak, it is very likely that the probe is also an
electron. These probe electrons then have further selection requirements
applied to them, and the fraction that pass is taken as the efficiency of that
selection step.

\section{Scale Factors}

The detector response to various signals is not always perfectly simulated in
MC and so the efficiencies of various selection requirements are not the same
in data and MC. In order to correct this, each event in MC is reweighed with a
series of scale factors (SF) which are the efficiency of some selection process
in data divided by the same efficiency as measured on MC:

\begin{equation}
    \label{eq:sf}
    \text{SF} = \frac{\effdata}{\effmc}
\end{equation}

\subsection{Electron Reconstruction}

Electron reconstruction begins with the assembly of a supercluster in ECAL, and
ends with the matching of a supercluster to a track in the tracker. The details
of Electron reconstruction are described in
\SEC~\ref{sec:electron_reconstruction}. The efficiency of an electron with $\pt
> 20 \GeV$ depositing enough energy in ECAL to be reconstructed into a
supercluster is very high, although the efficiency must be measured in
simulation as there is no more basic object with which to perform \TnP to
measure it in data. SF for matching a track given that a supercluster has
already been found are provided by the collaboration
\cite{gsf_scale_factors_2013}.

The events used to measure the reconstruction SF are selected with the
dedicated electron \TnP Trigger: \TnPTrigger. This trigger requires one
electron with $\pt > 20 \GeV$ which must also pass very tight isolation and ID
requirements while requiring only a low energy ($\et > 4 \GeV$) supercluster as
the other leg. The trigger rate is kept down by requiring that the invariant
mass of these two objects is greater than $50 \GeV$.

These events are further required to pass a set of offline selection
requirements. The tag electron is required to pass \EGTIGHT, have $\pt > 25
\GeV$, and $|\eta| < 2.5$. Electrons are rejected if they fall in the seem
between EB and EE ($1.4442 < |\eta| < 1.566$). The tag must also be matched to
the tight leg of the \TnP trigger. The probe supercluster has minimal
requirements applied; it is required to have tracker isolation $< 0.15$. For
the MC sample, the tag is only required to be matched to a generator level
electron with $\Delta R < 0.2$. Additionally, the event was required to have
low \particleflow missing energy $\PFMET < 20 \GeV$. \TODO{Explain PF MET?}

The events were binned in terms of probe's $\pt$ and $\eta$ as well as whether
the probe passed or failed. In each bin the \mee distribution was constructed
and a template consisting of the sum of a Gaussian smeared \Ztoee MC sample and
an exponential background was fitted. The number of events predicted by the
signal fit on the passing sample, failing sample, and sum of the two samples
was used to get the efficiency. A similar process was performed on MC, although
instead of a fit a simple counting of passing events was performed (as there is
no background in MC). The resulting SF are given in
\TAB~\ref{table:gsf_scale_factor}.

\begin{table}[h]
\centering
\begin{center}
    \begin{tabular}{ | c | c | c | c | c |} \hline
	$|\eta|$                   & 20--30 \GeV                        & 30--40 \GeV                        & 40--50 \GeV                        & $>$ 50 \GeV                        \\ \hline
	$\numrange{0.0}{0.8}$      & $\effstatsys{0.982}{0.003}{0.012}$ & $\effstatsys{0.988}{0.001}{0.008}$ & $\effstatsys{0.990}{0.001}{0.004}$ & $\effstatsys{0.990}{0.001}{0.004}$ \\ \hline
	$\numrange{0.8}{1.4442}$   & $\effstatsys{0.993}{0.002}{0.012}$ & $\effstatsys{0.993}{0.001}{0.008}$ & $\effstatsys{0.993}{0.001}{0.004}$ & $\effstatsys{0.991}{0.001}{0.004}$ \\ \hline
	$\numrange{1.4442}{1.566}$ & $\effstatsys{1.016}{0.012}{0.020}$ & $\effstatsys{0.985}{0.004}{0.009}$ & $\effstatsys{0.987}{0.004}{0.004}$ & $\effstatsys{0.974}{0.009}{0.006}$ \\ \hline
	$\numrange{1.566}{2.0}$    & $\effstatsys{0.988}{0.003}{0.012}$ & $\effstatsys{0.993}{0.002}{0.008}$ & $\effstatsys{0.992}{0.001}{0.004}$ & $\effstatsys{0.990}{0.003}{0.004}$ \\ \hline
	$\numrange{2.0}{2.5}$      & $\effstatsys{1.002}{0.004}{0.012}$ & $\effstatsys{1.004}{0.002}{0.008}$ & $\effstatsys{1.005}{0.002}{0.004}$ & $\effstatsys{0.998}{0.004}{0.004}$ \\ \hline
    \end{tabular}
\end{center}
\caption{
    Scale factors for GSF electron reconstruction. The upper uncertainty listed
    is statistical, the lower is systematic.
}
\label{table:gsf_scale_factor}
\end{table}

\subsection{Electron Identification}

\begin{table}[h]
\centering
\begin{center}
    \begin{tabular}{ | c | c | c | c | c |} \hline
	$|\eta|$                   & 20--30 \GeV               & 30--40 \GeV & 40--50 \GeV               & 50--200 \GeV \\ \hline
	$\numrange{0.0}{0.8}$      & $0.960_{-0.003}^{+0.003}$ & $0.978_{-0.001}^{+0.001}$ & $0.981_{-0.001}^{+0.001}$ & $0.982_{-0.002}^{+0.002}$ \\ \hline
	$\numrange{0.8}{1.4442}$   & $0.936_{-0.004}^{+0.004}$ & $0.958_{-0.002}^{+0.002}$ & $0.969_{-0.001}^{+0.001}$ & $0.969_{-0.002}^{+0.002}$ \\ \hline
	$\numrange{1.4442}{1.566}$ & $0.933_{-0.017}^{+0.015}$ & $0.907_{-0.008}^{+0.008}$ & $0.904_{-0.004}^{+0.004}$ & $0.926_{-0.011}^{+0.011}$ \\ \hline
	$\numrange{1.566}{2.0}$    & $0.879_{-0.007}^{+0.007}$ & $0.909_{-0.003}^{+0.003}$ & $0.942_{-0.002}^{+0.002}$ & $0.957_{-0.004}^{+0.004}$ \\ \hline
	$\numrange{2.0}{2.5}$      & $0.974_{-0.004}^{+0.004}$ & $0.987_{-0.004}^{+0.004}$ & $0.991_{-0.003}^{+0.003}$ & $0.999_{-0.005}^{+0.005}$ \\ \hline
    \end{tabular}
\end{center}
\caption{
    Scale factors for \EGTIGHT electron ID.
}
\label{table:tight_scale_factor}
\end{table}

\begin{table}[h]
\centering
\begin{center}
    \begin{tabular}{ | c | c | c | c | c |} \hline
	$|\eta|$                   & 20--30 \GeV               & 30--40 \GeV & 40--50 \GeV               & 50--200 \GeV \\ \hline
	$\numrange{0.0}{0.8}$      & $0.986_{-0.001}^{+0.002}$ & $1.002_{-0.001}^{+0.001}$ & $1.005_{-0.001}^{+0.001}$ & $1.004_{-0.001}^{+0.001}$ \\ \hline
	$\numrange{0.8}{1.4442}$   & $0.959_{-0.003}^{+0.003}$ & $0.980_{-0.001}^{+0.001}$ & $0.988_{-0.001}^{+0.001}$ & $0.988_{-0.002}^{+0.002}$ \\ \hline
	$\numrange{1.4442}{1.566}$ & $0.967_{-0.013}^{+0.007}$ & $0.950_{-0.007}^{+0.006}$ & $0.958_{-0.005}^{+0.005}$ & $0.966_{-0.009}^{+0.009}$ \\ \hline
	$\numrange{1.566}{2.0}$    & $0.941_{-0.005}^{+0.005}$ & $0.967_{-0.003}^{+0.003}$ & $0.992_{-0.002}^{+0.002}$ & $1.000_{-0.003}^{+0.003}$ \\ \hline
	$\numrange{2.0}{2.5}$      & $1.020_{-0.003}^{+0.003}$ & $1.021_{-0.003}^{+0.003}$ & $1.019_{-0.002}^{+0.002}$ & $1.022_{-0.004}^{+0.004}$ \\ \hline
    \end{tabular}
\end{center}
\centering
\caption{
    Scale factors for \EGMEDIUM electron ID.
}
\label{table:medium_scale_factor}
\end{table}

\TODO{http://cms.cern.ch/iCMS/jsp/db\_notes/noteInfo.jsp?cmsnoteid=CMS\%20AN-2014/055}

\subsection{Single Electron Trigger}

The efficiency of the HLT trigger used in this analysis,
\SingleElectronTrigger, is measured using \TnP on the primary dataset. The
efficiency is measured in bins of probe \pt and probe $\eta$ with bin
boundaries of \{30, 40, 50, 70, 250\} in \pt and \{-2.1, -2.0, -1.556, -1.442,
-0.8, 0., 0.8, 1.442, 1.556, 2.0, 2.1\} in $\eta$.

Both the tag electron and the probe electron are required to satisfy $|\eta| <
2.1$, $\pt > 30$, and to pass \EGTIGHT requirements. It is required that the
pair have an invariant mass $60 < \mee < 120$. The tag electron is required to
be matched to an electron that fired the trigger with $\Delta R < 0.3$. There
is no requirement placed on the charge of the electron pair. Events with three
or more electrons are rejected.

Probes are considered passing if they are also matched to an electron that
fired the trigger with $\Delta R < 0.3$, and failing otherwise. The efficiency
in each bin is the number of passing probes divided by the number of failing
probes. In an individual event, both electrons are tried as a tag so that an
event may contribute to the efficiency measurement twice if both electrons pass
the tag requirements.

The efficiency is computed in exactly the same way on the \MADGRAPH sample. The
MC events are reweighted for pileup, reconstruction efficiency, and
identification efficiency before the efficiency is measured. The measured
efficiencies for data and MC are listed in \TABS~\ref{trigger_eff_data} and
\ref{trigger_eff_mc}, respectively.

In the case of the trigger, because either electron could cause the event to
pass, the SFs can not be computed for each bin, but instead must be computed
for each pair of bins. If only one electron in the event has $\pt > 30$ and
$|\eta| < 2.1$ than the SF is simply that given by \EQ~\ref{eq:sf}, but if both
electrons pass those requirements than either could have fired the trigger, and
the SF is given by:

\begin{equation}
    \label{eq:sf_double}
    \text{SF}_{1\text{ or }2} = \frac{1 - \left( \effdata_{0} \right) \left(
    \effdata_{1} \right)} {1 - \left( \effmc_{0} \right) \left( \effmc_{1}
    \right)}
\end{equation}

Where $\effdata_{0,1}$ is the efficiency as measured in data for the 0th and
1st electrons, and $\effmc_{0,1}$ is the efficiency as measured in MC.
\EQ~\ref{eq:sf_double} is just the probability that one or both of the
electrons fired the trigger divided by the same quantity in MC. This equation
assumes that the probability of one electron firing the trigger is uncorrelated
with the probability of the other electron firing the trigger.

% Data
\begin{table}[h]
\begin{center}
    \begin{tabular}{ | c | c | c | c | c | c |} \hline
	$\eta$ & 30---40 \GeV & 40---50 \GeV & 50---70 \GeV & 70---250 \GeV  \\ \hline
	$\numrange{-2.1}{-2}$ & $0.741^{+0.003}_{-0.003}$ & $0.773^{+0.003}_{-0.003}$ & $0.780^{+0.005}_{-0.005}$ & $0.79^{+0.01}_{-0.01}$  \\ \hline
	$\numrange{-2}{-1.556}$ & $0.734^{+0.001}_{-0.001}$ & $0.772^{+0.001}_{-0.001}$ & $0.786^{+0.002}_{-0.002}$ & $0.792^{+0.005}_{-0.005}$  \\ \hline
	$\numrange{-1.556}{-1.442}$ & $0.725^{+0.003}_{-0.003}$ & $0.821^{+0.002}_{-0.002}$ & $0.809^{+0.004}_{-0.004}$ & $0.848^{+0.010}_{-0.010}$  \\ \hline
	$\numrange{-1.442}{-0.8}$ & $0.8930^{+0.0005}_{-0.0005}$ & $0.9396^{+0.0003}_{-0.0004}$ & $0.9509^{+0.0006}_{-0.0006}$ & $0.966^{+0.001}_{-0.001}$  \\ \hline
	$\numrange{-0.8}{0}$ & $0.9213^{+0.0004}_{-0.0004}$ & $0.9528^{+0.0002}_{-0.0002}$ & $0.9601^{+0.0004}_{-0.0004}$ & $0.9692^{+0.0010}_{-0.0010}$  \\ \hline
	$\numrange{0}{0.8}$ & $0.9174^{+0.0004}_{-0.0004}$ & $0.9473^{+0.0003}_{-0.0003}$ & $0.9561^{+0.0004}_{-0.0004}$ & $0.963^{+0.001}_{-0.001}$  \\ \hline
	$\numrange{0.8}{1.442}$ & $0.8964^{+0.0005}_{-0.0005}$ & $0.9424^{+0.0003}_{-0.0003}$ & $0.9533^{+0.0006}_{-0.0006}$ & $0.966^{+0.001}_{-0.001}$  \\ \hline
	$\numrange{1.442}{1.556}$ & $0.714^{+0.003}_{-0.003}$ & $0.823^{+0.002}_{-0.002}$ & $0.827^{+0.004}_{-0.004}$ & $0.861^{+0.009}_{-0.010}$  \\ \hline
	$\numrange{1.556}{2}$ & $0.758^{+0.001}_{-0.001}$ & $0.800^{+0.001}_{-0.001}$ & $0.811^{+0.002}_{-0.002}$ & $0.823^{+0.005}_{-0.005}$  \\ \hline
	$\numrange{2}{2.1}$ & $0.764^{+0.003}_{-0.003}$ & $0.792^{+0.002}_{-0.002}$ & $0.797^{+0.005}_{-0.005}$ & $0.82^{+0.01}_{-0.01}$  \\ \hline
    \end{tabular}
\end{center}
\centering
\caption{
    Electron trigger efficiency in data.
}
\label{trigger_eff_data}
\end{table}

% MC
\begin{table}[h]
\begin{center}
    \begin{tabular}{ | c | c | c | c | c | c |} \hline
	$\eta$ & 30---40 \GeV & 40---50 \GeV & 50---70 \GeV & 70---250 \GeV  \\ \hline
	$\numrange{-2.1}{-2}$ & $0.734^{+0.004}_{-0.004}$ & $0.769^{+0.004}_{-0.004}$ & $0.771^{+0.008}_{-0.008}$ & $0.76^{+0.02}_{-0.02}$  \\ \hline
	$\numrange{-2}{-1.556}$ & $0.736^{+0.002}_{-0.002}$ & $0.768^{+0.002}_{-0.002}$ & $0.779^{+0.003}_{-0.003}$ & $0.789^{+0.008}_{-0.008}$  \\ \hline
	$\numrange{-1.556}{-1.442}$ & $0.791^{+0.004}_{-0.004}$ & $0.847^{+0.003}_{-0.003}$ & $0.850^{+0.006}_{-0.006}$ & $0.87^{+0.01}_{-0.02}$  \\ \hline
	$\numrange{-1.442}{-0.8}$ & $0.9395^{+0.0006}_{-0.0006}$ & $0.9612^{+0.0004}_{-0.0004}$ & $0.9690^{+0.0007}_{-0.0008}$ & $0.980^{+0.002}_{-0.002}$  \\ \hline
	$\numrange{-0.8}{0}$ & $0.9469^{+0.0005}_{-0.0005}$ & $0.9670^{+0.0003}_{-0.0003}$ & $0.9745^{+0.0005}_{-0.0005}$ & $0.982^{+0.001}_{-0.001}$  \\ \hline
	$\numrange{0}{0.8}$ & $0.9466^{+0.0005}_{-0.0005}$ & $0.9665^{+0.0003}_{-0.0003}$ & $0.9739^{+0.0005}_{-0.0006}$ & $0.982^{+0.001}_{-0.001}$  \\ \hline
	$\numrange{0.8}{1.442}$ & $0.9364^{+0.0007}_{-0.0007}$ & $0.9597^{+0.0004}_{-0.0004}$ & $0.9668^{+0.0008}_{-0.0008}$ & $0.979^{+0.002}_{-0.002}$  \\ \hline
	$\numrange{1.442}{1.556}$ & $0.779^{+0.004}_{-0.005}$ & $0.841^{+0.003}_{-0.003}$ & $0.842^{+0.006}_{-0.006}$ & $0.86^{+0.02}_{-0.02}$  \\ \hline
	$\numrange{1.556}{2}$ & $0.749^{+0.002}_{-0.002}$ & $0.786^{+0.002}_{-0.002}$ & $0.798^{+0.003}_{-0.003}$ & $0.810^{+0.008}_{-0.008}$  \\ \hline
	$\numrange{2}{2.1}$ & $0.737^{+0.004}_{-0.004}$ & $0.769^{+0.004}_{-0.004}$ & $0.779^{+0.007}_{-0.008}$ & $0.82^{+0.02}_{-0.02}$  \\ \hline
    \end{tabular}
\end{center}
\centering
\caption{
    Electron trigger efficiency in \MADGRAPH MC.
}
\label{trigger_eff_mc}
\end{table}

\section{Electron Dressing}

Electrons emit bremsstrahlung photons as they bend in CMS's magnetic field.
Photons which impact ECAL near to the location of the electron impact are
summed into the electron's energy as described in
\SEC~\ref{sec:electron_reconstruction}. In order to more closely match the
generator quantities to the reconstructed electrons that we observe, a process
known as \dressing is applied to create \dressedelectrons.

The three definitions of generator electrons are as follows:

\begin{description}
    \item[Born] A generator electron immediately after the \Ztoee decay
    \item[Bare] A generator electron after it has shed all of its
        bremsstrahlung photons
    \item[Dressed] A bare electron, but with its bremsstrahlung photons added
        back in vector sum if they are within $\Delta R < 0.1$ of the electron
\end{description}

Generator level selection requirements are applied to dress electrons in this
analysis, for example, when selecting MC events to use for unfolding.
