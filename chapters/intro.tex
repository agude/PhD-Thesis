\chapter{Introduction}
\label{chapter:intro}

High energy particle physics is the study of the properties and interactions of
the basic building blocks of all matter. These properties and interactions are
described by the Standard Model, the best tested and most accurate scientific
theory to date. Even so, there are regions of the Standard Model where
calculations are difficult to perform, as is the case with low energy quantum
chromodynamics (QCD) interactions which are not calculable via perturbation
theory.

This thesis describes the measurement of a process which probes this region of
the Standard Model using \Z bosons decaying to electron pairs. These decays are
a very clean probe of QCD as neither the \Z nor the electrons carry color
charge and so all of the affect from QCD is isolated in the initial
interaction. These initial interactions sometimes give the \Z boson a non-zero
momentum transverse to the beamline (\bosonpt), which is what this thesis
measures. The novel variable \phistar is used in place of \bosonpt because it
is less susceptible to detector resolution effects and systematic uncertainties
while providing a probe of the same physics.

The data used in this thesis were collected in 2012 at the Compact Muon
Solenoid, one of four particle detectors at the Large Hadron Collider. In
total, \GoodLumiNumber of integrated luminosity was recorded at a
center-of-mass energy of \SI{8}{\TeV}, making this thesis the first measurement
of \phistar at that energy.

The final measurement is presented as a normalized differential cross section;
an additional absolute cross section measurement is also provided. These result
are compared to predictions from several sets of simulated events.

This thesis is organized as follows:

\begin{description}

    \item[\Cref{chapter:theory}] presents the history of the Standard Model and
        the motivation for the measurement.

    \item[\Cref{chapter:experiment}] gives an overview of the Large Hadron
        Collider (LHC) and the Compact Muon Solenoid (CMS).

    \item[\Cref{chapter:reconstruction}] describes the way in which electrons
        are measured at CMS.

    \item[\Cref{chatper:data_and_mc_samples}] introduces the simulated data
        samples and the actual data samples used in the measurement, as well as
        the scale factors used to correct the simulated data.

    \item[\Cref{chapter:event_selection}] discusses the method of selecting
        events and correcting them for various detector effects.

    \item[\Cref{chapter:analysis}] presents the analysis including the
        uncertainties, the final results of the measurement, and a short
        discussion on the results.

    \item[\Cref{app:dressed_measurements}] presents additional measurements.

    \item[\Cref{app:uncertainty_tables}] contains tables detailing the
        uncertainties on the final measurements.

    \item[\Cref{app:qcd_fits}] shows the plots from the \QCDjets and \wjets
        background fits.

    \item[\Cref{app:glossary}] reviews the terms and acronyms used in this
        thesis.

\end{description}
