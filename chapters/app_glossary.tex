\chapter{Glossary and Acronyms}
\label{app:glossary}

Every occupation develops its own jargon, and while this aides in communication
between members of the group, it often hinders the understanding of those
unfamiliar with it. Where possible, jargon has been minimized, but as with any
technical publication, some is unavoidable. To aid the understanding of the
reader, commonly used terms and acronyms have been defined below.

% Glossary
\section{Glossary}
\label{sec:gloassary}

\begin{itemize}

\GlossaryEntry{Barrel}{The central region in $\eta$ of the detector.}
\GlossaryEntry{Compact Muon Solenoid (CMS)}{The detector that collected the data used in this thesis. See \cref{sec:cms}.}
\GlossaryEntry{Endcap}{The portion of each subdetector that is flat and covers the high $|\eta|$ regions.}
\GlossaryEntry{Generator Level}{The information about an MC event as determined by the MC generator, before the event is passed through a detector simulation.}
\GlossaryEntry{Hadronization}{The process by which color charge is hidden from observation by producing colorless hadrons.}
\GlossaryEntry{Hadron}{Color neutral combinations of three quarks, for example protons and neutrons.}
\GlossaryEntry{Interaction Point}{The region at the center of the detector where proton-proton collisions occur.}
\GlossaryEntry{Jet}{A spray of high energy particles that originate from a colored object as it tries to maintain its colorless state.}
\GlossaryEntry{Large Hadron Collider (LHC)}{The collider used to produce the data used in this thesis. See \cref{sec:lhc}.}
\GlossaryEntry{\Moliere Radius}{The radius in which 90\% of the energy of an electromagnetic shower is contained within for a given material.}
\GlossaryEntry{Monte Calro (MC)}{Simulated data. See \cref{sec:mc}.}
\GlossaryEntry{Particle Flow}{An algorithm for reconstruction particles using information from multiple subdetectors. See \cref{ssec:iso}.}
\GlossaryEntry{Parton}{The individual constituents of a proton including the valence quarks, gluons, and sea quarks. See \cref{ssec:parton_model}.}
\GlossaryEntry{Pileup}{Additional proton-proton interactions which occur during an event.}
\GlossaryEntry{Prescaled}{To reduce the rate of a trigger by randomly throwing out events the trigger accepted.}
\GlossaryEntry{Primary Vertex}{The reconstructed location of the proton-proton interaction.}
\GlossaryEntry{Reconstructed Level}{The information about an event determined from the detector (for data) or the simulation of the detector response (in MC).}
\GlossaryEntry{Reconstruction}{The process of taking raw data from the detector and creating objects useful for physics. See \cref{chapter:reconstruction}.}
\GlossaryEntry{Sea Quarks}{Pairs of quarks and antiquarks from gluon splitting that exist within each hadron.}
\GlossaryEntry{Simulation}{See \textit{Monte Carlo}.}
\GlossaryEntry{Transverse Momentum}{Momentum transverse to the beamline. In general denoted \pt, but \bosonpt is used specifically to mean the transverse momentum of the \Z or \W boson.}
\GlossaryEntry{Trigger}{A system that analyses events as they are happening and decides which ones to keep. See \cref{ssec:trigger}.}
\GlossaryEntry{Truth Level}{See \textit{Generator Level}.}
\GlossaryEntry{Valence Quark}{The quarks which give rise to the quantum numbers of the proton, specifically the two up quarks and the down quark.}

\end{itemize}

% Acronyms
\section{Acronyms}
\label{sec:acronyms}

% Table formatting

% Heading for the first page
\begin{longtable}{p{0.25\textwidth} p{0.75\textwidth}}
\caption{Acronyms} \label{tab:acronyms} \\

\toprule
Acronym & Meaning \\
\midrule
\endfirsthead

% Heading for all subsequent pages
\multicolumn{2}{l}{\textit{\tablename\ \thetable{} -- Continued from previous page}} \\
\toprule
Acronym & Meaning \\
\midrule
\endhead

% Footer for each page that wraps over to the next
\multicolumn{2}{r}{\textit{Continued on next page}} \\
\bottomrule
\endfoot

% Footer for the end of the table
\bottomrule
\endlastfoot

% End table formatting

\AcronymEntry{CB}{Crystal ball}
\AcronymEntry{CERN}{Originally from \textit{Conseil Europ\'{e}en pour la Recherche Nucl\'{e}aire}, now the European Organization for Nuclear Research}
\AcronymEntry{CMS}{Compact Muon Solenoid}
\AcronymEntry{CSC}{Cathode stripe chambers}
\AcronymEntry{DAQ}{Data acquisition}
\AcronymEntry{DT}{Drift tubes}
\AcronymEntry{EB}{Electromagnetic calorimeter barrel}
\AcronymEntry{ECAL}{Electromagnetic calorimeter}
\AcronymEntry{EE}{Electromagnetic calorimeter endcap}
\AcronymEntry{ES}{Electromagnetic calorimeter preshower}
\AcronymEntry{FNAL}{Fermi National Accelerator Laboratory}
\AcronymEntry{FSR}{Final state radiation}
\AcronymEntry{GSF}{Gaussian-sum filter}
\AcronymEntry{HB}{Hadronic calorimeter barrel}
\AcronymEntry{HB}{Hadronic calorimeter endcap}
\AcronymEntry{HCAL}{Hadronic calorimeter}
\AcronymEntry{HF}{Forward hadronic calorimeter}
\AcronymEntry{HLT}{High-level trigger}
\AcronymEntry{HO}{Hadronic calorimeter outer}
\AcronymEntry{ID}{Electron identification}
\AcronymEntry{ISR}{Initial state radiation}
\AcronymEntry{L1}{\Lone trigger}
\AcronymEntry{LHC}{Large Hadron Collider}
\AcronymEntry{LO}{Leading order}
\AcronymEntry{MC}{Monte Carlo}
\AcronymEntry{NLO}{Next-to-leading order}
\AcronymEntry{NNLO}{Next-to-next-to-leading order}
\AcronymEntry{PDF}{Parton distribution function}
\AcronymEntry{PSB}{Proton synchrotron booster}
\AcronymEntry{PS}{Proton synchrotron}
\AcronymEntry{QCD}{Quantum Chromodynamics}
\AcronymEntry{QED}{Quantum Electrodynamics}
\AcronymEntry{RPC}{Resistive place chambers}
\AcronymEntry{SPS}{Super Proton Synchrotron}
\AcronymEntry{TEC}{Tracker endcap}
\AcronymEntry{TIB}{Tracker inner barrel}
\AcronymEntry{TID}{Tracker inner disk}
\AcronymEntry{TOB}{Tracker outer barrel}

\end{longtable}
