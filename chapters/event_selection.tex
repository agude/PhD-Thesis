\chapter{Event Selection}
\label{event_selection_chapter}

This chapter details the requirements used to select events for the analysis.
It also covers the data used and the Monte Carlo (MC) used. The final state we
are considering is \Ztoee.

\section{Data and Monte Carlo}

\subsection{Data}

\TODO{Figure of lumis \label{fig:2012_luminosity}}

The data used in this analysis were collected by the CMS detector in 2012 at a
center of mass energy of \rootseight. The LHC delivered 23 \fbinv of integrated
luminosity during the year as seen in \FIG~\ref{fig:2012_luminosity}. This
period was divided into four eras called 2012A, B, C, and D. During an era, the
LHC run parameters are kept roughly static to allow for consistent data taking
conditions. In between eras, maintenance and minor upgrades are performed on
the LHC in order to deliver higher luminosity. After accounting for detector
dead time and beam quality, \GoodLumiNumber of integrated luminosity are used
for physics analysis.

The data collected by CMS are split into smaller datasets based on the physics
objects contained within the events. This allows analyses to use only one or
two datasets, instead of requiring them to deal with the entirety of the CMS
data (which is petabyte scale, and hence too large for most institutes to store
locally). The HLT sorts events into the various datasets based on the triggers
that the event fired. In this manner, and event can end up in multiple datasets
if it fired multiple triggers. This analysis uses the \SingleElectron dataset
which was collected with the HLT trigger \SingleElectronTrigger. These datasets
were reconstructed---converted from raw detector response into physics
objects---in January, 2013, in order to make use of the most recent
calibrations derived from the entire 2012 run. A summary of the datasets used
are listed in \TAB~\ref{table:datasets}.

\begin{table}[h]
\centering
\begin{center}
    \begin{tabular}{ | l | c | c |}
    \hline
	Dataset Name                          & Run Range      & Luminosity       \\ \hline
	/SingleElectron/Run2012A-22Jan2013-v1 & 190456--193621 & $889.362 \pbinv$ \\ \hline
	/SingleElectron/Run2012B-22Jan2013-v1 & 193833--196531 & $4.429 \fbinv$   \\ \hline
	/SingleElectron/Run2012C-22Jan2013-v1 & 198022--203742 & $7.152 \fbinv$   \\ \hline
	/SingleElectron/Run2012D-22Jan2013-v1 & 203777--208686 & $7.318 \fbinv$   \\ \hline
    \end{tabular}
\end{center}
\caption{
    The datasets used in this analysis.
}
\label{table:datasets}
\end{table}

Although there is a \DoubleElectron dataset which uses a trigger designed to
find Z bosons, this analysis uses the \SingleElectron dataset selected with the
\SingleElectronTrigger trigger. The primary motivation behind using this
trigger was to allow a direct comparison with a similar \phistar analysis being
performed by CMS which used \Ztomumu events selected with a single muon
trigger. The single electron trigger requires an electron with $\pt > 27$ which
passes Working Point 80 (\WPEighty), a set of cuts on lepton isolation and
shower shape designed to be 80\% efficient on electrons. The cuts that make up
\WPEighty are listed in \TAB~\ref{table:wp80}. This trigger had the lowest \pt
threshold of any single electron trigger that was unprescaled run during 2012.
To prescale a trigger means to apply a rate reduction by randomly throwing out
a certain fraction of events in order to keep the total trigger rate
manageable; as this trigger was unprescaled, no events were discarded in this
manner.

\begin{table}[h]
\centering
\label{table:wp80}
\begin{center}
    \begin{tabular}{ | c | c | c |} \hline
        Value                      & EB     & EE     \\ \hline
        $|\eta| <$                 & 1.4791 & 2.65   \\
        $\pt >$                    & 27     & 27     \\
        $\sigmaietaieta <$         & 0.1    & 0.03   \\
        ECAL Iso $/ \et <$         & 0.15   & 0.1    \\
        $H/E <$                    & 0.1    & 0.05   \\
        HCAL Iso $/ \et <$         & 0.1    & 0.1    \\
        Pixel Matching $\ge$       & 1      & 1      \\
        $|\ooeoop| <$              & 0.05   & 0.05   \\
        $|\Delta \eta| <$          & 0.007  & 0.007  \\
        $|\Delta \phi| <$          & 0.06   & 0.03   \\ \hline
    \end{tabular}
\end{center}
\caption{
    The selection requirements for the \SingleElectronTrigger trigger for
    electrons which end up in the barrel region or the endcap region of ECAL.
}
\end{table}

\TODO{Put variables in an electron reconstruction chapter.}
