\chapter{Physics of Z Transverse Momentum}
\label{chapter:theory}

\section{The Standard Model}
\label{section:standard_model}

Our current understanding of how matter interacts at high energy is entirely
described by the standard model (SM) as constructed by Weinberg, Glashow, and
Salam \cite{glashow1961}\cite{weinberg1967}\cite{salam1968}. The model combines
three of the four fundamental forces (leaving out only gravity, which is so
weak as to be negligible) and explains almost everything we see in the
universe.

\subsection{The Electromagnetic Force}
\label{subsection:electronmagnetic_force}

The modern theory of electromagnetism began with Maxwell's theory developed in
the middle of the 19th century \cite{maxwell1863}. Maxwell was the first to
conclude that light was an electromagnetic wave, the full importance of which
was only later understood when it was discovered that the photon was the force
carrier of the electromagnetic force \cite{maxwell1864}. 

In the early 20th century, Lorentz and Einstein developed relativistic
mechanics and showed that Maxwell's theory was Lorentz invariant
\cite{lorentz1899}\cite{einstein1904}. Dirac updated the theory in 1920 when he
was able to quantize the electromagnetic field as an ensemble of harmonic
oscillators \cite{dirac1927}. Dirac would go on to discover that anti-particles
were a natural consequences of his equations \cite{dirac1928}\cite{dirac1930}.
These anti-particles were found by Anderson in 1932 as he observed cosmic rays
in a cloud chamber \cite{anderson1933}.

As microwave technology improved in the 1940s, more accurate measurements of
the energy level shifts in hydrogen were made, resulting in the discovery of
the Lamb shift by Lamb and Rutherford \cite{lamb1947}. This discovery pointed
to discrepancies in the theory, discrepancies which Bethe would explain after
completing a set of non-relativistic calculations using it \cite{bethe1947}.
Bethe's work inspired multiple other physicist including Dyson, Feynman,
Schwinger, and Tomonaga to work along similar lines. They created quantum
electrodynamics, a fully relativistic and self-consistent theory of
electromagnetic interactions
\cite{tomonaga1946}\cite{schwinger1948}\cite{feynman1949}\cite{dyson1949}.

\subsection{The Weak Force}
\label{subsection:weak_force}

The need for a weak force, and hence a theory describing it, was first hinted
at by beta decay experiments in the 1911 performed by Meitner and
Hahn showed that the energy spectrum of the electron ejected in beta decay was
continuous instead of a delta function as would be expected for a two-body
decay .\TODO{Cite} While some proposed that this discovery indicated that
momentum and energy were not conserved, Pauli proposed an alternative: there
was a neutral and invisible particle that carried away some of the energy---the
neutrino \cite{pauli1930}. Fermi began working on this idea and invented a
four fermion contact interaction in which a neutron decayed into a proton,
an electron, and a neutrino \cite{fermi1934}.

In 1947, Rochester and Butler discovered a particle that decayed to two pions
which they called the $\theta$; in 1949, Brown and Powell discovered a particle
that decayed to three pions which they called the $\tau$
\cite{Rochester1947}\cite{brown1949}. It was soon discovered that these
particles had the same mass and lifetime---indicating that they were the same
particle---but, based on their decay products, they must have different parity.
Lee and Yang proposed that perhaps there was only one particle undergoing a  a
parity violating decay \cite{lee1956}. Their idea was confirmed by Wu in 1956, who
showed that electrons were preferential emitted from \cobaltsixty in one
direction, and by Garwin, Lederman, and Weinrich in 1957, who studied \pitomunu
in a storage ring \cite{wu1956}\cite{garwin1957}.
\subsection{The Strong Force}
\label{subsection:Strong_force}
