\chapter{Physics of \texorpdfstring{\Z}{Z} Transverse Momentum}
\label{chapter:theory}

\section{The Standard Model}
\label{section:standard_model}

Our current understanding of how matter interacts at high energies is entirely
described by the Standard Model as constructed by Weinberg, Glashow, and Salam
\cite{glashow1961,weinberg1967,salam1968}. The model combines three of the four
fundamental forces (leaving out only gravity, which is so weak as to be
negligible) and is the most accurate scientific theory ever formulated.

\subsection{The Electromagnetic Force}
\label{subsection:electronmagnetic_force}

The modern theory of electromagnetism began with Maxwell's theory developed in
the middle of the 19th century \cite{maxwell1873}. Maxwell was the first to
conclude that light was an electromagnetic wave, the full importance of which
was only later understood when it was discovered that the photon was the force
carrier of the electromagnetic force \cite{maxwell1865}.

In the early 20th century, Lorentz and Einstein developed relativistic
mechanics and showed that Maxwell's theory was Lorentz invariant
\cite{lorentz1899,einstein1904}. Dirac updated the theory in 1920 when he was
able to quantize the electromagnetic field as an ensemble of harmonic
oscillators \cite{dirac1927}. Dirac would go on to discover that anti-particles
were a natural consequences of his equations \cite{dirac1928,dirac1930}. These
anti-particles were found by Anderson in 1932 as he observed cosmic rays in a
cloud chamber \cite{anderson1933}.

As microwave technology improved in the 1940s, more accurate measurements of
the energy level shifts in hydrogen were made, resulting in the discovery of
the Lamb shift by Lamb and Rutherford \cite{lamb1947}. This shift was not
immediately explainable, but Bethe would resolve this discrepancies by showing
that the theory could account for the Lamb shift using non-relativistic
calculations \cite{bethe1947}. Bethe's work inspired multiple other physicist
including Dyson, Feynman, Schwinger, and Tomonaga to work along similar lines.
They created quantum electrodynamics (QED), a fully relativistic and
self-consistent theory of electromagnetic interactions
\cite{tomonaga1946,schwinger1948,feynman1949,dyson1949}.

QED is a perturbation theory with expansions performed in terms of the fine
structure constant, \fsc. As $\fsc \approx 7.297 \cdot 10^{-3}$, the higher
order terms contribute smaller and smaller corrections and so only a few orders
need to be computed to make very accurate predictions. For example, the
anomalous magnetic moment of the electron is calculated to $\BigO{\fsc^{4}}$,
and agrees with experiment to more than \num{10} significant figures.

\subsection{The Weak Force}
\label{subsection:weak_force}

The need for a weak force, and hence a theory describing it, was first hinted
at by beta decay experiments in the early 1900s. These experiments culminated,
in 1914, in Chadwick's discovery that the energy spectrum of electrons ejected
in beta decay was continuous instead of a delta function as would be expected
for a two-body decay \cite{chadwick1914}. While some proposed that this
discovery indicated that momentum and energy were not conserved, Pauli proposed
an alternative: that there was a neutral and invisible particle that carried
away some of the energy \cite{pauli1930}. This particle was the neutrino. Fermi
began working on this idea and invented a four fermion contact interaction in
which a neutron decayed into a proton, an electron, and a neutrino
\cite{fermi1934}.

In 1947, Rochester and Butler discovered a particle that decayed to two pions
which they called the $\theta$; in 1949, Brown and Powell discovered a particle
that decayed to three pions which they called the $\tau$
\cite{Rochester1947,brown1949}. It was soon discovered that these particles had
the same mass and lifetime---indicating that they were the same particle---but,
based on their decay products, they must have different parity. Lee and Yang
proposed that the $\theta$ and the $\tau$ were in fact the same particle, but
that it was undergoing a parity violating decay \cite{lee1956}. Their idea was
confirmed by Wu in 1956, who showed that electrons were preferential emitted
from \cobaltsixty in one direction, and by Garwin, Lederman, and Weinrich in
1957, who studied \pitomunu decays in a storage ring \cite{wu1956,garwin1957}.

In 1954, Yang and Mills replaced Fermi's contact interaction with a non-Abelian
gauge theory that contained a spin-1 boson to mediate the force
\cite{yang1954}. However, this boson was massless, which if true would have
given the weak force infinite range, a feature that was not observed. In 1960,
Glashow was able to modify Yang and Mill's theory by adding Sudarshan and
Marshak's vector minus axial ($V-A$) model to produce a unified electroweak
force described by the \SUtwoUone gauge group \cite{glashow1961,sudarshan1958}.
\SUtwo is a left-handed interaction and so violates parity as expected.
Weinberg and Salamn finished up the model in 1967 when they added the Brout,
Englert, and Brout--Englert--Higgs mechanism which gave the vector bosons mass
and so explained the short ranged nature of the weak interaction
\cite{weinberg1967,salam1968,englert1964,higgs1964}.

\subsection{The Strong Force}
\label{ssec:strong_force}

The theory of the strong force grew out of studies of atomic nuclei. In 1911,
Rutherford discovered that that atomic nucleus was a compact, positively
charged object \cite{rutherford1911}. In 1917, Rutherford showed that larger
nuclei were composed of hydrogen nuclei and so discovered the proton
\cite{rutherford1919}. The discovery of the uncharged neutron in 1932 by
Chadwick indicated that the atomic nucleus was made up of multiple types of
nucleons, and that it could not be held together by the electromagnetic force
\cite{chadwick1932}. In 1934, Yukawa---having noted that Fermi's contact
interaction was too weak to hold nuclei together---tried to explain this
nuclear force using meson exchange \cite{yukawa1935}. In 1947, Lettes,
Occhialini, and Powell discovered the pion which seemed to confirm Yukawa's
theory \cite{lattes1947}.

In the 1950s and early 1960s, dozens of new mesons were discovered, indicating
the need for a new theory. Some of these new mesons seemed to have a new type
of quantum number that limited their available decays, leading them to be
called ``strange'' particles. An effort to explain these particles lead to the
Gell-Mann--Nishijima formula \cite{nakano1953,nishijima1955,gellmann1956}. This
theory lead Gell-Mann and \Neeman to come up with a classification scheme for
mesons and baryons based on the \SUthree group which Gell-Mann named the
Eightfold Way. In 1964, Gell-Mann and Zweig realized that the Eightfold Way
implied that mesons were composed of sub-atomic particles which became known as
quarks \cite{gellmann1964,zweig1964}. Their model explained both isospin and
the ``strangeness'' observed earlier as a consequence of the existence of three
quarks, the up, down, and strange. Their model was modified in 1973 by Gross,
Wilczek, and Politzer to include asymptotic freedom in which quarks interact
weakly at high energies but strongly at low energies
\cite{gross_1973,politzer_1973}. This new combined model became known as
quantum chromodynamics (QCD). QCD was used to predict the existence of the
charm quark and, upon its success, was incorporated with electroweak theory to
form the full \SUthreeSUtwoUone symmetry group of the Standard Model.

QCD, like QED, can also be expanded in terms of a coupling constant,
\alphastrong. However, unlike \fsc in QED, the value of \alphastrong is
dependent on the energy of the interaction as follows:

\begin{equation}\label{eq:alpha_strong}
    \alphastrong \left( Q \right)
    \approx
    \frac{1}{
        \logn \left( Q / \LambdaQCD \right)
    }
\end{equation}

where \LambdaQCD is the QCD scale and $Q$ is the interaction energy. This means
that QCD can be expanded perturbatively for large $Q$, but not for low energy
interactions.

\subsection{Experimental Verification}

The Standard Model has has made numerous predictions which have been borne out
by experiment. The neutral current interaction was observed by the Gargemelle
experiment at CERN in 1973 shortly after it was predicted by Weinberg, Glashow,
and Salam \cite{hasert1973}. This confirmation of their theory won the trio the
Nobel Prize in 1979.

The remaining quarks were found over the next twenty years. The charm was found
at the Stanford Linear Accelerator Center and Brookhaven National Laboratory in
1974 via \jpsi decays \cite{aubert1974,augustin1974}. The bottom was discovered
at Fermi National Accelerator Laboratory (FNAL) in 1977 by the E288 experiment
\cite{herb1977}. The final quark, the incredibly heavy top, had to wait until
1995 to be discovered by the CDF and D0 experiments running at the Tevatron at
FNAL \cite{cdf1995,d01995}.

The \W and \Z bosons were discovered at CERN in 1983 at the UA1 and UA2
experiments running on the Super Proton Synchrotron
\cite{ua1_w,ua2_w,ua1_z,ua2_z}. These bosons were an excellent test of the
Standard Model as their masses could be very exactly calculated, and the newly
discovered bosons had masses that matched the calculations to high precision.
The final piece of the Standard Model, the Higgs bosons, was discovered by
\ATLAS and CMS in 2012 using the LHC at CERN \cite{atlas_higgs,cms_higgs}.
Although precision measurements of the Higgs are still ongoing, it so far
precisely matches the predictions of the Standard Model.

\subsection{Components of the Standard Model}

\begin{figure}[!htbp]
    \centering
    \includegraphics[width=\textwidth]{figures/standard_model.pdf}
    \caption[
        The particles of the Standard Model.
    ]{
        The particles of the Standard Model with information about their
        type, their mass, charge, and spin. The quarks and leptons make up
        matter, while the gauge bosons mediate interactions. The Higgs gives
        mass to the \W and \Z bosons.
    }
    \label{fig:standard_model}
\end{figure}

There are two types of particles in the Standard Model: fermions---with half
integer spin---and bosons---with integer spin. The fermions, which make up all
of the matter in the universe, are further subdivided into two groups, leptons
and quarks, while the bosons are divided into gauge bosons and the Higgs.  The
various particles are show schematically in \cref{fig:standard_model}.

Leptons have \spinhalf and have charge $q=-1 \text{ or } 1$ for the massive
leptons, or 0 for the neutrinos. They interact electromagnetically (if they
have a non-zero charge) and weakly. There are three generations, or flavors, of
leptons and each generation consists of a charged, massive lepton, and an
uncharged, nearly massless neutrino. The lightest of the charged lepton
generations is the electron, which is stable. The next two generations contain
the muon and the tau, which are unstable and eventually decay to electrons. The
tau, being very heavy, decays quickly, while the muon is stable long enough to
escape a particle detector. Massive leptons are important in particle detection
as they provide a very clean decay signature. Neutrinos are very difficult to
detect in particle detectors, and so their presence is inferred from the
missing energy in the vector sum of all particles in the collision. This
analysis uses electrons to make its measurement.

Quarks also have \spinhalf although, unlike the leptons, their charge is
fractional and so takes values of $q = 2/3 \text{ or } -1/3$. They interact
strongly, electromagnetically, and weakly. There are three generations of
quarks, with each successive generation having higher mass constituents. There
are six flavors of quarks, with each generation containing two. The first
generation consists of the up and down (\upquark and \downquark) quarks. These
quarks are stable and make up protons and neutrons, as well as pions. The next
generation of quarks contains the charm and the strange (\charmquark and
\strangequark). They form heavier states that decay quickly like the \jpsi and
kaons. The final generation consist of the heavy bottom (\bottomquark), and the
extraordinarily heavy top (\topquark)---the most massive particle in the Standard
Model. Bottom quarks can form bound states, but top quarks are so heavy they
decay before any bound states can form.

Quarks carry color charge, of which there are three: red (\red), blue (\blue),
and green (\green). Strongly interacting objects obey confinement, which means
that only color neutral (colorless) states are allowed. Because of confinement,
quarks bind together into colorless composite particles. These particles are
called mesons---with two quarks (\qqbar)---and baryons---with three quarks
(\baryon). When an object containing quarks breaks up, the individual colored
fragments will create additional colored objects to remain color neutral. This
leads to the formation of ``jets'' which are sprays of high energy particles
that originate from one of these fragments as it tries to maintain its
colorless state.

Bosons are the second type of particle in the Standard Model. They are further
subdivided into gauge bosons---which mediate the three forces---and the Higgs
boson---which gives mass to the \W and \Z bosons.

The gauge boson that mediates the strong force is the gluon. Gluons interact
with objects that carry color, and are themselves carriers of color, allowing
gluons to interact not only with quarks but with other gluons. Gluons can have
any one of eight different possible color-anticolor superpositions that form a
color-octet. This number comes from the number of generators of \SUthree. Such
octets are not unique, but a commonly used definition is listed in
\cref{table:gluon_color}.

% table:gluon_color
\begin{table}[h]
    \centering
    \spacerows{1.2}
    \begin{center}
        \begin{tabular}{c  c}
            %\toprule
            $\left( \xxbar{\red}{\blue} + \xxbar{\blue}{\red} \right) / \sqrt{2}$ &
            $-i \left( \xxbar{\red}{\blue} - \xxbar{\blue}{\red} \right) / \sqrt{2}$ \\
            $\left( \xxbar{\red}{\green} + \xxbar{\green}{\red} \right) / \sqrt{2}$ &
            $-i \left( \xxbar{\red}{\green} - \xxbar{\green}{\red} \right) / \sqrt{2}$ \\
            $\left( \xxbar{\blue}{\green} + \xxbar{\green}{\blue} \right) / \sqrt{2}$ &
            $-i \left( \xxbar{\blue}{\green} - \xxbar{\green}{\blue} \right) / \sqrt{2}$ \\
            $\left( \xxbar{\red}{\red} - \xxbar{\blue}{\blue} \right) / \sqrt{2}$ &
            $\left( \xxbar{\red}{\red} - 2\xxbar{\blue}{\blue} + \xxbar{\green}{\green} \right) / \sqrt{6}$ \\
            %\bottomrule
        \end{tabular}
        \caption[
            One possible QCD color-octet.
        ]{
            One of the possible color-octets. The colors are red (\red), blue
            (\blue), green (\green), and their anticolors ($\overline{\red}$,
            $\overline{\blue}$, and $\overline{\green}$) .
        }
        \label{table:gluon_color}
    \end{center}
\end{table}


There are four gauge bosons that mediate the electroweak interaction: the
photon (\photon), the \Z, and the \Wpm. The photon and the \Z are uncharged,
while the \Wpm carry charge of $\pm1$. The \W and \Z are not the particles
described by the \SUtwoUone group, but are instead linear combinations of these
fields created through combination with the Brout--Englert--Higgs mechanism.
The \W participates in interactions that change quark and lepton flavor, for
example \tWdecay or \mutoWnu, while the \Z does not.

The \Z boson is an excellent probe of precision physics as its well measured
mass (\Zmass) and its sharp width (\Zwidth) make it easy to identify from its
decay products \cite{pdg2014}. In a hadron collider, the most common \Ztoqq
decay mode is difficult to select as there are many hadronic jets in each
event, and so \Ztoll decay modes are preferred. In this analysis we look at the
\Ztoee decay mode; our collaborators are working on performing a similar
measurement with \Ztomumu decays. A few common decay modes and their branching
fraction are listed in \cref{table:z_decays}.

% table:z_decays
\begin{table}[h]
    \centering
    \spacerows{1.2}
    \begin{center}
        \begin{tabular}{@{}l r@{}}
            \toprule
            Mode         & Fraction $\left( \Gamma_{i} / \Gamma \right)$ \\
            \midrule
            $\Ztoqq$     & $\SI[parse-numbers=false]{69.91 \pm 0.06}{\percent}$ \\
            $\Ztoee$     & $\SI[parse-numbers=false]{3.363 \pm 0.004}{\percent}$ \\
            $\Ztomumu$   & $\SI[parse-numbers=false]{3.366 \pm 0.007}{\percent}$ \\
            $\Ztotautau$ & $\SI[parse-numbers=false]{3.370 \pm 0.008}{\percent}$ \\
            $\Ztonunu$   & $\SI[parse-numbers=false]{20.00 \pm 0.06}{\percent}$ \\
            \bottomrule
        \end{tabular}
        \caption{
            Selected decay modes of the \Z boson.
        }
        \label{table:z_decays}
    \end{center}
\end{table}


The final piece of the Standard Model, the Higgs boson, gives mass to the weak
bosons through interactions with the Higgs field. In the Lagrangian of the
Standard Model it is not possible to write a mass term for the bosons that is
gauge invariant. The Brout--Englert--Higgs mechanism adds a complex scalar
field whose symmetry is spontaneously broken leading to the \W and \Z masses.
This rotates the primordial vector boson plane consisting of the massless
\PrimordialB and \PrimordialW fields into the \Z boson and the photon. The
angle of this rotation is called the Weinberg angle, \WeinbergAngle. This
symmetry breaking also produces a \spinzero boson, the Higgs.

The Weinberg angle is defined by the mass of the \Z and \W bosons as follows:

\begin{equation}
\label{eq:weinberg_angle}
    \sin \WeinbergAngle \equiv \frac{\MassW}{\MZ}
\end{equation}

This angle is present in many terms of the electroweak Lagrangian, and so
controls the rates of electromagnetic decays as well as the couplings of the
Higgs to the vector bosons. It can not be calculated and so must be measured
experimentally via the masses of the vector bosons.

\section{\texorpdfstring{\Z}{Z} Boson Transverse Momentum}

The leptonic decays of the \Z are an excellent probe of QCD as neither the \Z
nor the leptons carry color charge. This means that there is no color flow
between the initial and final states, and so the only QCD signature encoded in
the decay of the \Z is that of the initial interaction. One particularly useful
variable for probing QCD is the transverse momentum of the \Z boson, \bosonpt,
where transverse is defined relative to the beamline in the collider
\cite{bozzi_2011,mantry_2011,becher_2011}. As the beam protons have near-zero
momentum transverse to the beamline, the \bosonpt of the \Z boson comes from
QCD process like initial state radiation, which are discussed in
\cref{sec:z_production}. Low values of \bosonpt probe the non-perturbative
regions of QCD as, due to asymptotic freedom, it is these low momentum transfer
interactions where \alphastrong is large.

In addition to aiding in the understanding of QCD processes, measuring the \Z
\bosonpt spectrum helps improve measurements of the \W mass (\MassW). The \W
mass is an interesting quantity to measure because of the role it plays in
determining the Weinberg angle and because it can be calculated once all the
other inputs to the Standard Model are known. With the discovery of the Higgs,
it is now possible to compared the measured value of \MassW with the
theoretical value. Performing such a fit to determine \MassW leads to a theory
result that is better constrained then the experimental measurement
\cite{baak_2014}. There is some disagreement between the fit and the
measurement of \MassW, as shown in \cref{fig:gfitter_mw}, which could be
indicative of new physics, but the comparison is inconclusive; a better
measurement is needed to resolve the issue. % Current pull: 1.4

% fig:gfitter_mw
\begin{figure}[!htbs]
    \centering
    \includegraphics[width=\textwidth]{figures/gfitter_wm.pdf}
    \caption[
        %\cite{baak_2014}
        The measured value of \MassW compared to fit results from \GFitter.
    ]{
        The measured value (point) of \MassW compared to fit results from
        \GFitter (band). There is some tension between the measurement and the
        fit, but a better measurement is needed to determine if it is
        significant.
    }
    \label{fig:gfitter_mw}
\end{figure}


The \W mass is difficult to measure. In an electron collider like \LEP, \Z
bosons are easy to produce at rest as the \eetoZ channel is available allowing
a high precision mass measurement. Unfortunately, it is not possible to build a
high luminosity collider to produce \W bosons at rest through the \enutoW
channel, and so instead the \qqtoW channel is used at hadron colliders like the
LHC. However, because of the large numbers of hadronic jets at a hadron
collider, only the \Wtolnu decay modes can be selected, and so the mass must be
measured using the single observable lepton because the neutrino escapes
unseen. This is done with simulate date samples---colloquially referred to as
Monte Carlo (MC)---which are used to create a template. This template is then
used to fit for the mass to be done from lepton's momentum. As both the
\bosonpt of the \W and the \W mass affect the lepton's momentum, any
uncertainty in the \bosonpt distribution of the simulation translates to an
uncertainty on the final mass measurement. As the \W is subject to the same
processes described below for the \Z, the measurement of the \Z \bosonpt
spectrum constrains the \W \bosonpt spectrum.

\subsection{\texorpdfstring{\Z}{Z} Boson Differential Cross Section}

The measurement that will be performed in this analysis is a differential cross
section measurement. A cross section measurement is, at its heart, a simple
counting experiment. The cross section of a process, $\Z$ production for
instance, is given by:

\begin{equation}\label{eq:xsec}
    \sigma(\Z) = \frac{N_{\Z}}{\luminosity}
\end{equation}

where $N_{\Z}$ is the number of \Z bosons that were created and $\luminosity$
is the integrated luminosity of the data which is a measurement of how many
interaction opportunities took place. A differential cross section is used to
measure the cross section of a process as a function of a variable, for
instance \bosonpt. Then the cross section becomes:

\begin{equation}\label{eq:differ_xsec}
    \frac{\dir{\sigma(\Z)}}{\dir{\bosonptk} }
    =
    \frac{N_{\Z_{\bosonptk}}}{\Delta \bosonptk \cdot \luminosity}
\end{equation}

where $\bosonptk$ is a range of $\bosonpt$, $N_{\Z_{\bosonptk}}$ is the number
of \Z events with $\bosonpt$ that fall within the range, and $\Delta \bosonptk$
is size of the range. This measurement is still dependent on the luminosity
which can be difficult to accurately measure. This dependency can be removed by
dividing by the total cross section:

\begin{equation}\label{eq:normaized_differ_xsec}
    \frac{1}{\sigma(\Z)} \frac{\dir{\sigma(\Z)}}{\dir{\bosonptk} }
    =
    \frac{\luminosity}{N_{Z}} \frac{N_{\Z_{\bosonptk}}}{\Delta \bosonptk \cdot \luminosity}
    =
    \frac{N_{\Z_{\bosonptk}}}{\Delta \bosonptk \cdot N_{Z}}
\end{equation}

\subsection{The Proton Parton Model}
\label{ssec:parton_model}

At the very high energies of the LHC (\rootseight in 2012), a proton is not
well described by assuming that it is composed of only three valence quarks as
the energy of the proton is much higher than its own binding energy
($\ProtonMass \approx \SI{1}{\GeV}$). Instead, it is described by a parton
model---a model in which there are three valence quarks, but also numerous
gluons and ``sea quarks'', where the sea quarks are a superposition of quark
and antiquark same-flavor pairs. These constituents of the protons are referred
to as ``partons''. The parton model is formulated in the infinite momentum
frame, a frame where the proton has very high energy compared to its rest mass
($\Energy \gg \ProtonMass$). In this frame the mass of the proton can be
neglected.

When two protons collide at the LHC, it is a parton from each that interacts in
the hard scattering process. This parton-parton interaction can be considered
as independent of the other partons because the internal quantum state of a
proton in the infinite momentum frame is ``frozen'' due to relativistic time
dilation. Each of the partons that takes place in the interaction will carry
only a faction of their parent proton's total momentum. This fraction is
parameterized by the Bjorken \BjorkenX{i} variable, defined as $\BjorkenX{i} P
= \momentum_{i}$ where $P$ is the total proton momentum and $\momentum_{i}$ is the
momentum of a parton. In this frame work, all of the momentum is along the
beamline as the only possible source of \pt is from the binding energy of the
proton, which is approximately \ProtonMass, and so negligibly small.

The \BjorkenX{i} value of a parton is not fixed, but is instead a probability
distribution that is dependent on the flavor of the parton and the energy scale
(\InteractionEnergy) of the interaction. If there were only three,
non-interacting quarks within the proton, then the PDF function would simply be
a Dirac delta-function at $\BjorkenX{i} = 1/3$, but as the quarks are strongly
interacting they are constantly exchanging momentum with the other quarks via
gluons. The gluons carry some of the momentum, and so the delta function for
each quark is smeared out. Additionally, higher order processes tend to enhance
the lowest momentum regions, causing the distributions to rise at low
\BjorkenX{i}.

Collections of these distributions for the various flavors are called parton
distribution functions (PDFs). PDFs can not currently be calculated from QCD,
and it is not known if doing so is even possible. Instead, PDFs are models
which are fit to data and extrapolated to new interaction scales using
perturbative QCD. An example set of PDFs from the MSTW
collaboration\cite{martin_2009} is shown in \cref{fig:mstw_pdf}. In the low
interaction energy case, shown on the left plot, the \upquark and the
\downquark quark distributions have peaks at high \BjorkenX{i}, as the valence
quarks carry most of the momentum. In the high interaction energy case, shown
on the right plot, a small peak is still seen in the \upquark and \downquark
quark distributions, but the sea quarks and gluons also carry a large amount of
the total momentum.

% fig:mstw_pdf
\begin{figure}[!htbs]
    \centering
    \includegraphics[width=\textwidth]{figures/mstw_pdfs.pdf}
    \caption[
        % \cite{martin_2009}
        Example PDFs from the MSTW collaboration.
    ]{
        Example PDFs from the MSTW collaboration for interaction energy scales
        of $\InteractionEnergy = \SI{10}{\GeV\squared}$ (left) and
        $\InteractionEnergy = \SI{e4}{\GeV\squared}$ (right). The bands
        represent the \BjorkenX{i} distributions for the various partons.
    }
    \label{fig:mstw_pdf}
\end{figure}


\section{Production of \texorpdfstring{\Z}{Z} Bosons In Proton-Proton Collisions}
\label{sec:z_production}

In order to understand where the transverse momentum of a \Z boson comes from,
it is important to understand how they are produced at the LHC.

\subsection{\DrellYan Production}
\label{ssec:dy_production}

The \DrellYan (DY) process is the lowest order in \alphastrong process of
producing dilepton pairs via the \Z boson \cite{drell_1970,drell_1970a}. The
Feynman diagram of this process is shown in \cref{fig:feyn_s_channel}. As
discussed previously, the two quarks will have $\pt \approx 0$ as most of the
proton's momentum (and hence, the parton's momentum) is along the beamline, and
so the \Z will also have $\bosonpt \approx 0$. The cross section for the DY
process, in terms of the \BjorkenX{i} variables of the incoming partons, is:

\begin{equation}\label{eq:drell_yan_cross_section}
    \frac{\dirSquare{\sigma}}
    {\dir{\BjorkenX{1}} \dir{\BjorkenX{2}}}
    =
    \frac{4 \pi \fsc^{2}}
    {9 \BjorkenX{1} \BjorkenX{2} s}
    f \left( \BjorkenX{1}, \BjorkenX{2} \right)
\end{equation}

\begin{equation}\label{eq:fx}
    f \left( \BjorkenX{1}, \BjorkenX{2} \right)
    =
    \sum_{a}
    \InteractionEnergy_{a}
    \left[
        \PDF{a}{1}{\BjorkenX{1}}
        \PDF{\bar{a}}{2}{\BjorkenX{2}}
        +
        \PDF{\bar{a}}{1}{\BjorkenX{1}}
        \PDF{a}{2}{\BjorkenX{2}}
    \right]
\end{equation}

where \BjorkenX{i} is the momentum fraction carried by the parton from the
$i$th proton, \PDF{a}{i}{\BjorkenX{i}} is the individual PDF for a quark of
flavor $a$ from the $i$th proton, $s$ is the Mandelstam variable that is the
square of the center-of-mass energy, and the sum is over quark flavors.

% fig:feyn_s_channel
\begin{figure}[!htbp]
    \centering
    \includegraphics[width=\StackedPlotWidth]{figures/feyn_drell_yan_s_channel.pdf}
    \caption{
       Feynman diagram of \DrellYan \Ztoee production.
    }
    \label{fig:feyn_s_channel}
\end{figure}


\subsection{Higher Order Production}
\label{ssec:higher_order}

In addition to the DY process discussed previously, there are higher order
terms in \alphastrong that contribute to \Z boson production. Some of these
terms are shown in \cref{fig:higher_order_z_diagrams}.

The first type of term, shown in
\cref{fig:feyn_qqbar_to_zg,fig:feyn_qbarq_to_zg}, contains initial state
radiation (ISR). As discussed earlier, protons are highly dynamic systems with
their quarks interacting through multiple gluon exchanges. When a collision
happens, one of the partons is ``removed'' from the proton, leaving a ``beam
remnant'' behind that is unbalanced in both momentum and energy. This most
often tears the remnant apart, leading to colored objects escaping and
hadronizing. These colored objects are ISR. The diagrams shown here have a
gluon radiating from the incoming quark, one of the lowest order in
\alphastrong ISR processes. The quark can be left with some non-zero \pt as it
gains momentum opposite that of the gluon, and so it may impart some \bosonpt
on the \Z. The probability of radiating a gluon increases as the energy of the
radiated gluon decreases, and so ISR often introduces a small amount of
\bosonpt.

The second type of term, shown in
\cref{fig:feyn_qg_to_zq,fig:feyn_qg_to_q_to_zq}, is one where the quark
interacts with a gluon from the other proton and radiates a \Z boson. Gluons
are common in LHC interactions and so these terms are non-negligible, although
their exact contribution is difficult to calculate. These interactions have
quarks with energies close to the \Z mass, and so they have the potential to
generate higher \pt \Z bosons.

% fig:higher_order_z_diagrams
% fig:feyn_qqbar_to_zg, fig:feyn_qbarq_to_zg, fig:feyn_qg_to_zq,
% fig:feyn_qg_to_q_to_zq
\begin{figure}[!p]
    \centering
    \begin{subfigure}[b]{\SideBySidePlotWidth}
        \includegraphics[width=\linewidth]{figures/feyn_qqbar_to_zg.pdf}
        \caption{}
        \label{fig:feyn_qqbar_to_zg}
    \end{subfigure}%
    \begin{subfigure}[b]{\SideBySidePlotWidth}
        \includegraphics[width=\linewidth]{figures/feyn_qbarq_to_zg.pdf}
        \caption{}
        \label{fig:feyn_qbarq_to_zg}
    \end{subfigure}
    \begin{subfigure}[b]{\SideBySidePlotWidth}
        \includegraphics[width=\linewidth]{figures/feyn_qg_to_zq.pdf}
        \caption{}
        \label{fig:feyn_qg_to_zq}
    \end{subfigure}%
    \begin{subfigure}[b]{\SideBySidePlotWidth}
        \includegraphics[width=\linewidth]{figures/feyn_qg_to_q_to_zq.pdf}
        \caption{}
        \label{fig:feyn_qg_to_q_to_zq}
    \end{subfigure}
    \caption[
        Higher order in \alphastrong \Ztoee Feynman diagrams.
    ]{
        Higher order in \alphastrong \Ztoee Feynman diagrams. Figures~(a) and
        (b) are ISR where one of the incoming quarks radiates a gluon. In
        \FIGS~(c) and (d) the quark radiates a \Z.
    }
    \label{fig:higher_order_z_diagrams}
\end{figure}


\subsection{Final State Radiation}
\label{sec:electron_dressing}

\begin{figure}[!htb]
    \centering
    \includegraphics[width=0.65\textwidth]{figures/feyn_fsr.pdf}
    \caption{
        Feynman diagram of \Ztoll with FSR.
    }
    \label{fig:fsr_diagram}
\end{figure}

Although the \Ztoee decay is a very clean decay with no color interaction,
there is one final state process that can impact the measurement of \Z
\bosonpt. In this process, known as final state radiation (FSR),  the decay
leptons radiate photons.  The diagram of this process is shown in
\cref{fig:fsr_diagram}. Measurements and theories treat FSR differently and so
we define three types of generator level electron that each account for FSR in
a different manner: \born, \bare, and \dressed. Generator level refers to the
variables describing a collision event as computed by a Monte Carlo generator
program without considering interaction with the detector or the analysis
reconstruction chain. Generator level quantities are, therefore, free of
detector and measurement effects.

\Born generator level electrons are electrons directly after the \Ztoee decay
before the electrons have radiated any FSR photons. This definition is what
most theoretical results will provide.

\Bare generator level electrons are \born electrons after they have radiated
all of their FSR photons. This definition most closely matches how muons are
measured in a detector as the momentum of a muon is measured using only its
track, ignoring any photons.

\Dressed generator level electrons are \bare electrons with their FSR photons
added back in vector sum. The photons are only added if they are within a cone
of size $\Delta R < 0.1$ around the electron. This definition most closely
matches how electrons are measured in a detector; the energy of the electron is
measured using a calorimeter which also integrates the nearby photons into the
measurement.

Whenever generator level quantities are used, for example when performing the
unfolding as discussed in \cref{sec:unfolding}, \dressed electrons are used.

\TODO{We are not going to have bare and born results ready for the draft.}
\TODO{Whenever generator level quantities are used, for example when performing the
unfolding as discussed in \cref{sec:unfolding}, all three definitions
of generator level electron are used. This allows three final results to be
computed, one with each definition, allowing easy comparison of our results to
other experiments and theoretical models.}

\section{A New Variable: \texorpdfstring{\phistar}{Phistar}}

In the low \bosonpt region---the region which contains most of the \Z events
and is also the region governed by non-perturbative QCD---the measurement is
dominated by the systematic uncertainties and the experimental resolution.
Theorists have proposed a new variable, \phistar, that depends only on the
direction of the leptons and not on their energy in order to reduce the
systematic uncertainties associated with the measurement \cite{banfi_2011}.
\phistar is correlated to $\bosonpt / \mll$, where $\mll$ is the invariant mass
of the two leptons, so it probes the same physics as \bosonpt. This variable is
less susceptible to detector resolution effects as the energy and momentum
resolutions of a detector are, in general, worse than the position resolution.
This is due to the fact that the position of leptons is measured with a very
fine-grained tracking device whereas the energy is measured with a calorimeter.

The definition of \phistar is:

\begin{equation}\label{eq:phistar}
    \phistar = \cot \frac{\Delta \phi}{2} \sech \frac{\Delta \eta}{2}
\end{equation}

where $\Delta \phi$ is the azimuthal opening angle between the leptons, and
$\Delta \eta$ is a measure of the scattering angle of the leptons with respect
to the beam direction in the rest frame of the dilepton system.

\ATLAS, one of the other experiments on the Large Hadron Collider (LHC), has
measured \phistar at a center-of-mass energy of \SI{7}{\TeV} (\rootsseven)
\cite{atlas_phistar}. \DZERO has also measured \phistar at the Tevatron at
\rootsTevatron \cite{d0_phistar_2011,d0_phistar_2014}. This thesis presents the
first measurement of \phistar at \rootseight.
