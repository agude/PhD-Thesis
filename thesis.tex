% thesis.tex: Primary TeX control file for thesis.

\documentclass[11pt, oneside]{mnthesis}

\usepackage{longtable} % Tables that continue onto multiple pages
\usepackage{multirow} % Span rows in tables
\usepackage{amssymb} % AMS math symbols and helpers
\usepackage{graphicx} % Enhanced graphics support
\usepackage{xspace} % \xspace
\usepackage{amsmath} % \text, and other math formatting options
\usepackage{setspace} % Adjust spacing in captions, single by default
\usepackage{subcaption} % Allows subfigures
\usepackage{siunitx} % \num{} formatting and SI unit formatting
\usepackage{booktabs} % Enhanced tables with \toprule, etc.
\usepackage{enumitem} % noitemsep on lists
\usepackage[hidelinks]{hyperref} % Adds links within the document, hidelinks prevents drawing boxes around the links
\usepackage[noabbrev]{cleveref} % Automatically fill in fig., table, etc.

% Configure the siunitx package
\sisetup{
    group-separator = {,}, % Use , to separate groups of digits, like 12,345
    list-final-separator = {, and } % Always use the serial comma in \SIlist
}

% Configure the hyperref package
\hypersetup{pdftitle={
    Measurement of the phistar distribution of Z bosons decaying to
    electron pairs with the CMS experiment at a center-of-mass energy of
    8~TeV
}}
\hypersetup{pdfauthor={Alexander Erling Gude}}
\hypersetup{pdfkeywords={Physics, Particle Physics, CMS, LHC, Phistar, Z Boson}}
\hypersetup{pdfsubject={
    Measurements of the Z boson transverse momentum (Qt) spectrum serves as
    both a precision test of non-perturbative QCD and helps to reduce the
    uncertainty in the measurement of the W boson mass. However, Qt is limited
    at its lowest values by detector resolution, and so a new variable,
    phistar, which performs better in the low Qt region, is used instead. This
    thesis presents the first measurement the normalized differential cross
    section of Z bosons decaying to electron pairs in terms of phistar at sqrt
    s = 8 TeV. The data used in this measurement were collected by the CMS
    detector at the LHC in 2012 and totaled 19.7 inverse femtobarns of
    integrated luminosity. The results are compared to predictions from
    simulation, which are found to provide a poor description of the data.
}}

% Configure the cleveref package
\newcommand{\creflastconjunction}{, and } % Always use the serial comma

% Custom macros
% Elements
\newcommand{\Element}[1]{\ensuremath{\mathrm{#1}}\xspace}
\newcommand{\leadtungstate}{\Element{PbWO_{4}}}
\newcommand{\cobaltsixty}{\Element{Co_{60}}}
\newcommand{\lead}{\Element{Pb}}

% TODO Notes and placeholders
%\newcommand\TODO[1]{\textcolor{red}{TODO: #1}} % Turn on comments
\newcommand\TODO[1]{} % Turn off comments

% Ranges of numbers in math mode
\newcommand{\effstatsys}[3]{#1^{\pm #2}_{\pm #3}}

% Add space between rows of tables
\newcommand{\spacerows}[1]{\renewcommand{\arraystretch}{#1}}

% Table header GeV ranges
\newcommand{\GeVRange}[2]{\SIrange[range-units=single,range-phrase=--]{#1}{#2}{\GeV}}

% Gloassary and acronym table entries
\newcommand{\GlossaryEntry}[2]{\item \textbf{#1:} #2}
\newcommand{\AcronymEntry}[2]{#1 & #2 \\}

% Image sources
\newcommand{\Source}[4]{
    \textit{Source:}
    \href{#1}{\textit{#2}}
    \href{#3}{\textit{#4}}
}

% Theory
\newcommand{\GroupShort}[2]{\ensuremath{\text{#1}(#2)}\xspace}
\newcommand{\Group}[3]{\ensuremath{\GroupShort{#1}{#2}_{\text{#3}}}\xspace}
\newcommand{\SUthree}{\Group{SU}{3}{C}}
\newcommand{\SUtwo}{\Group{SU}{2}{L}}
\newcommand{\Uone}{\GroupShort{U}{1}}
\newcommand{\SUtwoUone}{\ensuremath{\SUtwo \times \Uone}\xspace}
\newcommand{\SUthreeSUtwoUone}{\ensuremath{\SUthree \times \SUtwoUone}\xspace}

% Coordinates
\newcommand{\Coord}[1]{\ensuremath{#1}\xspace}
\newcommand{\TwoCoord}[2]{\ensuremath{#1\text{--}#2}\xspace}
\newcommand{\Axis}[1]{\ensuremath{#1\text{-axis}}\xspace}
\newcommand{\Plane}[1]{\ensuremath{#1\text{~plane}}\xspace}

\newcommand{\coordx}{\Coord{x}}
\newcommand{\coordy}{\Coord{y}}
\newcommand{\coordz}{\Coord{z}}

\newcommand{\xaxis}{\Axis{\coordx}}
\newcommand{\yaxis}{\Axis{\coordy}}
\newcommand{\zaxis}{\Axis{\coordz}}

\newcommand{\coordxy}{\TwoCoord{\coordx}{\coordy}}
\newcommand{\xyplane}{\Plane{\coordxy}}

\newcommand{\coordr}{\Coord{r}}
\newcommand{\coordphi}{\Coord{\phi}}
\newcommand{\coordeta}{\Coord{\eta}}
\newcommand{\coordtheta}{\Coord{\theta}}

\newcommand{\coordrphi}{\TwoCoord{\coordr}{\coordphi}}
\newcommand{\rphiplane}{\Plane{\coordrphi}}

\newcommand{\rzplane}{\Plane{\TwoCoord{\coordr}{\coordz}}}
\newcommand{\coordetaphi}{\TwoCoord{\coordeta}{\coordphi}}

\newcommand{\phisc}{\Coord{\phi_{\text{SC}}}}
\newcommand{\phizero}{\Coord{\phi_{0}}}
\newcommand{\Reffective}{\Coord{R_{\text{effective}}}}

% Trig
\DeclareMathOperator{\sech}{sech}
\DeclareMathOperator{\logn}{ln}
\DeclareMathOperator{\erfc}{erfc}

% Big O
\newcommand{\BigO}[1]{\ensuremath{\operatorname{O}\!\left( #1 \right)}\xspace}

% Derivatives
\newcommand{\dir}[1]{\ensuremath{\text{d}#1}\xspace}
\newcommand{\dirSquare}[1]{\ensuremath{\text{d}^{2}#1}\xspace}

% Anaylsis

%% Electron Definitions
\newcommand{\CentralElectron}{central electron\xspace}
\newcommand{\CentralElectrons}{central electrons\xspace}
\newcommand{\ExtendedElectron}{extended electron\xspace}
\newcommand{\ExtendedElectrons}{extended electrons\xspace}

%% Electron Dressing
\newcommand{\born}{Born\xspace}
\newcommand{\bare}{bare\xspace}
\newcommand{\dressed}{dressed\xspace}
\newcommand{\Born}{Born\xspace}
\newcommand{\Bare}{Bare\xspace}
\newcommand{\Dressed}{Dressed\xspace}

%% Techniques
\newcommand{\TnP}{T\&P\xspace}
\newcommand{\dressing}{``dressing''\xspace}
\newcommand{\dressedelectrons}{``\dressed electrons''\xspace}

% Experiments

%% Parts of CMS
\newcommand{\Lone}{Level-1\xspace}
\newcommand{\Ltwo}{Level-2\xspace}
\newcommand{\Lthree}{Level-3\xspace}
\newcommand{\threebythree}{3x3\xspace}
\newcommand{\fivebyfive}{5x5\xspace}

%% LHC locations, experiments, etc.
\newcommand{\pointfive}{Point 5\xspace}
\newcommand{\linactwo}{Linac 2\xspace}

\newcommand{\Experiment}[1]{\text{#1}\xspace}
\newcommand{\ALICE}{\Experiment{ALICE}}
\newcommand{\ATLAS}{\Experiment{ATLAS}}
\newcommand{\LHCB}{\Experiment{LHCb}}
\newcommand{\LEP}{\Experiment{LEP}}
\newcommand{\DZERO}{\Experiment{D0}}

% Theory Groups
\newcommand{\GFitter}{Gfitter\xspace}

% Working Groups
\newcommand{\PDFforLHC}{PDF4LHC\xspace}

% People
\newcommand{\Cherenkov}{Cherenkov\xspace}
\newcommand{\DAgostini}{D'Agostini\xspace}
\newcommand{\DrellYan}{Drell--Yan\xspace}
\newcommand{\Moliere}{Moli\`{e}re\xspace}
\newcommand{\Neeman}{Ne'eman\xspace}

% General
\newcommand{\Particle}[1]{\ensuremath{\text{#1}}\xspace}
\newcommand{\ParticleMath}[1]{\ensuremath{#1}\xspace}

\newcommand{\xxbar}[2]{\ensuremath{#1 \overline{#2}}\xspace}
\newcommand{\xxpm}[2]{\ensuremath{#1^{+} #2^{-}}\xspace}

% Leptons
\newcommand{\lepton}{\ParticleMath{\ell}}
\newcommand{\neutrino}{\ParticleMath{\nu}}
\newcommand{\electron}{\Particle{e}}
\newcommand{\muon}{\ParticleMath{\mu}}
\newcommand{\tauon}{\ParticleMath{\tau}}
\newcommand{\ee}{\xxpm{\electron}{\electron}}

\newcommand{\mutoWnu}{\ensuremath{\mu \to \W \neutrino_{\mu}}\xspace}

% Quarks
\newcommand{\quark}{\Particle{q}}
\newcommand{\upquark}{\Particle{u}}
\newcommand{\downquark}{\Particle{d}}
\newcommand{\charmquark}{\Particle{c}}
\newcommand{\strangequark}{\Particle{s}}
\newcommand{\topquark}{\Particle{t}}
\newcommand{\bottomquark}{\Particle{b}}

\newcommand{\tbar}{\overline{\topquark}\xspace}
\newcommand{\bbar}{\overline{\bottomquark}\xspace}

\newcommand{\QuarkColor}[1]{\ensuremath{#1}\xspace}
\newcommand{\red}{\QuarkColor{r}}
\newcommand{\green}{\QuarkColor{g}}
\newcommand{\blue}{\QuarkColor{b}}

% Mesons
\newcommand{\qqbar}{\xxbar{\quark}{\quark}}
\newcommand{\meson}{\qqbar}

%% Pions
\newcommand{\pion}{\ParticleMath{\pi}}
\newcommand{\pionplus}{\ParticleMath{\pion^{+}}}
\newcommand{\pionminus}{\ParticleMath{\pion^{-}}}
\newcommand{\pionzero}{\ParticleMath{\pion^{0}}}

\newcommand{\ChargeExchange}{\ensuremath{\pionplus + \neutron \to \pionzero + \proton}\xspace}
\newcommand{\pitogammagamma}{\ensuremath{\pionzero \to \photon \photon}\xspace}
\newcommand{\pitomunu}{\ensuremath{\pionplus \to \mu^{+} + \neutrino_{\mu}}\xspace}

%% JPsi
\newcommand{\jpsi}{\ensuremath{\Particle{J}\hspace{-.08em}/\hspace{-.14em}\psi}\xspace}

% Baryons
\newcommand{\baryon}{\ensuremath{\quark \quark \quark}\xspace}
\newcommand{\proton}{\Particle{p}}
\newcommand{\neutron}{\Particle{n}}

% Bosons

%% Photon
\newcommand{\photon}{\ParticleMath{\ensuremath{\gamma}}}
\newcommand{\PhotonConversion}{\ensuremath{\photon \to \xxpm{\electron}{\electron}}\xspace}

%% W
\newcommand{\W}{\Particle{W}}
\newcommand{\Wpm}{\ParticleMath{\W^{\pm}}}
\newcommand{\Wp}{\ParticleMath{\W^{+}}}
\newcommand{\Wm}{\ParticleMath{\W^{-}}}

\newcommand{\Wtoqq}{{\ensuremath{\W \to \xxbar{\quark}{\quark}}}\xspace}
\newcommand{\Wtolnu}{{\ensuremath{\W \to \lepton \neutrino}}\xspace}

%% Z
\newcommand{\Z}{\Particle{Z}}
\newcommand{\DY}{\Particle{DY}\xspace}

\newcommand{\Zto}[1]{\ensuremath{\Z \to #1}\xspace}
\newcommand{\Ztoqq}{\ensuremath{\Zto{\qqbar}}\xspace}
\newcommand{\Ztoll}{\ensuremath{\Zto{\xxpm{\lepton}{\lepton}}}\xspace}
\newcommand{\Ztoee}{\ensuremath{\Zto{\xxpm{\electron}{\electron}}}\xspace}
\newcommand{\Ztomumu}{\ensuremath{\Zto{\xxpm{\muon}{\muon}}}\xspace}
\newcommand{\Ztotautau}{\ensuremath{\Zto{\xxpm{\tauon}{\tauon}}}\xspace}
\newcommand{\Ztonunu}{\ensuremath{\Zto{\xxbar{\neutrino}{\neutrino}}}\xspace}

\newcommand{\DYto}[1]{\ensuremath{\DY \to #1}\xspace}
\newcommand{\DYtoqq}{\ensuremath{\DYto{\qqbar}}\xspace}
\newcommand{\DYtoll}{\ensuremath{\DYto{\xxpm{\lepton}{\lepton}}}\xspace}
\newcommand{\DYtoee}{\ensuremath{\DYto{\xxpm{\electron}{\electron}}}\xspace}
\newcommand{\DYtomumu}{\ensuremath{\DYto{\xxpm{\muon}{\muon}}}\xspace}
\newcommand{\DYtotautau}{\ensuremath{\DYto{\xxpm{\tauon}{\tauon}}}\xspace}
\newcommand{\DYtonunu}{\ensuremath{\DYto{\xxbar{\neutrino}{\neutrino}}}\xspace}

%% Higgs
\newcommand{\higgs}{\Particle{H}}
\newcommand{\higgstogammagamma}{\ensuremath{\higgs \to \photon \photon}\xspace}
\newcommand{\higgstoZZ}{\ensuremath{\higgs \to \Z \Z}\xspace}

% Backgrounds
\newcommand{\ttbar}{\ensuremath{\topquark \tbar}\xspace}
\newcommand{\tW}{\ensuremath{\topquark \W}\xspace}
\newcommand{\tbarW}{\ensuremath{\tbar \W}\xspace}
\newcommand{\ZZ}{\ensuremath{\Z \Z}\xspace}
\newcommand{\WZ}{\ensuremath{\W \Z}\xspace}
\newcommand{\WW}{\ensuremath{\W \W}\xspace}
%\DYtotautau
\newcommand{\wjets}{\ensuremath{\W + \text{jets}}\xspace}
\newcommand{\QCDjets}{\text{QCD multi-jet}\xspace}

%% Expanded tW decays
\newcommand{\tWdecay}{\ensuremath{\topquark \to \W \bottomquark}\xspace}
\newcommand{\tbarWdecay}{\ensuremath{\tbar \to \W \bbar}\xspace}

%% Control Sample
\newcommand{\emu}{\ensuremath{\electron\text{--}\mu}\xspace}

% General
\newcommand{\Dataset}[1]{``#1''\xspace}
\newcommand{\Trigger}[1]{\texttt{#1}\xspace}
\newcommand{\WorkingPoint}[1]{\texttt{#1}\xspace}

% Datasets
\newcommand{\SingleElectron}{\Dataset{SingleElectron}}
\newcommand{\DoubleElectron}{\Dataset{DoubleElectron}}
\newcommand{\SingleMuon}{\Dataset{SingleMuon}}

% Triggers
\newcommand{\SingleElectronTrigger}{\Trigger{HLT\_Ele27\_WP80}}
\newcommand{\TnPTrigger}{\Trigger{HLT\_Ele20\_CaloIdVT\_CaloIsoVT\_TrkIdT\_TrkIsoVT\_SC4\_Mass50}}
\newcommand{\TnPTriggerSecond}{\Trigger{HLT\_Ele17\_CaloIdVT\_CaloIsoVT\_TrkIdT\_TrkIsoVT\_Ele8\_Mass50}}
\newcommand{\SingleMuonTrigger}{\Trigger{HLT\_IsoMu24\_eta2p1}}

% Working points
\newcommand{\WPEighty}{\WorkingPoint{WP80}}
\newcommand{\EGTIGHT}{\WorkingPoint{Tight}}
\newcommand{\EGMEDIUM}{\WorkingPoint{Medium}}

% General
\newcommand{\ParameterSet}[1]{\texttt{#1}\xspace}
\newcommand{\Software}[1]{\textsc{#1}\xspace}
\newcommand{\SoftwareVersion}[1]{v#1\xspace}

% MC
\newcommand{\FEWZ}{\Software{fewz}}
\newcommand{\GEANTfour}{\Software{Geant4}}
\newcommand{\MADGRAPH}{\Software{MadGraph}}
\newcommand{\POWHEG}{\Software{powheg}}
\newcommand{\PYTHIAeight}{\Software{Pythia8}}
\newcommand{\PYTHIAsix}{\Software{Pythia6}}
\newcommand{\PYTHIA}{\Software{Pythia}}
\newcommand{\Tauola}{\Software{Tauola}}

% Analysis
\newcommand{\FSRWeightProducer}{\Software{FSRWeightProducer}}
\newcommand{\PDFWeightProducer}{\Software{PDFWeightProducer}}
\newcommand{\RooUnfold}{\Software{RooUnfold}}

% Tunes
\newcommand{\ZTwoStar}{\ParameterSet{Z2star}}
\newcommand{\CTten}{\ParameterSet{CT10}}
\newcommand{\TunePPfive}{\ParameterSet{Tunepp5}}
\newcommand{\TunePPfourteen}{\ParameterSet{Tunepp14}}

% Units (Most replaced with siunitx)
\newcommand{\megarads}{\ensuremath{\text{\,Mrad}}\xspace}

% Energies
\newcommand{\EnergyIneV}[1]{\ensuremath{\,\text{#1e\hspace{-.08em}V}}\xspace}
\newcommand{\MeV}{\EnergyIneV{M}}
\newcommand{\GeV}{\EnergyIneV{G}}
\newcommand{\TeV}{\EnergyIneV{T}}

% Luminosity
\newcommand{\pb}{\mbox{\ensuremath{\,\text{pb}}}\xspace}
\newcommand{\fb}{\mbox{\ensuremath{\,\text{fb}}}\xspace}
\newcommand{\nb}{\mbox{\ensuremath{\,\text{nb}}}\xspace}
\newcommand{\mub}{\ensuremath{\,\mu\mathrm{b}}\xspace}
\newcommand{\pbinv}{\mbox{\ensuremath{\,\text{pb}^\text{$-$1}}}\xspace}
\newcommand{\fbinv}{\mbox{\ensuremath{\,\text{fb}^\text{$-$1}}}\xspace}
\newcommand{\nbinv}{\mbox{\ensuremath{\,\text{nb}^\text{$-$1}}}\xspace}
\newcommand{\mubinv}{\ensuremath{\,\mu\mathrm{b}^{-1}}\xspace}

% The luminosity used in the analysis
\newcommand{\GoodLumiNumber}{19.7 \fbinv}

% The luminosity uncertainty
\newcommand{\LumiUncertainty}{\ensuremath{2.6\%}\xspace}

% The Z Mass and Width from the PDG
\newcommand{\Zmass}{\ensuremath{91.1876 \pm 0.0021 \GeV}\xspace}
\newcommand{\Zwidth}{\ensuremath{2.4952 \pm 0.0023 \GeV}\xspace}

% General
\newcommand{\Variable}[1]{\ensuremath{#1}\xspace}

% Physics Variables

%% Phistar (!!)
\newcommand{\phistar}{\Variable{\phi^{*}}}
\newcommand{\phistarSC}{\Variable{\phistar_{\text{SC}}}}
\newcommand{\phistarReco}{\Variable{\phistar_{\text{Reco}}}}
\newcommand{\phistarGen}{\Variable{\phistar_{\text{Gen}}}}

% Momentum
\newcommand{\momentum}{p}
\newcommand{\pt}{\Variable{\momentum_{\mathrm{T}}}}
\newcommand{\bosonpt}{\Variable{Q_{\mathrm{T}}}}
\newcommand{\bosonptk}{\Variable{Q_{\mathrm{T},k}}}

%% Energy
\newcommand{\Energy}{\ensuremath{E}\xspace}
\newcommand{\ET}{\ensuremath{E_{\mathrm{T}}}\xspace}
\newcommand{\et}{\ET}
\newcommand{\MET}{\ensuremath{E_{\mathrm{T}}^{\text{miss}}}\xspace}
\newcommand{\ETslash}{\ensuremath{E_{\mathrm{T}}\hspace{-1.1em}/\kern0.45em}\xspace}
\newcommand{\PFMET}{\ensuremath{E_{\mathrm{T}}^{\text{PF miss}}}\xspace}

%% Rapidity
\newcommand{\rapidity}{\Variable{Y}}

%% Accelerator
\newcommand{\roots}[1]{\Variable{\sqrt{s} = \SI{#1}{\TeV}}}
\newcommand{\rootseight}{\roots{8}}
\newcommand{\rootsseven}{\roots{7}}
\newcommand{\rootsTevatron}{\roots{1.96}}
\newcommand{\luminosity}{\mathcal{L}\xspace}

%% Detector
\newcommand{\radiationlength}{\Variable{X_{0}}}

%% Electron Reconstruction
\newcommand{\DeltaRSum}{\Variable{\sum_{\Delta \text{R}<0.3}}}
\newcommand{\ECALISO}{\Variable{\text{Iso}_{\text{ECAL}}}}
\newcommand{\EECAL}{\Variable{E^{\text{ECAL}}}}
\newcommand{\EHCAL}{\Variable{E^{\text{HCAL}}}}
\newcommand{\ESC}{\Variable{E^{\text{SC}}}}
\newcommand{\HCALISO}{\Variable{\text{Iso}_{\text{HCAL}}}}
\newcommand{\HOverE}{\Variable{\text{H}/\text{E}}}
\newcommand{\PFISO}{\Variable{\text{Iso}_{\text{PF}}}}
\newcommand{\RNine}{\Variable{R_{9}}}
\newcommand{\detain}{\Variable{\Delta \eta_{\text{in}}}}
\newcommand{\dphiin}{\Variable{\Delta \phi_{\text{in}}}}
\newcommand{\dzero}{\Variable{\text{d}0}}
\newcommand{\dz}{\Variable{\text{d}\coordz}}
\newcommand{\etElectron}{\Variable{\et^{\text{Electron}}}}
\newcommand{\nmiss}{\Variable{\text{N}_{\text{miss}}}}
\newcommand{\ooeoop}{\Variable{\left( 1/E - 1/p \right)}}
\newcommand{\ptElectron}{\Variable{\pt^{\text{Electron}}}}
\newcommand{\ptTrack}{\Variable{\pt^{\text{Track}}}}
\newcommand{\pvtx}{\Variable{\text{P}_{\text{vtx}}}}
\newcommand{\sigmaietaieta}{\Variable{\sigma_{i \eta i \eta}}}

%% Efficiency
\newcommand{\eff}{\Variable{\epsilon}}
\newcommand{\effdata}{\Variable{\eff^{\text{Data}}}}
\newcommand{\effmc}{\Variable{\eff^{\text{MC}}}}

%% Masses
\newcommand{\MassW}{\Variable{M_{\W}}}
\newcommand{\MZ}{\Variable{M_{\Z}}}
\newcommand{\ProtonMass}{\Variable{m_{\proton}}}
\newcommand{\mee}{\Variable{m_{\electron \electron}}}
\newcommand{\mll}{\Variable{m_{\lepton\lepton}}}

\newcommand{\MassRange}{\Variable{\SI{60}{\GeV} < \mee < \SI{120}{\GeV}}}

%% Widths
\newcommand{\GammaZ}{\Variable{\Gamma_{Z}}}

%% QED and QCD
\newcommand{\BjorkenX}[1]{\Variable{x_{#1}}}
\newcommand{\InteractionEnergy}{\Variable{Q^{2}}}
\newcommand{\LambdaQCD}{\Variable{\Lambda_{\text{QCD}}}}
\newcommand{\PDF}[3]{\Variable{f^{#1}_{#2} \left( #3 \right)}}
\newcommand{\alphastrong}{\Variable{\alpha_{\text{s}}}}
\newcommand{\fsc}{\Variable{\alpha}}

%% Spin
\newcommand{\spin}[1]{spin--{#1}\xspace}
\newcommand{\spinhalf}{\spin{1/2}}
\newcommand{\spinone}{\spin{1}}
\newcommand{\spinzero}{\spin{0}}

%% QCD Background Fit
\newcommand{\BGFunc}{\Variable{F_{\text{BG}}}}
\newcommand{\BGFuncArgs}{\Variable{\BGFunc(x; \gamma, \delta, \varepsilon)}} \newcommand{\MCTemplate}{\ensuremath{T_{\text{MC}}}}

%% Uncertainties
\newcommand{\err}{\Variable{\varepsilon}}
\newcommand{\errnorm}{\Variable{\err^{\text{Norm.}}}}
\newcommand{\errabs}{\Variable{\err^{\text{Abs.}}}}


% Custom LaTeX variables
% General
\newcommand{\Variable}[1]{\ensuremath{#1}\xspace}

% Physics Variables

%% Phistar (!!)
\newcommand{\phistar}{\Variable{\phi^{*}}}
\newcommand{\phistarSC}{\Variable{\phistar_{\text{SC}}}}
\newcommand{\phistarReco}{\Variable{\phistar_{\text{Reco}}}}
\newcommand{\phistarGen}{\Variable{\phistar_{\text{Gen}}}}

% Momentum
\newcommand{\momentum}{p}
\newcommand{\pt}{\Variable{\momentum_{\mathrm{T}}}}
\newcommand{\bosonpt}{\Variable{Q_{\mathrm{T}}}}
\newcommand{\bosonptk}{\Variable{Q_{\mathrm{T},k}}}

%% Energy
\newcommand{\Energy}{\ensuremath{E}\xspace}
\newcommand{\ET}{\ensuremath{E_{\mathrm{T}}}\xspace}
\newcommand{\et}{\ET}
\newcommand{\MET}{\ensuremath{E_{\mathrm{T}}^{\text{miss}}}\xspace}
\newcommand{\ETslash}{\ensuremath{E_{\mathrm{T}}\hspace{-1.1em}/\kern0.45em}\xspace}
\newcommand{\PFMET}{\ensuremath{E_{\mathrm{T}}^{\text{PF miss}}}\xspace}

%% Rapidity
\newcommand{\rapidity}{\Variable{Y}}

%% Accelerator
\newcommand{\roots}[1]{\Variable{\sqrt{s} = \SI{#1}{\TeV}}}
\newcommand{\rootseight}{\roots{8}}
\newcommand{\rootsseven}{\roots{7}}
\newcommand{\rootsTevatron}{\roots{1.96}}
\newcommand{\luminosity}{\mathcal{L}\xspace}

%% Detector
\newcommand{\radiationlength}{\Variable{X_{0}}}

%% Electron Reconstruction
\newcommand{\DeltaRSum}{\Variable{\sum_{\Delta \text{R}<0.3}}}
\newcommand{\ECALISO}{\Variable{\text{Iso}_{\text{ECAL}}}}
\newcommand{\EECAL}{\Variable{E^{\text{ECAL}}}}
\newcommand{\EHCAL}{\Variable{E^{\text{HCAL}}}}
\newcommand{\ESC}{\Variable{E^{\text{SC}}}}
\newcommand{\HCALISO}{\Variable{\text{Iso}_{\text{HCAL}}}}
\newcommand{\HOverE}{\Variable{\text{H}/\text{E}}}
\newcommand{\PFISO}{\Variable{\text{Iso}_{\text{PF}}}}
\newcommand{\RNine}{\Variable{R_{9}}}
\newcommand{\detain}{\Variable{\Delta \eta_{\text{in}}}}
\newcommand{\dphiin}{\Variable{\Delta \phi_{\text{in}}}}
\newcommand{\dzero}{\Variable{\text{d}0}}
\newcommand{\dz}{\Variable{\text{d}\coordz}}
\newcommand{\etElectron}{\Variable{\et^{\text{Electron}}}}
\newcommand{\nmiss}{\Variable{\text{N}_{\text{miss}}}}
\newcommand{\ooeoop}{\Variable{\left( 1/E - 1/p \right)}}
\newcommand{\ptElectron}{\Variable{\pt^{\text{Electron}}}}
\newcommand{\ptTrack}{\Variable{\pt^{\text{Track}}}}
\newcommand{\pvtx}{\Variable{\text{P}_{\text{vtx}}}}
\newcommand{\sigmaietaieta}{\Variable{\sigma_{i \eta i \eta}}}

%% Efficiency
\newcommand{\eff}{\Variable{\epsilon}}
\newcommand{\effdata}{\Variable{\eff^{\text{Data}}}}
\newcommand{\effmc}{\Variable{\eff^{\text{MC}}}}

%% Masses
\newcommand{\MassW}{\Variable{M_{\W}}}
\newcommand{\MZ}{\Variable{M_{\Z}}}
\newcommand{\ProtonMass}{\Variable{m_{\proton}}}
\newcommand{\mee}{\Variable{m_{\electron \electron}}}
\newcommand{\mll}{\Variable{m_{\lepton\lepton}}}

\newcommand{\MassRange}{\Variable{\SI{60}{\GeV} < \mee < \SI{120}{\GeV}}}

%% Widths
\newcommand{\GammaZ}{\Variable{\Gamma_{Z}}}

%% QED and QCD
\newcommand{\BjorkenX}[1]{\Variable{x_{#1}}}
\newcommand{\InteractionEnergy}{\Variable{Q^{2}}}
\newcommand{\LambdaQCD}{\Variable{\Lambda_{\text{QCD}}}}
\newcommand{\PDF}[3]{\Variable{f^{#1}_{#2} \left( #3 \right)}}
\newcommand{\alphastrong}{\Variable{\alpha_{\text{s}}}}
\newcommand{\fsc}{\Variable{\alpha}}

%% Spin
\newcommand{\spin}[1]{spin--{#1}\xspace}
\newcommand{\spinhalf}{\spin{1/2}}
\newcommand{\spinone}{\spin{1}}
\newcommand{\spinzero}{\spin{0}}

%% QCD Background Fit
\newcommand{\BGFunc}{\Variable{F_{\text{BG}}}}
\newcommand{\BGFuncArgs}{\Variable{\BGFunc(x; \gamma, \delta, \varepsilon)}} \newcommand{\MCTemplate}{\ensuremath{T_{\text{MC}}}}

%% Uncertainties
\newcommand{\err}{\Variable{\varepsilon}}
\newcommand{\errnorm}{\Variable{\err^{\text{Norm.}}}}
\newcommand{\errabs}{\Variable{\err^{\text{Abs.}}}}


% Set the line spacing; 1.3 is one and a half, 1.6 is double spacing
\linespread{1.3}

% Compile only the chapters listed here
\includeonly{
    preliminaries/title,
    chapters/intro,
    chapters/theory,
    chapters/experiment,
    chapters/reconstruction,
    chapters/data_and_mc,
    chapters/event_selection,
    chapters/analysis,
    chapters/app_other_dressed_measurements,
    chapters/app_uncertainty_tables,
    chapters/app_qcd_fits,
    chapters/app_glossary,
}

\begin{document}
\bibliographystyle{hunsrt} % style of bibliography

% Make LaTeX care less about word spacing and more about not letting words run
% into the margins
%\sloppy

% Title, copyright, etc.
% Title and other preliminaries
\phd % Set \degree and \initials to PhD
%\draft % Print a draft instead

\title{
    % No math or symbols allowed!
    \textbf{
        Measurement of the phistar distribution of Z bosons decaying to
        electron pairs with the CMS experiment at a center-of-mass energy of
        8~TeV
    }
}
\author{Alexander Erling Gude}
\campus{University of Minnesota}
\program{Physics}
\degree{DOCTORATE OF PHILOSOPHY}
\director{Jeremiah Mans}

% Optionally specify the month and year.
\submissionmonth{May} % defaults to current month.
\submissionyear{2015} % defaults to current year.

%Comment out below on final copy
\abstract{% Abstract

Measurements of the \Z boson transverse momentum (\bosonpt) spectrum serve as
both precision tests of non-perturbative QCD and help to reduce the uncertainty
in the measurement of the \W boson mass. However, \bosonpt is
limited at its lowest values by detector resolution, and so a new variable,
\phistar, which performs better in the low \bosonpt region, is used instead.
This thesis presents the first measurement of the normalized differential cross
section of \Z bosons decaying to electron pairs in terms of \phistar at
\rootseight. The data used in this measurement were collected by the CMS
detector at the LHC in 2012 and totaled \GoodLumiNumber of integrated
luminosity. The results are compared to predictions from simulation, which are
found to provide a poor description of the data.
}
\copyrightpage
\acknowledgements{No scientific work is created in isolation and this thesis is no exception.
Many people were instrumental in ensuring its completion, either directly
through their work, or indirectly through their influences on my life.

To begin, I would like to offer thanks to my parents, Hans and Laurel Gude, who
always encouraged my love of science. Whether it was when I wanted to study
dinosaurs, or planets and stars, or whales, or finally supernovae and
sub-atomic particles, they were always there cheering me on.

Thanks to my science and math teachers; they saw something in me when I was
young and helped to nurture the skills and interests I would need to be a
scientist, and for that I am forever in their debt.

A big thanks to the members of the Supernova Cosmology Project, especially Saul
Perlmutter, Anthony Spadafora, Nao Suzuki, and David Rubin. They made me part
of their group when I was a student at Berkeley and taught me how to do
precision physics. Working with these world-class scientists, and being able to
contribute, gave me the confidence I needed when I otherwise felt like I was
inadequate as a scientist.

A special thanks to the graduate students at the University of Minnesota:
Michael Albright, Dan Endean, Charles McEachern, Roxanne Radpour, Dominic
Rocco, Allan and Mandy Straub, and Jan Zirnstein. They offered their friendship
to a Californian sorely out of place in the Midwest and so filled the last six
years with joy.

I would especially like to thank the members of the CMS group at the University
of Minnesota: Roger Rusack, Yuichi Kubota, Bryan Dahmes, Nathaniel Joseph
Pastika, and Jared Turkewitz. They offered invaluable advice about difficulties
I encountered when I was just starting out in high energy physics. Thanks to
Kevin Klapoetke, who worked on this measurement before me, to Alexey Finkel,
who assisted me with this measurement, and Zachary Lesko, who also assisted me
and will take over now that I have graduated. A special thanks to my adviser,
Jeremiah Mans, who guided me through my formative graduate school years and who
made me the scientist I am today. Also thanks to Nicole Ruckstuhl, who worked
side-by-side on the analysis with me and without whom I would have been
completely lost. If this work is of any value, it is because of the
contributions from those two.

Of course, none of this work would have been possible without the thousands of
scientists, engineers, and other people working hard at CERN, the LHC, and CMS.
\textit{Merci beaucoup pour votre travail acharn\'{e}}!

Thanks to Alan Wu for his extensive edits and for letting me bounce ideas off
him when I was not sure how to phrase things. Without his help, this work would
have been even more impenetrable than it already is.

Finally, I would like to thank my wife, Connie Lam. She offered me her love and
support through the entire process, and never wavered even when the thesis
began to take over all of my free time. I love you!
}
\dedication{To my wife, Connie, and to our daughter; I can not wait to share the wonder
that is the exploration of the natural world with you!
}

% The \beforepreface command actually causes insertion of the title,
% abstract, signature, and copyright pages into the new document.
\beforepreface

% Define the text to go before the table of contents
\figurespage
\tablespage

% The \afterpreface command actually causes insertion of the
% contents, list of figures, etc. into the new document.
\afterpreface


% Introduction
\chapter{Introduction}
\label{chapter:intro}

\begin{description}

    \item[Chapter~\ref{chapter:theory}] presents the history of the standard
        model and the motivation for the measurement.

    \item[Chapter~\ref{chapter:experiment}] gives an overview of the Large
        Hadron Collider and the Compact Muon Solenoid (CMS).

    \item[Chapter~\ref{chapter:reconstruction}] describes the way in which
        electrons are measured at CMS.

    \item[Chapter~\ref{chatper:data_and_mc_samples}] reviews the simulated data
        samples and the actual data samples used in the measurement.

    \item[Chapter~\ref{chapter:event_selection}] discusses the method of
        selecting events and correcting them.

    \item[Chapter~\ref{chapter:analysis}] presents the analysis including the
        uncertainties and the final results of the measurement.

\end{description}


% Theory
\chapter{Physics of Z Transverse Momentum}
\label{chapter:theory}

\section{The Standard Model}
\label{section:standard_model}

Our current understanding of how matter interacts at high energy is entirely
described by the standard model (SM) as constructed by Weinberg, Glashow, and
Salam \cite{glashow1961}\cite{weinberg1967}\cite{salam1968}. The model combines
three of the four fundamental forces (leaving out only gravity, which is so
weak as to be negligible) and explains almost everything we see in the
universe.

\subsection{The Electromagnetic Force}
\label{subsection:electronmagnetic_force}

The modern theory of electromagnetism began with Maxwell's theory developed in
the middle of the 19th century \cite{maxwell1863}. Maxwell was the first to
conclude that light was an electromagnetic wave, the full importance of which
was only later understood when it was discovered that the photon was the force
carrier of the electromagnetic force \cite{maxwell1864}. 

In the early 20th century, Lorentz and Einstein developed relativistic
mechanics and showed that Maxwell's theory was Lorentz invariant
\cite{lorentz1899}\cite{einstein1904}. Dirac updated the theory in 1920 when he
was able to quantize the electromagnetic field as an ensemble of harmonic
oscillators \cite{dirac1927}. Dirac would go on to discover that anti-particles
were a natural consequences of his equations \cite{dirac1928}\cite{dirac1930}.
These anti-particles were found by Anderson in 1932 as he observed cosmic rays
in a cloud chamber \cite{anderson1933}.

As microwave technology improved in the 1940s, more accurate measurements of
the energy level shifts in hydrogen were made, resulting in the discovery of
the Lamb shift by Lamb and Rutherford \cite{lamb1947}. This discovery pointed
to discrepancies in the theory, discrepancies which Bethe would explain after
completing a set of non-relativistic calculations using it \cite{bethe1947}.
Bethe's work inspired multiple other physicist including Dyson, Feynman,
Schwinger, and Tomonaga to work along similar lines. They created quantum
electrodynamics, a fully relativistic and self-consistent theory of
electromagnetic interactions
\cite{tomonaga1946}\cite{schwinger1948}\cite{feynman1949}\cite{dyson1949}.

\subsection{The Weak Force}
\label{subsection:weak_force}

The need for a weak force, and hence a theory describing it, was first hinted
at by beta decay experiments in the 1911 performed by Meitner and
Hahn showed that the energy spectrum of the electron ejected in beta decay was
continuous instead of a delta function as would be expected for a two-body
decay .\TODO{Cite} While some proposed that this discovery indicated that
momentum and energy were not conserved, Pauli proposed an alternative: there
was a neutral and invisible particle that carried away some of the energy---the
neutrino \cite{pauli1930}. Fermi began working on this idea and invented a
four fermion contact interaction in which a neutron decayed into a proton,
an electron, and a neutrino \cite{fermi1934}.

In 1947, Rochester and Butler discovered a particle that decayed to two pions
which they called the $\theta$; in 1949, Brown and Powell discovered a particle
that decayed to three pions which they called the $\tau$
\cite{Rochester1947}\cite{brown1949}. It was soon discovered that these
particles had the same mass and lifetime---indicating that they were the same
particle---but, based on their decay products, they must have different parity.
Lee and Yang proposed that perhaps there was only one particle undergoing a  a
parity violating decay \cite{lee1956}. Their idea was confirmed by Wu in 1956, who
showed that electrons were preferential emitted from \cobaltsixty in one
direction, and by Garwin, Lederman, and Weinrich in 1957, who studied \pitomunu
in a storage ring \cite{wu1956}\cite{garwin1957}.
\subsection{The Strong Force}
\label{subsection:Strong_force}


% The experiment, and how we use it
% experiment.tex: Chapter describing the experiment

\chapter{The CMS Experiment}
\label{experiment_chapter}

\section{The Large Hadron Collider}
\label{lhc_section}

\TODO{LHC Graphic of accelerator chain}

The Large Hadron Collider (LHC) is the worlds highest energy and largest
particle accelerator with a maximum design luminosity of 14\TeV and a radius of
2804\meters \cite{bruning2004}. It collides protons on protons. In 2012, when
LHC most recently produced collisions, it had a center of mass energy of 7\TeV;
when it turns back on in 2015 after upgrades it will run at either 13\TeV or
14\TeV.

The LHC is located near Geneva, Switzerland, although parts of the accelerator
(including \pointfive, where the CMS detector is located) are in France.

A number of smaller accelerators are used together in series to accelerate
protons to the energies necessary to be injected into the LHC. The first step
is a linear accelerator, \linactwo, which accelerates protons from rest to to
50\MeV. These protons are then injected into a chain of three circular
accelerators, each injecting into the next. The first of these accelerators is
the Proton Synchrotron Booster (PSB) which accelerates the protons to 1.4\GeV.
The second is the Proton Synchrotron (PS) which accelerates the protons to
26\GeV. The third and final accelerator in the Super Proton Synchrotron (SPS)
which accelerates the protons to 450\GeV and injects directly into the LHC.

Bunches of protons are accelerated using this system and injected into the LHC
to form two, counter-rotating beams. When the desired number of bunches have
been injected into the LHC, the LHC accelerates them to 4\TeV. When the beams
have reached their nominal energy, they are focused and brought into collision
at four different points on the ring where the various experiments (\ALICE,
\ATLAS, \LHCB, and CMS) are located.

The beams are steered around the accelerator ring by a series of
superconducting, dipole magnets. When running at a center of mass energy of
8\TeV, these magnets operate at roughly 7.5\Tesla. There are also quadrapole
and some higher order magnets around the ring used to focus the beams. The
bunches of protons are accelerated by 16 superconducting radio frequency
cavities. These cavities accelerate slower protons while slowing faster ones,
thereby keeping the proton bunches compact in both real and momentum space.
There is room for 2808 bunches in LHC, although in 2012 there were only
\TODO{how many bunches?}.

The LHC is also the highest luminosity collider in the world. The instantaneous
luminosity is given by
\begin{equation}
    \TODO{\luminosity=fn\frac{N^{2}}{\sigma}}
\end{equation}
where $f$ is the frequency of interaction (which is fixed by the LHC's
circumference), $n$ is the number of bunches in a beam, $N$ is the number of
protons per bunch (with the $N^{2}$ coming from the assumption that there are
the same number of protons in the two colliding bunches), and $\sigma$ is the
area profile of the beams. Although a higher luminosity means more particles
are produced and more data can be collected, the maximum luminosity is limited
by several practical factors. The first factor is cost; a higher luminosity
general requires a more expensive machine as the ring is either made larger or
the technology needed to run the machine is made more complex. The second is
the challenge that higher luminosities present to the analyzers. The luminosity
can be increased by increasing the number of protons in a bunch or by squeezing
the bunches more tightly, but eventually the probability of getting multiple
proton-proton interactions per bunch crossing becomes large, leading to a
phenomenon known as pileup. These extra interactions add additional particles
to the detector and can make it difficult to separate interesting events from
uninteresting background. The luminosity can also be increased by increasing
the number of bunches in the machine, but this decreases the time between the
collisions and leads to a phenomenon known as out of time pileup which can also
obscure interesting events. In 2012, the optimal luminosity was achieved by
running with bunches spaced by 50\ns instead of the design nominal bunch
spacing of 25\ns. The decision to use this bunch spacing was driven by
\linactwo which had trouble injecting high population bunches.

\section{The Compact Muon Solenoid}
\label{cms_section}

\TODO{CMS cutaway detector picture \label{cms_cutaway_fig}}

The Compact Muon Solenoid (CMS) is one of two general purpose particle
detectors built on the LHC ring \cite{cms_tdr_1}\cite{cms_tdr_2}. CMS is
designed to detect the very high energy, sub-atomic particles that are produced
in the LHC's proton-proton collisions. CMS is designed to have a high
acceptance and efficiency for these collisions. Here acceptance indicates the
area in both physical space as well as the area in energy and momentum space in
which the detector can detect particles. Efficiency is the probability that a
particle in CMS's acceptance region is properly measured. In order to have a
high acceptance and efficiency, CMS must be both large---to cover a large area
of physical space---and dense---to cover a large area in momentum and energy
space. CMS is \TODO{21\meters in length, 15\meters in diameter, and weighs
14\kilotonne}.

CMS is built as a series of nested, finite cylinders, where each cylinder is a
separate subdetector. The beams enter the detector along the axis of the
cylinder. The collision point is in the center. There are a pair of endcaps on
either side of the cylinder to increase the acceptance of the detector. The
endcaps and central region of the cylinder (call the barrel) overlap to prevent
particles from escaping undetected through the crack. A cutaway of the detector
is shown in \FIG~\ref{cms_cutaway_fig}.

The coordinate system used by CMS is as follows: the origin is the nominal
interaction point at the center of CMS, the \xaxis is defined to point to the
center of the LHC ring, the \yaxis is defined as vertically up, and the \zaxis
points counter-clockwise and tangent to the LHC ring such that it forms a
right-handed coordinate system with the \xaxis and \yaxis. CMS uses a
cylindrical coordinate system with coordinates ($\eta$, $\phi$). The azimuthal
angel $\phi$ is in the \xyplane measured from the \xaxis while $\eta$ is the
pseudorapidity defined by
\begin{equation}
    \eta = -\logn \tan \frac{\theta}{2}
\end{equation}
where $\theta$ is the polar angle with respects to the \zaxis.

\subsection{Inner Tracking System}

\TODO{$\eta$ coverage, electric field, performance}

The inner tracking system, referred to as the tracker, is the subdetector
closest to the interaction point. The tracker's primary purpose is to measure
the charge of particles, the momentum of these same particles, and the location
of interaction vertices---both the primary vertex, the various secondary
proton-proton vertices from pileup, and the vertices of long-lived particles
like b mesons. The tracker consists two types of silicon detectors: silicon
pixels and silicon strips.

\subsubsection{Pixel Tracker}

\begin{figure}[tb]
    \centering
    \includegraphics[width=\textwidth]{figures/pixel_layout.pdf}
    \caption{A cross-sectional view of the CMS pixel Tracker.}
    \label{fig:pixel_layout}
\end{figure}

The silicon pixels are used in the region closest to the beam pipe where the
particle flux is the highest and hence the finest granularity is needed. Their
primary purpose is to very accurately locate the primary and secondary vertices
in a collision.

There are three barrel layers of the pixel tracker at radii of of 4.3, 7.3, and
10.2\centimeters, each with a length of 53\centimeters. There are two endcap
annular disks as well placed at $|\coordz|=$ 34.5 and 46.5\centimeters. Each of
these disks has an inner radius of 6\centimeters and an outer radius of
15\centimeters.

Each pixel has an area of $100 \times 150 \micrometers^{2}$, but the resolution
of the tracker is better than that because of charge sharing. If a charged
particle ionizes multiple pixels, then a weighted average of the charges can be
used to get sub-pixel resolution on the location of the hit. In order to
increase the charge sharing, a large Lorentz angle (23\degrees) is used. The
blades which make up the endcap disks are fanned out in a turbine-like geometry
with a rotation of 20\degrees to benefit from the same effect. The layout of
the pixel detector is show in \FIG~\ref{fig:pixel_layout}.

\subsubsection{Strip Tracker}

\begin{figure}[tb]
    \centering
    \includegraphics[width=\textwidth]{figures/strip_layout.jpg}
    \caption{A quarter cross-section view of the CMS strip tracker.}
    \label{fig:strip_layout}
\end{figure}

The silicon strips are used further from the beam pipe than the pixels and
cover a much large radius from the beam pipe. The strip tracker has 200 times
the area of the pixel tracker, and so it was not economically feasible to use
the more expensive pixel detectors in this region. The strips' primary purpose
is to measure the momentum and curvature of the charged particles by extending
the tracker's coverage over a larger radius.

Although the strips provide only two points in space to locate a hit (as
compared to three for the pixels), this coarser geometry is adequate because of
the much lower particle flux in this region of the detector. For the momentum,
the location of the hit in the \rphiplane is the most important information,
and so the strips are aligned parallel to the magnetic field. \TODO{Explain
more?}

The strip tracker is itself divided into multiple components. In the barrel
there is the TIB (Tracker Inner Barrel) and the TOB (Tracker Outer Barrel). The
TIB consists of four layers and has coverage up to $|\coordz|<$ 65\centimeters.
The silicon sensors making up the TIB have a strip pitch which varies between
80 to 120\micrometers and have a thickness of 320\micrometers. The first two
layers are constructed with a double layer of modules with a stereo angle of
100\millirads, providing information about the location of the hit in both the
\coordrphi and \rzplane. The TOB consists of six layers and has coverage up to
$|\coordz|<$ 110\centimeters. The TOB has a strip pitch which varies between
120 and 180\micrometers and have a thickness of 500\micrometers. The thicker
sensors are able to be used in this region because the radiation levels are
lower. Having a thicker sensors helps to maintain a high signal-to-noise ratio.
Just like the TIB, the first two layers of the TOB are also build with a double
layer of modules with a stereo angle of 100\millirads.

The ends of the strip tracker consists of the TEC (Tracker Endcap) and the TID
(Tracker Inner Disk). The TEC consists of nine disks in the region
120\centimeters $< |\coordz| <$ 280\centimeters. The TID consists of three
disks in between the end of the TIB and the start of the TEC. The TEC and TID
consist of modules arrayed in rings around the beam line, with the face of each
module pointed towards the interaction point so that they have varying
orientations depending on their distance from the interaction point. The first
two rings of the TID and the first, second, and fifth rings of the TEC are
build with double layer of modules. The modules in the TID and the first three
disks of the TEC have a thickness of 320\micrometers while the rest of the
modules in the TEC have a thickness of 500\micrometers. The strip pitch in the
TID and the first three disks of the TEC have a strip pitch which varies
between 97 and 143\micrometers while the rest of the modules in the TEC have a
strip pitch between 143 and 183\micrometers. The layout of the entire tracker,
including the strip tracker, is shown in \FIG~\ref{fig:strip_layout}.

\subsection{Electromagnetic Calorimeter}

\chapter{Event Reconstruction}
\label{reconstruction_chapter}

\section{Electron Reconstruction}
\label{sec:electron_reconstruction}

Reconstruction of electrons in CMS is complicated by the fact that electrons,
because of their low mass and the high magnetic field, emit photon which must
be accounted for in the final energy sum. Additionally, the amount of material
in front of ECAL causes many electrons begin showering before entering ECAL,
adding additional scattered energy deposits that must be included in the final
sum. Information about the material in front of ECAL is given in
\FIG~\ref{fig:tracker_material}\cite{cms_tracker_2014}.

\begin{figure}[!htbp]
    \centering
    \includegraphics[width=\textwidth]{figures/tracker_material_budget.png}
    \caption[
        Material thickness infront of of ECAL.
    ]{
        The thickness of material, $t$, divided by the radiation length,
        \radiationlength, encountered by particles leaving the nominal
        interaction point before reaching ECAL.
    }
    \label{fig:tracker_material}
\end{figure}

The reconstruction of electrons with $\pt > 20 \GeV$ in CMS starts with an
electron-like cluster of energy in ECAL \cite{eg_reco_2010}. In order to
account for energy lost to photon radiation, additional clusters at constant
$\eta$ but changing $\phi$ are added together to form superclusters
\cite{baffioni_2007}. \TODO{A bit about how various deposits of energy are
connected to form the supercluster.}

From these superclusters, a volume in the tracker where the electron is likely
to have come from is determined by propagating the energy weighted mean
position of the supercluster back through the magnetic field. This spot is then
used to seed  track finding algorithm in the pixel layer. Hits in the tracker
are searched from starting at the innermost layer and working outward. In order
to account for the changing shape of the track as the electron loses energy
from interacting with the tracker material, a ``Gaussian Sum Filter'' (GSF)
\cite{adam_2005} is used rather than the simpler Kalman filter used for muons
and hadrons. Low energy electrons ($\ET < 15 \GeV$) are constructed with \pt
from the tracker and \ET from ECAL. For higher energy electrons, the ECAL
energy is used without the tracker \pt to avoid issues introduced by the
possible poor fits in the tracker. The $\eta$ and $\phi$ of all electron
candidates, regardless of energy, is taken from the track by projecting back to
the interaction point.

\section{Additional Corrections}

Although the measurement of \phistar is insensitive to the energy of the
electrons, the energy still plays a role in determining the electron \pt and
hence whether an event passes the selection criteria used in this analysis
(discussed in \SEC~\ref{ssec:electron_selection}). Therefore, it is important
to accurately measure this quantity, even if it does not directly effect the
final observable. To this end, two sets of centrally produced electron energy
and momentum corrections are applied to both the data and the reconstructed MC
quantities.

\subsection{Regression}

The first set of corrections were calculated using a multivariate regression
trained on \Ztoee and \higgstoZZ MC \cite{cms_an_2012-327}. The regression used
a boosted decision tree trained on 41 different variables parameterizing
electron shower shape, the electron track, and the location of the shower in
EB. The algorithm was trained separately for EB and EE electrons. In order to
prevent over-training the MC samples were split in half, with one half used for
training and the other half used for validation. Electrons used in the
regression were required to have low radiated energy fraction ($< 0.01$) as
determined by the generator level MC and $\pt > 7 \GeV$. The target variable
was the ratio of the gen level bare electron energy over the reconstructed
energy. This correction was applied to both the data and the reconstructed MC
electrons used in this analysis.

\subsection{Energy Scale and Resolution}

The second set of corrections were calculated with \Ztoee MC using two
independent methods \cite{cms_an_2013-253}. The first method was used to
correct for the energy scale while the second method was used to correct for
the resolution.

In the first method, the MC sample and the data were fit with the convolution
of a Breit-Wigner with a Crystal Ball (CB) function. The CB function is used to
model the resolution of the detector and losses due to bremsstrahlung from the
material in front of ECAL. It consists of a Gaussian with a power-law low-side
tail. It was first used by the Crystal Ball Collaboration \cite{oreglia_1980}.
The Breit--Wigner function models the analytic shape predicted for the \Z mass
resonance. The parameters of the Breit--Wigner were fixed to the nominal values
from the Particle Data Group: $\MZ = 91.188 \GeV$, $\GammaZ = 2.495 \GeV$. The
parameters of the CB function were free parameters in the fit.

The scale correction is both time dependent and $\eta$ dependent and so fits
were performed separately in four pseudorapidity bins in various run ranges.
The peak of the CB function in data and MC were compared with the relative
shift taken as the scale correction, $\Delta P$, given by
\EQ~\ref{eq:scale_correction}.

\begin{equation} \label{eq:scale_correction}
    \Delta P = \frac{\Delta m_{\text{data}} - \Delta m_{\text{MC}}}{\MZ}
\end{equation}

In the second method, two categories of electrons are defined: showering
electrons ($\RNine < 0.94$) and non-showering electrons ($\RNine > 0.94$). A
probability density function (PDF) is created using the \Ztoee MC. For each
event in the PDF, the supercluster energy is modified by applying a Gaussian
multiplicative factor $1+\Delta P$ with a standard deviation of $\Delta \sigma$,
where $\Delta P$ is the scale correction and $\Delta \sigma$ models the
resolution. The resolution parameter was selected for each of the two types of
electrons, the four pseudorapidity bins, and various run ranges using a
likelihood maximization. In the EB pseudorapidity bins, there were enough
events to bin in \ET and so these corrections are also \ET dependent.

\section{Electron Variables}
\label{sec:electron_variables}

Reconstructed electrons have multiple variables that indicate their quality.
These variables are broken down into three main categories:

\begin{description}
    \item[Isolation:] \hfill \\
        These variables measures how much energy from other particles is
        deposited near the electron in the detector and are used to help reject
        jets.
    \item[Identification (ID):] \hfill \\
        These variables quantify how much like an electron the reconstructed
        particle looks and are used to reject pions and other charged
        particles.
    \item[Conversion Rejection:] \hfill \\
        These variables are used to reject electrons from \photontoee
        conversions.
\end{description}

\subsection{Isolation}

Hadronic jets sometimes produce electrons in the numerous decays happening
within them. These electrons can be rejected by looking at the sum of the
energy in the tacker, ECAL, and HCAL around the electron, as electrons in jets
will have a large amount of energy surrounding them, whereas electrons from \Z
decays will tend to be isolated.

The isolations used in the trigger (\SingleElectronTrigger) are defined as
follows:

\begin{equation}
    \HCALISO = \DeltaRSum \EHCAL
\end{equation}

\begin{equation}
    \ECALISO = \DeltaRSum \EECAL - \ESC
\end{equation}

Where \DeltaRSum is a sum on the energy in a $\Delta R < 0.3$ cone around the
supercluster location, \EHCAL is the energy in HCAL, \EECAL is the energy in
ECAL, and \ESC is the energy of the supercluster, which is subtracted out of
the ECAL isolation sum. No subtraction is applied to the HCAL isolation as we
expect all of the electron's energy to be contained in ECAL.

A different isolation variable is defined for offline selection that is more
expensive to compute but takes advantage of the tracker as well as ECAL and
HCAL. This isolation uses a \particleflow\cite{particle_flow_2010} technique
which is a method of reconstructing jets that uses information for every
subdetector and tries to reconstruct the individual particles in a jet by
matching them to their responses in the various subdetectors. To keep the
algorithm simple, \particleflow categorizes every particle into one of five
types: photons, electrons, muons, charged hadrons, and neutral hadrons. A
photon is a particle with energy only deposited in ECAL. An electron is a
particle with an ECAL energy deposit and a track. A muon is a track in the
central tracker matched to a track in the muon system. A charged hadron is any
energy cluster in HCAL with a matching ECAL cluster and track. A neutral hadron
is any energy cluster in HCAL with a matching ECAL cluster without a matching
track. These particle flow jets are used to calculate an energy per area due to
pileup, $\rho$, in the detector which is used to remove the pileup contribution
from the isolation sum. The \particleflow isolation, $\PFISO$, is given by:

\begin{equation}
    \PFISO = \DeltaRSum \left(\ptTrack + \EECAL + \EHCAL\right) - \ptElectron
    - \ESC - 0.3^{2} \pi \rho
\end{equation}

Where the variables are the same as above, with the addition of \ptTrack, which
is the \pt of all tracks in the tracker, \ptElectron, which is the \pt of the
electron's track, and $\left(0.3^{2} \pi \rho\right)$, which is the energy
around the electron calculated from particle flow. A comparison of the
distribution of \PFISO in a minimally biased ensemble of electrons from events
selected with a muon trigger and the same distribution in the \MADGRAPH signal
sample is shown in \FIG~\ref{fig:pf_iso}; a similar comparisons distributions
of \HCALISO and \ECALISO are shown in \FIGS~\ref{fig:hcal_iso} and
\ref{fig:ecal_iso}.

\begin{figure}[!htbp]
    \centering
    \includegraphics[width=\StackedPlotWidth]{figures/e_reco_var_iso.pdf}
    \caption[
        Distributions of particle flow isolation variables in data and MC.
    ]{
        The particle flow isolation variable distribution for all electrons
        with $\pt > 20 \GeV$ and $|\eta| < 2.4$ in a set of events selected
        with a muon trigger (circles) and in \MADGRAPH \Ztoee MC (triangles).
    }
    \label{fig:pf_iso}
\end{figure}

\begin{figure}[!htbp]
    \centering
    \begin{subfigure}[b]{\StackedPlotWidth}
        \includegraphics[width=\textwidth]{figures/e_reco_var_hcal_iso.pdf}
        \caption{}
        \label{fig:hcal_iso}
    \end{subfigure}
    \begin{subfigure}[b]{\StackedPlotWidth}
        \includegraphics[width=\textwidth]{figures/e_reco_var_ecal_iso.pdf}
        \caption{}
        \label{fig:ecal_iso}
    \end{subfigure}
    \caption[
        Distributions of HCAL and ECAL isolation variables in data and MC.
    ]{
        The HCAL (top) and ECAL (bottom) isolation variable distributions for
        all electrons with $\pt > 20 \GeV$ and $|\eta| < 2.4$ in a set of
        events selected with a muon trigger (circles) and in \MADGRAPH \Ztoee
        MC (triangles).
    }
    \label{fig:hcal_ecal_isos}
\end{figure}

\subsection{Identification}

The shape of the electromagnetic shower in the calorimeters is used to
discriminate between electrons and other particles. Electrons generally have
very narrow showers whereas hadronic particles have wide showers. The size of
the shower in $\eta$ is characterized by \sigmaietaieta. Electron showers are
mostly contained within ECAL and so the ratio of energy around the hit in HCAL
over the energy around the hit in ECAL, \HOverE, is also used to parameterize
the shower shape.

The distance between the track and the supercluster in \coordetaphi space is
given by \dphiin and \detain. The compatibility of energy of the supercluster
and the momentum of the track is parameterized by \ooeoop. Using these
variables photons can be rejected as they interact in ECAL but leave no track
in the tracker. A comparison of the distribution of $\HOverE$ is shown in
\FIG~\ref{fig:he}; compairisons of the distributions of $\sigmaietaieta$ and of
$\ooeoop$ are shown in \FIGS~\ref{fig:sieie} and \ref{fig:ooeoop}; comparisons
of the distributions of $\detain$ and $\dphiin$ are shown in
\FIGS~\ref{fig:deta} and \ref{fig:dphi}.

\begin{figure}[!htbp]
    \centering
    \includegraphics[width=\StackedPlotWidth]{figures/e_reco_var_he.pdf}
    \caption[
        Distributions of \HOverE variable in data and MC.
    ]{
        The \HOverE variable distribution for all electrons with $\pt > 20
        \GeV$ and $|\eta| < 2.4$ in a set of events selected with a muon
        trigger (circles) and in \MADGRAPH \Ztoee MC (triangles).
    }
    \label{fig:he}
\end{figure}

\begin{figure}[!htbp]
    \centering
    \begin{subfigure}[b]{\StackedPlotWidth}
        \includegraphics[width=\textwidth]{figures/e_reco_var_sigma_ieta_ieta.pdf}
        \caption{}
        \label{fig:sieie}
    \end{subfigure}
    \begin{subfigure}[b]{\StackedPlotWidth}
        \includegraphics[width=\textwidth]{figures/e_reco_var_1oe_1op.pdf}
        \caption{}
        \label{fig:ooeoop}
    \end{subfigure}
    \caption[
        Distributions of $\sigmaietaieta$ and $\ooeoop$ in data and MC.
    ]{
        The $\sigmaietaieta$ (top) and $\ooeoop$ (bottom) variable
        distributions for all electrons with $\pt > 20 \GeV$ and $|\eta| < 2.4$
        in a set of events selected with a muon trigger (circles) and in
        \MADGRAPH \Ztoee MC (triangles).
    }
    \label{fig:sieie_ooeoop}
\end{figure}

\begin{figure}[!htbp]
    \centering
    \begin{subfigure}[b]{\StackedPlotWidth}
        \includegraphics[width=\textwidth]{figures/e_reco_var_deta.pdf}
        \caption{}
        \label{fig:deta}
    \end{subfigure}
    \begin{subfigure}[b]{\StackedPlotWidth}
        \includegraphics[width=\textwidth]{figures/e_reco_var_dphi.pdf}
        \caption{}
        \label{fig:dphi}
    \end{subfigure}
    \caption[
        Distributions of $\detain$ and $\dphiin$ in data and MC.
    ]{
        The $\detain$ (top) and $\dphiin$ (bottom) variable distributions for
        all electrons with $\pt > 20 \GeV$ and $|\eta| < 2.4$ in a set of
        events selected with a muon trigger (circles) and in \MADGRAPH \Ztoee
        MC (triangles).
    }
    \label{fig:dtrack}
\end{figure}

\subsection{Conversion Rejection}

Conversions generally happen away from the vertex and so the distances of the
hits in the track from the vertex are useful quantities to reject conversions.
The transverse and longitudinal separation between the track and the primary
vertex are given by $d_{0}$ and $d_{z}$. Conversions generally happen deep in
the tracker and so tracks from them will have missing layers, given by \nmiss.
Conversions also have low vertex fit probability, \pvtx, indicating that their
track likely did not come from the primary vertex. A comparison of the
distributions of $d_{0}$ and of $d_{z}$ are shown in \FIGS~\ref{fig:d0} and
\ref{fig:dz}.

\begin{figure}[!htbp]
    \centering
    \begin{subfigure}[b]{\StackedPlotWidth}
        \includegraphics[width=\textwidth]{figures/e_reco_var_d0.pdf}
        \caption{}
        \label{fig:d0}
    \end{subfigure}
    \begin{subfigure}[b]{\StackedPlotWidth}
        \includegraphics[width=\textwidth]{figures/e_reco_var_dz.pdf}
        \caption{}
        \label{fig:dz}
    \end{subfigure}
    \caption[
        Distributions of $d_{0}$ and $d_{z}$ in data and MC.
    ]{
        The $d_{0}$ (top) and $d_{z}$ (bottom) variable distributions for all
        electrons with $\pt > 20 \GeV$ and $|\eta| < 2.4$ in a set of events
        selected with a muon trigger (circles) and in \MADGRAPH \Ztoee MC
        (triangles).
    }
    \label{fig:d0_dz}
\end{figure}


% The data and simulation we use
\chapter{Data and Simulation Samples}
\label{chatper:data_and_mc_samples}

\section{Data}

The data used in this analysis were collected by the CMS detector in 2012 at a
center-of-mass energy of \rootseight. The LHC delivered 23 \fbinv of integrated
luminosity during the year as seen in \FIG~\ref{fig:2012_luminosity}. This
period was divided into four run eras refered to as 2012A, B, C, and D. During
an era, the LHC run parameters are kept roughly static to allow for consistent
data taking conditions. In between eras, maintenance and beam adjustments were
performed on the LHC in order to deliver higher luminosity.

\begin{figure}[!htbp]
    \centering
    \includegraphics[width=\textwidth]{figures/2012_lumi.pdf}
    \caption[
        The integrate luminosity delivered and recorded by CMS in 2012
    ]{
        The integrate luminosity delivered and recorded by CMS in 2012. The
        flat periods in May, July, and September correspond to the boundaries
        between the run eras.
    }
    \label{fig:2012_luminosity}
\end{figure}

The data collected by CMS are split into smaller datasets based on the physics
objects contained within the events. This allows analyses to use only one or
two datasets, instead of requiring them to deal with the entirety of the CMS
data (which is many petabytes, and hence too large for most institutes to store
locally). The HLT sorts events into the various datasets based on the triggers
that the event fired. In this manner, an event can end up in multiple datasets
if it fired multiple triggers. This analysis uses the \SingleElectron dataset
which was collected with the HLT trigger \SingleElectronTrigger. These datasets
were reconstructed---converted from raw detector response into physics
objects---in January 2013, in order to make use of the most recent
calibrations derived from the entire 2012 run. A summary of the datasets used
is provided in \TAB~\ref{table:datasets}.

% table:datasets
\begin{table}[h]
    \centering
    \spacerows{1.2}
    \begin{center}
        \begin{tabular}{@{}l l r@{}}
            \toprule
            Dataset Name                          & Run Dates                & Luminosity       \\
            \midrule
            /SingleElectron/Run2012A-22Jan2013-v1 & 2012-04-05 to 2012-05-08 & \SI{876.225}{\per\pico\barn} \\
            /SingleElectron/Run2012B-22Jan2013-v1 & 2012-05-10 to 2012-06-18 & \SI{4.412}{\per\femto\barn}   \\
            /SingleElectron/Run2012C-22Jan2013-v1 & 2012-07-01 to 2012-09-27 & \SI{7.055}{\per\femto\barn}   \\
            /SingleElectron/Run2012D-22Jan2013-v1 & 2012-09-28 to 2012-12-06 & \SI{7.369}{\per\femto\barn}   \\
            \bottomrule
        \end{tabular}
    \end{center}
    \caption[
        Summary of datasets.
    ]{
        The name, run dates, and integrated luminosity of the datasets used in this
        analysis
    }
    \label{table:datasets}
\end{table}


Although there is a \DoubleElectron dataset which uses a trigger designed to
find Z bosons, this analysis uses the \SingleElectron dataset selected with the
\SingleElectronTrigger trigger. The primary motivation behind using this
trigger was to allow a direct comparison with a similar \phistar analysis being
performed by CMS which used \Ztomumu events selected with a single muon
trigger. The single electron trigger requires an electron with $\pt > 27$ which
passes Working Point 80 (\WPEighty). \WPEighty is a set of selection
requirements on lepton isolation and shower shape variables designed to be 80\%
efficient in selecting real electrons. The requirements that make up \WPEighty
are listed in \TAB~\ref{table:wp80}. This trigger had the lowest \pt threshold
of any single electron trigger that was unprescaled run during 2012. To
prescale a trigger means to apply a rate reduction by randomly throwing out a
certain fraction of events in order to keep the total trigger rate manageable;
as this trigger was unprescaled, no events were discarded in this manner.

% table:wp80
\begin{table}[h]
    \centering
    \spacerows{1.2}
    \begin{center}
        \begin{tabular}{@{}l r r@{}}
            \toprule
            Value                      & EB     & EE     \\
            \midrule
            $|\eta| <$                 & 1.4791 & 2.65   \\
            $\pt >$                    & 27     & 27     \\
            $\sigmaietaieta <$         & 0.1    & 0.03   \\
            $\HOverE <$                & 0.1    & 0.05   \\
            $\ECALISO <$               & 0.15   & 0.1    \\
            $\HCALISO <$               & 0.1    & 0.1    \\
            Pixel Matching $\ge$       & 1      & 1      \\
            $|\ooeoop| <$              & 0.05   & 0.05   \\
            $|\detain| <$              & 0.007  & 0.007  \\
            $|\dphiin| <$              & 0.06   & 0.03   \\
            \bottomrule
        \end{tabular}
    \end{center}
    \caption[
        The selection requirements for the \SingleElectronTrigger trigger.
    ]{
        The selection requirements for the \SingleElectronTrigger trigger for
        electrons which end up in the barrel region or the endcap region of ECAL.
        The variables used are detailed in \cref{sec:electron_variables}.
    }
    \label{table:wp80}
\end{table}


The events from the \SingleElectron sample are further filtered for quality. A
centrally-produced list of good luminosity segments is used to select only
events in which no part of the detector was malfunctioning or disabled. After
accounting for detector dead time and beam quality, \GoodLumiNumber of
integrated luminosity are used for physics analysis.

\section{Monte Carlo}

This analysis makes use of numerous simulated data samples---colloquially
referred to as Monte Carlo (MC)---in order to estimate backgrounds and signal
yields, derive scale factors, and correct for the effects of bin migration on
the final measurement.

\subsection{Monte Carlo Generators}

There are multiple MC generators, each of which makes its own assumptions about
the behavior of the various particles and interactions it simulates. In
generally, the process of generating a simulated a proton-proton collision
event is broken up into multiple steps. These steps can all be performed by one
program, but in general are handled by several. These steps are summarized
below:

\begin{description}

    \item[Parton distributions:] In order to simulate the collision of two
        protons, a set of PDFs is needed to calculate the probability of
        finding a parton with a specific \BjorkenX{i}. PDFs are provided by
        various theory groups and MC software can generally run with multiple
        different PDF sets.

    \item[Initial state shower:] A shower-initiating parton from each beam of
        protons initiates a sequence of partons branchings (for example, $q \to
        qg$). One parton from each of these showers will go on to interact in
        the hard scattering process, the others will become initial state
        radiation. These shower initiators are given non-zero \pt to account
        for the primordial transverse momentum of the partons in the proton.

    \item[Hard scattering process:] In the hard scattering the two partons
        collide and a set of out going particles is generated based on the
        matrix element level calculations.

    \item[Resonances and decays:] Some of the outgoing particles may be
        resonances (like the \W and \Z) which decay, while others may be stable
        (but which may still radiate, for example, $e \to e \photon$). In
        either case, these particles and their decay chains are simulated in
        detail.

    \item[Multiple interactions and the underlying event:] In addition to the
        hard scattering, there are further interactions from other partons in
        the shower. After the partons from these interaction are removed from
        the protons, what is left are ``beam remnants'' that may have internal
        color and charge. Additionally, these remnants can have non-zero \pt as
        the shower initiators had non-zero \pt. These remnants are still color
        connected to the hard interaction, and must decay accordingly.

    \item[Hadronization:] Bare quarks and gluons can not be observed, so the
        must be combined into colorless objects. The exact manner in which this
        happens is not understood from QCD, and so the different MC software
        suites use different internally developed models to perform
        hadronization.

    \item[Detector Simulation:] After the event is generated and all of the
        particles decayed, the event is run through a simulation of the
        detector which particle interactions with mater are simulated along
        with the response of the detector. This step is almost always done by a
        stand alone program as the job of simulating a detector is very
        different (and almost wholly separable from) simulating a collision and
        the resulting event.

\end{description}

In the MC samples that we use, that parton distributions, initial state shower,
hard scattering, and resonances and decays are handled by \MADGRAPH, \POWHEG,
and \PYTHIAsix, as indicated in \TAB~\ref{table:mc}. All of the samples use
\PYTHIAsix for the underlying event and hadronization. The CMS detector is
simulated using \GEANTfour.

MC programs have multiple different parameters that control their behavior in
each of these steps. As not all of these parameters can be derived from QCD,
and so many of them must be ``tuned'' to best fit the data. The parameters are
correlated and so need to be fit together. One such example of tuned parameters
are the various PDF inputs, which are generally produced by theory groups by
fitting their parton models to collider data. Parameters that control
hadronization and final-state showers are fit using data from \LEP, as it
offers the cleanest environment as the incoming electrons have no color charge.
Parameters that control multiparton interactions and initial state radiation
are fit using earlier LHC data.

\subsection{Monte Carlo Datasets}
\label{ssec:monte_carlo}

All the MC used in this analysis were centrally-generated by the CMS
collaboration. A \DYtoll signal sample and a $\ttbar\text{+jets}$ background
sample were generated with \MADGRAPH \cite{alwall2014}. Diboson ($\Z\Z$,
$\W\Z$, $\W\W$) background samples were generated with \PYTHIAsix
\cite{sjostran2006}. Background samples consisting of $\tbar \W$, $t \W$, and
$\DYtotautau$ were generated using \POWHEG
\cite{nason2004}\cite{alioli2010}\cite{re2011}. A secondary signal MC was also
generated with \POWHEG. The details of these samples are listed in
\TAB~\ref{table:mc}.

All of the MC samples use \PYTHIAsix with the \ZTwoStar tune for modeling the
underlying event and hadronization except for the di-boson samples. The
\ZTwoStar tune was centrally-produced by the CMS collaboration by tuning
\PYTHIAsix to match the 2011 CMS data. \Tauola is used for tau decays in all of
the samples except for \ttbar and \POWHEG $\text{Drell-Yan} \to \ee$ MC samples
\cite{was_2007}.

% table:mc
\begin{table}[h]
    \centering
    \spacerows{1.2}
    \begin{center}
        \begin{tabular}{@{}l l l r r@{}}
            \toprule
            Process     & Requirements           & Generator   & $\sigma$ (pb) & Events $(\times 10^{6})$ \\
            \midrule
            \DYtoll     & $\mll > \SI{50}{\GeV}$ &  \MADGRAPH  & 3531.9 (NNLO) & 30.460 \\
            \DYtoee     & $\mll > \SI{20}{\GeV}$ &  \POWHEG    & 1966.7        & 3.297  \\
            \DYtotautau & $\mll > \SI{20}{\GeV}$ &  \POWHEG    & 1966.7        & 3.297  \\
            \ttbar      &                        &  \MADGRAPH  & 23.64         & 3.984  \\
            \tWdecay    &                        &  \POWHEG    & 11.1          & 0.498  \\
            \tbarWdecay &                        &  \POWHEG    & 11.1          & 0.493  \\
            \WW         &                        &  \PYTHIAsix & 54.84         & 10.000 \\
            \WZ         &                        &  \PYTHIAsix & 33.21         & 10.000 \\
            \ZZ         &                        &  \PYTHIAsix & 17.7          & 9.800  \\
            \bottomrule
        \end{tabular}
    \end{center}
    \caption[
        Summary of MC samples.
    ]{
        Summary of the MC samples used in this analysis. The Drell-Yan MC
        samples have mass requirements on the events while the other MC samples
        have none. All cross sections are NLO unless otherwise stated.
    }
    \label{table:mc}
\end{table}


After the generation step, MC is sent through a full detector simulation which
uses \GEANTfour \cite{agostinelli2003} to mimic the detector response. This
detector response is reconstructed using the full CMS reconstruction chain to
produce MC files in a format identical to actual data.

MC events have additional simulated minimum-bias events overlaid on top of them
to better match the conditions found in actual running. These minimum-bias
events attempt to simulate what a typical proton-proton interaction looks like
in the detector and are used to mimic pileup. The number of pileup events to be
added is drawn from a distribution decided upon before data is taken and so it
events must be reweighted after the data taking period is over to make the
number of pileup in MC match the same quantity in data. The data distribution
is calculated based on the instantaneous luminosity and the inelastic
proton-proton cross section. The ratio of these measurements is used to
reweight the MC.

\section{Scale Factors}
\label{sec:scale_factors}

The detector response to various signals is not always perfectly simulated in
MC and so the efficiencies of various selection requirements are not the same
in data and MC. In order to correct for this difference, each event in MC is
reweighed with a series of scale factors which are the efficiency of some
selection requirement in data divided by the same efficiency as measured on MC,
as follows:

\begin{equation}
    \label{eq:sf}
    \text{SF} = \frac{\effdata}{\effmc}
\end{equation}

Three scale factors are applied to each event: trigger, reconstruction, and
identification. The trigger scale factors were measured by us and are detailed
in \SEC~\ref{ssec:sf_trigger} while the reconstruction and identification scale
factors were measured centrally by the CMS collaboration. The
centrally-produced values are used as doing so is a requirement of passing the
internal analysis review. The methods used to measure the reconstruction and
isolation scale factors are summarized in \SECS~\ref{ssec:sf_reconstruction}
and \ref{ssec:sf_id} because the papers detailing them are not public

\subsection{Tag and Probe}

Tag and Probe (\TnP) is a minimally-biased method of calculating the efficiency
of some analysis selection requirement. \TnP takes advantage of the well-known
mass and narrow width of the \Z boson to select a set of electrons for which
very few selection requirements have been applied. This is done by finding one
high-quality electron, the tag, and another minimally-biased object, the probe,
that could be an electron, such as a supercluster. The invariant mass of these
objects is computed and if it is near the \Z mass peak, it is very likely that
the probe is also an electron.

Once a set of minimally-biased probe electrons is constructed, the selection
requirement can be applied to them. The efficiency of that requirement is then
the number of probes that passs divided by the total number in the sample as
follows:

\begin{equation}
    \label{eq:eff}
    \eff = \frac{
        \text{n}^{\text{obs.}}_{\text{pass}}
    }{
        \text{n}^{\text{obs.}}_{\text{total}}
    }
\end{equation}

\subsection{Single Electron Trigger}
\label{ssec:sf_trigger}

The efficiency of the HLT trigger used in this analysis,
\SingleElectronTrigger, is measured using \TnP on the primary dataset. The
efficiency is measured in bins of probe \pt and probe $\eta$ with bin
boundaries of \{30, 40, 50, 70, 250\} in \pt and \{-2.1, -2.0, -1.556, -1.442,
-0.8, 0., 0.8, 1.442, 1.556, 2.0, 2.1\} in $\eta$.

Both the tag electron and the probe electron are required to satisfy $|\eta| <
2.1$, $\pt > 30$, and to pass \EGTIGHT requirements. These requirements are the
same as required of the \CentralElectron in the full analysis selection, and
hence the efficiency is measured relative to that selection. The pair must have
an invariant mass such that \MassRange. The tag electron is required to be
matched to an electron that fired the trigger with $\Delta R < 0.3$. There is
no requirement placed on the charge of the electron pair. Events with three or
more electrons that pass these requirements are rejected.

Probes are considered passing if they are also matched to an electron that
fired the trigger with $\Delta R < 0.3$, and failing otherwise. The efficiency
in each bin is the number of passing probes divided by the number of failing
probes, where the number of passing and failing probes is determined by using a
simple count. In an individual event, both electrons are tried as a tag so that
an event may contribute to the efficiency measurement twice if both electrons
pass the tag requirements.

The efficiency is computed in exactly the same way on the \MADGRAPH sample. The
MC events are reweighted for pileup, reconstruction efficiency, and
identification efficiency before the trigger efficiency is measured. The
measured efficiencies for data and MC are listed in
\TABS~\ref{table:trigger_eff_data} and \ref{table:trigger_eff_mc}, respectively.

In the case of the trigger, because either electron could cause the event to
pass, the scale factors can not be computed for each bin, but instead must be computed
for each pair of bins. If only one electron in the event has $\pt > 30$ and
$|\eta| < 2.1$ than the scale factor is simply that given by \EQ~\ref{eq:sf}, but if both
electrons pass the requirements than either could have fired the trigger and so
the scale factor is given by:

\begin{equation} \label{eq:sf_double}
    \text{SF}_{1 \text{ or } 2}
    =
    \frac{
        1 - \left( 1 - \effdata_{0} \right) \left( 1 - \effdata_{1} \right)
    } {
        1 - \left( 1 - \effmc_{0} \right) \left( 1 - \effmc_{1} \right)
    }
\end{equation}

Where $\effdata_{0,1}$ is the efficiency as measured in data for the 0th and
1st electrons, and $\effmc_{0,1}$ is the efficiency as measured in MC.
Equation~\ref{eq:sf_double} is just the probability that one or both of the
electrons fired the trigger divided by the same quantity in MC. This equation
assumes that the probability of one electron firing the trigger is uncorrelated
with the probability of the other electron firing the trigger.

% table:trigger_eff_data
\begin{table}[h]
    \centering
    \spacerows{1.5}
    \begin{center}
        \begin{tabular}{@{}l r r r r r@{}}
            \toprule
            $\eta$ & 30--40 \GeV & 40--50 \GeV & 50--70 \GeV & 70--250 \GeV  \\
            \midrule
            \numrange{-2.1}{-2} & $0.741^{+0.003}_{-0.003}$ & $0.773^{+0.003}_{-0.003}$ & $0.780^{+0.005}_{-0.005}$ & $0.79^{+0.01}_{-0.01}$  \\
            \numrange{-2}{-1.556} & $0.734^{+0.001}_{-0.001}$ & $0.772^{+0.001}_{-0.001}$ & $0.786^{+0.002}_{-0.002}$ & $0.792^{+0.005}_{-0.005}$  \\
            \numrange{-1.556}{-1.442} & $0.725^{+0.003}_{-0.003}$ & $0.821^{+0.002}_{-0.002}$ & $0.809^{+0.004}_{-0.004}$ & $0.848^{+0.010}_{-0.010}$  \\
            \numrange{-1.442}{-0.8} & $0.8930^{+0.0005}_{-0.0005}$ & $0.9396^{+0.0003}_{-0.0004}$ & $0.9509^{+0.0006}_{-0.0006}$ & $0.966^{+0.001}_{-0.001}$  \\
            \numrange{-0.8}{0} & $0.9213^{+0.0004}_{-0.0004}$ & $0.9528^{+0.0002}_{-0.0002}$ & $0.9601^{+0.0004}_{-0.0004}$ & $0.9692^{+0.0010}_{-0.0010}$  \\
            \numrange{0}{0.8} & $0.9174^{+0.0004}_{-0.0004}$ & $0.9473^{+0.0003}_{-0.0003}$ & $0.9561^{+0.0004}_{-0.0004}$ & $0.963^{+0.001}_{-0.001}$  \\
            \numrange{0.8}{1.442} & $0.8964^{+0.0005}_{-0.0005}$ & $0.9424^{+0.0003}_{-0.0003}$ & $0.9533^{+0.0006}_{-0.0006}$ & $0.966^{+0.001}_{-0.001}$  \\
            \numrange{1.442}{1.556} & $0.714^{+0.003}_{-0.003}$ & $0.823^{+0.002}_{-0.002}$ & $0.827^{+0.004}_{-0.004}$ & $0.861^{+0.009}_{-0.010}$  \\
            \numrange{1.556}{2} & $0.758^{+0.001}_{-0.001}$ & $0.800^{+0.001}_{-0.001}$ & $0.811^{+0.002}_{-0.002}$ & $0.823^{+0.005}_{-0.005}$  \\
            \numrange{2}{2.1} & $0.764^{+0.003}_{-0.003}$ & $0.792^{+0.002}_{-0.002}$ & $0.797^{+0.005}_{-0.005}$ & $0.82^{+0.01}_{-0.01}$  \\
            \bottomrule
        \end{tabular}
    \end{center}
    \caption{
        The electron trigger efficiency in data.
    }
    \label{table:trigger_eff_data}
\end{table}


% table:trigger_eff_mc
\begin{table}[h]
    \centering
    \spacerows{1.5}
    \begin{center}
        \begin{tabular}{@{}l r r r r r@{}}
            \toprule
            $\eta$ & 30--40 \GeV & 40--50 \GeV & 50--70 \GeV & 70--250 \GeV  \\
            \midrule
            \numrange{-2.1}{-2} & $0.734^{+0.004}_{-0.004}$ & $0.769^{+0.004}_{-0.004}$ & $0.771^{+0.008}_{-0.008}$ & $0.76^{+0.02}_{-0.02}$  \\
            \numrange{-2}{-1.556} & $0.736^{+0.002}_{-0.002}$ & $0.768^{+0.002}_{-0.002}$ & $0.779^{+0.003}_{-0.003}$ & $0.789^{+0.008}_{-0.008}$  \\
            \numrange{-1.556}{-1.442} & $0.791^{+0.004}_{-0.004}$ & $0.847^{+0.003}_{-0.003}$ & $0.850^{+0.006}_{-0.006}$ & $0.87^{+0.01}_{-0.02}$  \\
            \numrange{-1.442}{-0.8} & $0.9395^{+0.0006}_{-0.0006}$ & $0.9612^{+0.0004}_{-0.0004}$ & $0.9690^{+0.0007}_{-0.0008}$ & $0.980^{+0.002}_{-0.002}$  \\
            \numrange{-0.8}{0} & $0.9469^{+0.0005}_{-0.0005}$ & $0.9670^{+0.0003}_{-0.0003}$ & $0.9745^{+0.0005}_{-0.0005}$ & $0.982^{+0.001}_{-0.001}$  \\
            \numrange{0}{0.8} & $0.9466^{+0.0005}_{-0.0005}$ & $0.9665^{+0.0003}_{-0.0003}$ & $0.9739^{+0.0005}_{-0.0006}$ & $0.982^{+0.001}_{-0.001}$  \\
            \numrange{0.8}{1.442} & $0.9364^{+0.0007}_{-0.0007}$ & $0.9597^{+0.0004}_{-0.0004}$ & $0.9668^{+0.0008}_{-0.0008}$ & $0.979^{+0.002}_{-0.002}$  \\
            \numrange{1.442}{1.556} & $0.779^{+0.004}_{-0.005}$ & $0.841^{+0.003}_{-0.003}$ & $0.842^{+0.006}_{-0.006}$ & $0.86^{+0.02}_{-0.02}$  \\
            \numrange{1.556}{2} & $0.749^{+0.002}_{-0.002}$ & $0.786^{+0.002}_{-0.002}$ & $0.798^{+0.003}_{-0.003}$ & $0.810^{+0.008}_{-0.008}$  \\
            \numrange{2}{2.1} & $0.737^{+0.004}_{-0.004}$ & $0.769^{+0.004}_{-0.004}$ & $0.779^{+0.007}_{-0.008}$ & $0.82^{+0.02}_{-0.02}$  \\
            \bottomrule
        \end{tabular}
    \end{center}
    \caption{
        The electron trigger efficiency in \MADGRAPH MC.
    }
    \label{table:trigger_eff_mc}
\end{table}


\subsection{Electron Reconstruction}
\label{ssec:sf_reconstruction}

Electron reconstruction begins with the assembly of a supercluster in ECAL and
ends with the matching of a supercluster to a track in the tracker. The details
of electron reconstruction are described in
\SEC~\ref{sec:electron_reconstruction}. The efficiency of an electron with $\pt
> 20 \GeV$ depositing enough energy in ECAL to be reconstructed into a
supercluster is very high, although the exact efficiency must be measured in
MC as there is no more basic object with which to perform \TnP to measure it in
data. Failure to form superclusters is generally due to dead crystals in ECAL,
which are accounted for in the detector response simulation. Scale factors for
matching a track given that a supercluster has already been found were measured
centrally by the CMS collaboration using \TnP \cite{gsf_scale_factors_2013}. A
summary of their method follows.

The events used to measure the reconstruction scale factors are selected with the
dedicated electron \TnP Trigger: \TnPTrigger. This trigger requires one
electron with $\pt > 20 \GeV$ which must also pass very tight isolation and ID
requirements while requiring only a low energy ($\et > 4 \GeV$) supercluster as
the other leg. The trigger rate is kept down by requiring that the invariant
mass of these two objects is greater than $50 \GeV$.

The events selected by the trigger are further required to pass a set of
selection requirements. The tag electron is required to pass \EGTIGHT, have
$\pt > 25 \GeV$, and $|\eta| < 2.5$. Electrons are rejected if they fall in the
seam between EB and EE ($1.4442 < |\eta| < 1.566$). The tag must also be
matched to the tight leg of the \TnP trigger. The probe supercluster has
minimal requirements applied; it is required to have tracker isolation $<
0.15$. For the MC sample, the tag is required to be matched to a generator
level electron with $\Delta R < 0.2$. Additionally, the event is required to
have low particle flow missing energy $\PFMET < 20 \GeV$ in order to reject
poorly reconstructed events.

The events were binned in terms of probe's $\pt$ and $\eta$ as well as whether
the probe passed or failed. In each bin, the \mee distribution was constructed
and a template consisting of the sum of a Gaussian smeared \Ztoee MC sample and
an exponential background was fitted. The number of events predicted by the
signal fit on the passing sample, failing sample, and sum of the two samples
was used to get the efficiency. A similar process was performed on MC, although
instead of a fit a simple counting of passing events was performed (as there is
no background in MC). The resulting scale factors are given in
\TAB~\ref{table:gsf_scale_factor}.

% table:gsf_scale_factor
\begin{table}[h]
    \centering
    \spacerows{1.5}
    \begin{center}
        \begin{tabular}{@{}l r r r r@{}}
            \toprule
            $|\eta|$                 & \GeVRange{20}{30}                  & \GeVRange{30}{40}                  & \GeVRange{40}{50}                  & $> \SI{50}{\GeV}$ \\
            \midrule
            \numrange{0.0}{0.8}      & $\effstatsys{0.982}{0.003}{0.012}$ & $\effstatsys{0.988}{0.001}{0.008}$ & $\effstatsys{0.990}{0.001}{0.004}$ & $\effstatsys{0.990}{0.001}{0.004}$ \\
            \numrange{0.8}{1.4442}   & $\effstatsys{0.993}{0.002}{0.012}$ & $\effstatsys{0.993}{0.001}{0.008}$ & $\effstatsys{0.993}{0.001}{0.004}$ & $\effstatsys{0.991}{0.001}{0.004}$ \\
            \numrange{1.4442}{1.566} & $\effstatsys{1.016}{0.012}{0.020}$ & $\effstatsys{0.985}{0.004}{0.009}$ & $\effstatsys{0.987}{0.004}{0.004}$ & $\effstatsys{0.974}{0.009}{0.006}$ \\
            \numrange{1.566}{2.0}    & $\effstatsys{0.988}{0.003}{0.012}$ & $\effstatsys{0.993}{0.002}{0.008}$ & $\effstatsys{0.992}{0.001}{0.004}$ & $\effstatsys{0.990}{0.003}{0.004}$ \\
            \numrange{2.0}{2.5}      & $\effstatsys{1.002}{0.004}{0.012}$ & $\effstatsys{1.004}{0.002}{0.008}$ & $\effstatsys{1.005}{0.002}{0.004}$ & $\effstatsys{0.998}{0.004}{0.004}$ \\
            \bottomrule
        \end{tabular}
    \end{center}
    \caption[
        Scale factors for GSF electron reconstruction.
    ]{
        Scale factors for GSF electron reconstruction. The upper uncertainty listed
        is statistical, the lower is systematic.
    }
    \label{table:gsf_scale_factor}
\end{table}


\subsection{Electron Identification}
\label{ssec:sf_id}

Not all electrons which are reconstructed pass the ID criteria used in this
analysis, specifically \EGMEDIUM and \EGTIGHT, the details of which are covered
in \SEC~\ref{sec:cut_based_id}. The efficiency of going from a
reconstructed electron to one which passes the identification criteria is
measured centrally by the CMS collaboration using \TnP \cite{cms_an_2014-055}.
A summary of their method follows.

The events used for this measurement were selected using two triggers: \\
\TnPTrigger, which is described above, and \TnPTriggerSecond, which requires
one electron with $\pt > 17 \GeV$ and tight isolation and ID requirements while
also requiring a reconstructed second electron (as opposed to a supercluster as
required by the first trigger) with $\pt > 8 \GeV$. It further requires a
dielectron invariant mass of $\mee > 50 \GeV$.

The tag electrons are required to pass \EGTIGHT, have $\pt > 25 \GeV$, and
$|\eta| < 2.5$; they are rejected if they fall in the seem between EB and EE
($1.4442 < |\eta| < 1.566$). The tag is not required to match the trigger.
Probe electrons have the same $\eta$ requirements as tags, but are only
required to have $\pt > 10 \GeV$. Passing probes pass the ID criteria under
investigation, failing probes fail the ID criteria. The invariant mass of the
tag and probe pair is required to be near the \Z mass peak (\MassRange). The
electrons are required to have charges of opposite sign. In MC, the probe is
only required to be matched to a generator electron with $\Delta R < 0.2$.

The efficiencies are then calculated by fitting the \mee distributions using a
template constructed with a \Ztoee MC sample and an exponential background. The
three categories (passing probes, failing probes, and all probes) are then
simultaneously fit with this template and the number of fitted signal events is
used to derive an efficiency. A simple count of events is used for the MC
efficiency instead of a fit. The resulting scale factors are given in
\TABS~\ref{table:tight_scale_factor} and \ref{table:medium_scale_factor}.

% table:tight_scale_factor
\begin{table}[h]
    \centering
    \spacerows{1.5}
    \begin{center}
        \begin{tabular}{@{}l r r r r@{}}
            \toprule
            $|\eta|$                 & \GeVRange{20}{30}         & \GeVRange{30}{40}         & \GeVRange{40}{50}         & \GeVRange{50}{200} \\
            \midrule
            \numrange{0.0}{0.8}      & $0.960_{-0.003}^{+0.003}$ & $0.978_{-0.001}^{+0.001}$ & $0.981_{-0.001}^{+0.001}$ & $0.982_{-0.002}^{+0.002}$ \\
            \numrange{0.8}{1.4442}   & $0.936_{-0.004}^{+0.004}$ & $0.958_{-0.002}^{+0.002}$ & $0.969_{-0.001}^{+0.001}$ & $0.969_{-0.002}^{+0.002}$ \\
            \numrange{1.4442}{1.566} & $0.933_{-0.017}^{+0.015}$ & $0.907_{-0.008}^{+0.008}$ & $0.904_{-0.004}^{+0.004}$ & $0.926_{-0.011}^{+0.011}$ \\
            \numrange{1.566}{2.0}    & $0.879_{-0.007}^{+0.007}$ & $0.909_{-0.003}^{+0.003}$ & $0.942_{-0.002}^{+0.002}$ & $0.957_{-0.004}^{+0.004}$ \\
            \numrange{2.0}{2.5}      & $0.974_{-0.004}^{+0.004}$ & $0.987_{-0.004}^{+0.004}$ & $0.991_{-0.003}^{+0.003}$ & $0.999_{-0.005}^{+0.005}$ \\
            \bottomrule
        \end{tabular}
    \end{center}
    \caption{
        Scale factors for \EGTIGHT electron ID.
    }
    \label{table:tight_scale_factor}
\end{table}


% table:medium_scale_factor
\begin{table}[h]
    \centering
    \spacerows{1.5}
    \begin{center}
        \begin{tabular}{@{}l r r r r@{}}
            \toprule
            $|\eta|$                 & 20--30 \GeV               & 30--40 \GeV               & 40--50 \GeV               & 50--200 \GeV \\
            \midrule
            \numrange{0.0}{0.8}      & $0.986_{-0.001}^{+0.002}$ & $1.002_{-0.001}^{+0.001}$ & $1.005_{-0.001}^{+0.001}$ & $1.004_{-0.001}^{+0.001}$ \\
            \numrange{0.8}{1.4442}   & $0.959_{-0.003}^{+0.003}$ & $0.980_{-0.001}^{+0.001}$ & $0.988_{-0.001}^{+0.001}$ & $0.988_{-0.002}^{+0.002}$ \\
            \numrange{1.4442}{1.566} & $0.967_{-0.013}^{+0.007}$ & $0.950_{-0.007}^{+0.006}$ & $0.958_{-0.005}^{+0.005}$ & $0.966_{-0.009}^{+0.009}$ \\
            \numrange{1.566}{2.0}    & $0.941_{-0.005}^{+0.005}$ & $0.967_{-0.003}^{+0.003}$ & $0.992_{-0.002}^{+0.002}$ & $1.000_{-0.003}^{+0.003}$ \\
            \numrange{2.0}{2.5}      & $1.020_{-0.003}^{+0.003}$ & $1.021_{-0.003}^{+0.003}$ & $1.019_{-0.002}^{+0.002}$ & $1.022_{-0.004}^{+0.004}$ \\
            \bottomrule
        \end{tabular}
    \end{center}
    \caption{
        Scale factors for \EGMEDIUM electron ID.
    }
    \label{table:medium_scale_factor}
\end{table}



% Analysis and conclusion
\chapter{Event Selection}
\label{event_selection_chapter}

This chapter details the requirements used to select events for the analysis.
It also covers the data used and the Monte Carlo (MC) used. The final state we
are considering is \Ztoee.

\section{Data and Monte Carlo}

\subsection{Data}

\TODO{Figure of lumis \label{fig:2012_luminosity}}

The data used in this analysis were collected by the CMS detector in 2012 at a
center of mass energy of \rootseight. The LHC delivered 23 \fbinv of integrated
luminosity during the year as seen in \FIG~\ref{fig:2012_luminosity}. This
period was divided into four eras called 2012A, B, C, and D. During an era, the
LHC run parameters are kept roughly static to allow for consistent data taking
conditions. In between eras, maintenance and minor upgrades are performed on
the LHC in order to deliver higher luminosity. After accounting for detector
dead time and beam quality, \GoodLumiNumber of integrated luminosity are used
for physics analysis.

The data collected by CMS are split into smaller datasets based on the physics
objects contained within the events. This allows analyses to use only one or
two datasets, instead of requiring them to deal with the entirety of the CMS
data (which is petabyte scale, and hence too large for most institutes to store
locally). The HLT sorts events into the various datasets based on the triggers
that the event fired. In this manner, and event can end up in multiple datasets
if it fired multiple triggers. This analysis uses the \SingleElectron dataset
which was collected with the HLT trigger \SingleElectronTrigger. These datasets
were reconstructed---converted from raw detector response into physics
objects---in January, 2013, in order to make use of the most recent
calibrations derived from the entire 2012 run. A summary of the datasets used
are listed in \TAB~\ref{table:datasets}.

\begin{table}[h]
\centering
\begin{center}
    \begin{tabular}{ | l | c | c |}
    \hline
	Dataset Name                          & Run Range      & Luminosity       \\ \hline
	/SingleElectron/Run2012A-22Jan2013-v1 & 190456--193621 & $889.362 \pbinv$ \\ \hline
	/SingleElectron/Run2012B-22Jan2013-v1 & 193833--196531 & $4.429 \fbinv$   \\ \hline
	/SingleElectron/Run2012C-22Jan2013-v1 & 198022--203742 & $7.152 \fbinv$   \\ \hline
	/SingleElectron/Run2012D-22Jan2013-v1 & 203777--208686 & $7.318 \fbinv$   \\ \hline
    \end{tabular}
\end{center}
\caption{
    The datasets used in this analysis.
}
\label{table:datasets}
\end{table}

\chapter{Analysis}
\label{chapter:analysis}

\section{Uncertainties}
\label{sec:uncertainties}

\TODO{Intro?}

\subsection{Statistical Uncertainties}
\label{ssec:stat_uncertainty}

The statistical uncertainties due to the number of data events are propagated
through the unfolding method, as discussed in
\SEC~\ref{ssec:unfolding_statistical_uncertainties}. This uncertainty is
ranges from 0.26 to 1.21\% for both the absolute and normalized cross section
measurement. It is one of the dominant uncertainties for the normalized cross
section measurement.

\subsection{Statistical Uncertainties from the Monte Carlo Samples}
\label{ssec:mc_stat_uncertainty}

The \MADGRAPH signal MC sample has fewer events which pass our final selection
than there are events in the data. This sample is used to unfold the data and
so its statistical uncertainty affects the final measurement. The affect of the
statistical uncertainty on the bin migration matrix is propagated through to
the final result via the use of toy MC, detailed in
\SEC~\ref{ssec:unfolding_statistical_uncertainties}. The uncertainty found with
this method is 0.1 to 0.2\% for both the absolute and normalized cross section
measurements.

In addition to affecting the unfolding, the low number of events in the MC
affects the efficiency correction discussed in \SEC~\ref{ssec:eff_correction}
and shown in \FIG~\ref{fig:average_efficiencies}. These uncertainties vary from
0.3 to 1.3\% for both the absolute and normalized cross section measurements.

These two sources of uncertainty are measured separately. The finally
uncertainty due to the statistical uncertainties from the \MADGRAPH signal MC
sample is the sum in quadrature of both sources. This combined uncertainty is
the dominant uncertainty for the normalized cross section measurement, having a
slightly larger effect in each \phistar bin than the statistical uncertainty
due to the data.

\subsection{Luminosity Uncertainty}
\label{ssec:lumi_uncertainty}

The integrated luminosity is measured at CMS using the occupancy in the
pixel detector during minimum-bias events \cite{cms_lumi_2013}. This luminosity
measurement is calibrated by using van der Meer scans---a method to measure the
beam size in which the two beams are displaced and then ``swept'' across each
other as the offset is reduced \cite{vandermeer_1968}.

The integrated luminosity for the run period considered in this analysis is
known to \LumiUncertainty. This uncertainty is taken to be fully correlated
bin-by-bin in \phistar for the absolute cross section measurement, where it is
by far the dominant uncertainty. The luminosity cancels in the normalized
cross section measurement and so the uncertainty only affects the background
subtraction. This effect is negligible compared to the uncertainty already
present due to the background subtraction, which is discussed in
\SEC~\ref{ssec:background_subtraction_uncertainty}. The large uncertainty on
the luminosity is the primary motivation behind making a normalized cross
section measurement.

\subsection{Pileup Uncertainty}
\label{ssec:pileup_uncertainty}

As discussed in \SEC~\ref{ssec:monte_carlo}, the high beam intensity at the LHC
leads to multiple proton-proton interactions at each bunch crossing. This is
modeled in MC by overlaying multiple simulated minimum-bias events on top of
each simulated event. The distribution of pileup in MC is reweighted to match
the data distribution based on the calculated instantaneous luminosity and the
inelastic proton-proton cross section. The uncertainty due to this reweighting
process is calculated by varying the inelastic cross section by plus and minus
5\%, recalculating the data distribution of pileup, and reweighting to the MC
samples to match this new distribution. The full analysis is then performed
with these MC samples and the differences between the \phistar distributions is
taken as a systematic uncertainty. The pileup uncertainty for the absolute
cross section measurement ranges from 0.21 to 0.58\%, while the uncertainty for
the normalized cross section measurement is between $< 0.01$ and 0.64\%.

\subsection{Trigger, Reconstruction, and Identification Scale Factors Uncertainty}
\label{scale_factor_uncertainty}

Differences between the MC and data are corrected for using scale factors.
Three different sets of scale factors are used to reweight the MC samples:
trigger scale factors (discussed in \SEC~\ref{ssec:sf_trigger}), reconstruction
scale factors (discussed in \SEC~\ref{ssec:sf_reconstruction}), and
identification scale factors (discussed in \SEC~\ref{ssec:sf_id}).

In all three cases, the uncertainties on the scale factors are propagated
through to the final measurement using 500 toy MC variations. In this method,
every toy is constructed by drawing each scale factor from a Gaussian
probability distribution with its mean set to nominal value of the scale factor
and its width set to the quadrature sum of the uncertainties on the scale
factor. Each toy is then used to weight the MC samples used in this analysis,
and the full analysis is performed with that newly weighted sample. The
uncertainty on the final result due to one of the three types of scale factors
is taken to be defined by the central 68.2\% of results from the toys.

This procedure for propagating the uncertainty is performed independently for
each of the three types of scale factors. The total uncertainty due to the
scale factors is the sum in quadrature of the three results. For the absolute
cross section measurement this uncertainty is about 0.4\%, while for the
normalized cross section measurement it ranges from 0.02 to 0.35\%.

\subsection{\texorpdfstring{\pt}{PT} Scale Uncertainty}
\label{ssec:pt_scale_uncertainty}

One of the advantages of the \phistar variable is that it is not computed using
the momentum of the electrons and instead uses only the angles between them,
which are generally better measured. This makes \phistar less sensitive to any
potential problems with the \pt measurement of electrons.

However, the measurement of the \pt of the electrons is used to determine which
events are included in our sample. Therefore, a shift in the \pt scale of the
detector will either add or remove events that have electrons near the \pt
selection requirement boundaries. To determine the uncertainty due to the \pt
scale, we vary the \pt values of all of the electron up and down by 0.3\%,
which is a conservative estimate of the uncertainty on the \pt scale. The
largest difference in each \phistar bin between the nominal result and the
results with the modified \pt scale is taken as the uncertainty in that bin.
The uncertainty due to the \pt scale for the absolute cross section measurement
is 0.07 to 0.17\%, while the uncertainty for the normalized cross section
measure is $< 0.01$ to 0.10\%.

\subsection{Background Subtraction Uncertainty}
\label{ssec:background_subtraction_uncertainty}

The background subtraction, which is discussed in \SEC~\ref{sec:background},
deals with three separate categories of backgrounds. The uncertainty in
each category is determined with a different method, and these uncertainties
are added in quadrature to determine the total uncertainty due to the
background subtraction.

The first category consists of the various backgrounds with two independent
decay chains each of which can produce a lepton: $\ttbar$, $\W\W$,
$\DYtotautau$, $t\W$, and $\tbar\W$. The contributions from these backgrounds
are estimated by using an \emu control sample as discussed in
\SEC~\ref{ssec:emu_sample}. The uncertainty from the scale factors derived
using this method are propagated through to the final result using 500 toy MC
variations. For each variation, the scale factors are randomly drawn from a
Gaussian distribution with mean equal to the nominal value of that scale factor
and width equal to the uncertainty on the scale factor. Each toy is then used
to weight the background MC samples and the full which are then used to perform
the background subtraction. The full analysis is then run with the newly
background-subtracted data samples. The uncertainty due to the subtraction of
this category of background for each bin in \phistar is defined by the spread
of the central 68.2\% values obtained by the toys.

The second category consists of the backgrounds with a real \Z boson: $\Z\Z$
and $\W\Z$. For these samples, the uncertainty is calculated by taking a
correlated 20\% uncertainty on the theoretical cross section.

The third and final category consists of the \QCDjets and \wjets
backgrounds. The method of estimating this background is discussed in
\SEC~\ref{ssec:qcd_background}. Instead of taking the uncertainties from the
fit, which would not account for any systematics in the method used, a
conservative 100\% uncertainty is assigned to this category.

The uncertainty due to the background subtraction for the absolute cross
section measurement varies from 0.02 to 0.64\%, and from 0.03 to 0.59\% in the
normalized cross section measurement.

\subsection{PDF and Cross Section Uncertainties}
\label{ssec:pdf_uncertainties}

As discussed in \TODO{\SEC~\ref{} Link to theory section}, the kinematics of
the \Z boson depend on the internal composition and kinematics of the protons
as they collide. The reconstructed \phistar distribution is therefore dependent
on the PDFs used to generate the signal MC sample.

The uncertainty due to this choice of PDF is calculated following the
recommendation of \PDFforLHC working group for the \POWHEG MC signal sample.
\PDFWeightProducer is used to reweight the \POWHEG sample using the \CTten PDF
set. A total of \num{26} different pairs of weights are used by the tool to
fully account for the uncertainty inherent in the PDF set; these weights are
provided by the \CTten collaboration specifically for this purpose. Each pair
of weights consists of a variation of a PDF parameter, with one weight
corresponding to adjusting the parameter up and the other weight to down. Each of
these weights are used to reweight the \POWHEG sample, and the analysis is
performed with this newly weighted sample.
The uncertainty due to each weight is taken to be the difference with the
nominal \phistar distribution. Two uncertainties are calculated: the
one due to all of the upward parameter adjustments, and one to all the downward
parameter adjustments added in quadrature. The largest of these for each
\phistar bin is taken as the total uncertainty.

For the \MADGRAPH MC signal sample, which is an LO sample generated with a LO
PDF, the uncertainties can not be calculated in this manner. Instead, the PDF
includes an uncertainty on the cross section as calculated by \FEWZ which is
used to scale the sample. This uncertainty is propagated through the analysis
be reweighting the \MADGRAPH with this uncertainty both added and subtracted.
The difference in the final \phistar distribution is taken as the uncertainty
due to the \FEWZ cross section. This uncertainty is the dominant uncertainty
from the \MADGRAPH sample.

\subsection{Final State Radiation Uncertainties}
\label{ssec:fsr_uncertainties}

FSR, where an electron radiates a photon, is discussed in
\SEC~\ref{sec:electron_dressing}. These photons can affect the reconstruction
of the \Z, but this is taken into account during the unfolding as discussed in
\SEC~\ref{sec:unfolding}. Hence, uncertainties in the modeling of FSR affect
the unfolding and the final measurement.

The uncertainty is calculated using the \FSRWeightProducer, which augments the
\PYTHIA QED calculation with exact $\BigO{\alpha}$ and $\alpha(\pt^{2})$
couplings and reweights the MC sample as if it had been produced with these
calculations from the start. The effect of this reweighting on the final
\phistar distribution is $\le 0.34\%$ for the absolute cross section
measurement and $\le 0.03\%$ for the normalized cross section measurement.

%\subsection{Electron Angular Position Uncertainty}
%
%\TODO{What has Nicole seen in new tests?}
%
%Since \phistar depends on the angles between the two electrons, it is sensitive
%to mismeasurement of these angles. As discussed in
%\SEC~\ref{sec:electron_reconstruction}, the position of a reconstructed
%electron comes from the tracker, and therefore misalignment of the tracker
%would lead to a systematic uncertainty on the angle measurement. The magnitude
%of this systematic is estimated by using the position of the ECAL supercluster
%associated with the electron to calculate the electron's position instead of
%using the track. The new supercluster-only position is then used to calculate a
%new \phistarSC that does not depend on the alignment of the tracker.
%
%The position of the supercluster does not take into account the amount the
%bending of the electron due to the magnetic field. While $\eta$ is unchanged by
%the pretense of the field, $\phi$ is changed. A correction is applied to $\phi$
%to find the angle at the interaction point, $\phizero$, based on the angle of
%the supercluster, $\phisc$. This correction is given by
%\EQ~\ref{eq:b_field_correction} where $q$ is the charge of the electron, \pt is
%the transverse momentum of the electron, $B$ is the magnitude of the magnetic
%field, and \Reffective is the effective radius of ECAL as a function of $\eta$
%and $\theta$ as given in \EQ~\ref{eq:effective_radius}. The charge and momentum
%come from the electron matched to the supercluster, and although they are
%determined in combination with the tracker, they are far less susceptible to
%the small scale misalignments of the tracker that we are considering.
%
%\begin{equation}\label{eq:b_field_correction}
%    \sin \left( \phisc - \phizero \right)
%    =
%    - \Reffective \frac{q B}{2 \pt}
%\end{equation}
%
%\begin{equation}\label{eq:effective_radius}
%    \Reffective
%    =
%    \left\{
%        \begin{array}{ll}
%            1.29 \meters & \text{if } |\eta| < 1.4442 \\
%            3.14 \meters \times \tan \left(\coord{\theta}\right) & \text{otherwise}
%        \end{array}
%    \right.
%\end{equation}
%
%The resulting \phistarSC distribution is compatible within statistical
%uncertainties with the \phistar distribution and so no systematic uncertainty
%is assigned for the angular position.
%
%\TODO{Plot of \phistar vs \phistarSC?}

\subsection{Uncertainty from Four Vector Corrections}
\label{four_vector_uncertainty}

The \mee distribution in the \MADGRAPH signal MC sample does not precisely
match the distribution in data, as seen in \FIG~\ref{fig:z_mass}. This
discrepancy remains even after applying the various energy and momentum
corrections to the electrons discussed in \SEC~\ref{sec:corrections}. In order
to determine the effect this has on the final measurement, the \MADGRAPH signal
MC sample is reweighted to remove this difference. The ratio between the
nominal \phistar value and the value derived after this reweighting is shown in
\FIG~\ref{fig:z_mass_reweighted}. The circular points show the ratio of the
reconstructed \phistar distributions, while the square points show the ratio of
the generated \phistar distributions. The errors are binomial. Most of the
points are consistent with \num{1}, and so no systematic uncertainty is
assigned for this disagreement.

\begin{figure}[!htbp]
    \centering
    \includegraphics[width=\textwidth]{figures/ZMass_reweighed.pdf}
    \caption[
        The ratio of \phistar in \MADGRAPH before and after reweighting to
        remove the differnce in the \mee distribution between MC and data.
    ]{
        The ratio of \phistar in \MADGRAPH before and after reweighting to
        remove the difference in the \mee distribution between MC and data seen
        in \FIG~\ref{fig:z_mass}. The circular points are the ratio in the
        reconstructed quantity, while the square points are the ratio in the
        generated quantity. The uncertainties are binomial.
    }
    \label{fig:z_mass_reweighted}
\end{figure}

\TODO{Reweighting the \Z \pt plot?}

\section{Uncertainty Tables}

Tables summarizing the values of the various uncertainties as well as the total
uncertainty in each bin of \phistar for both the data and for the \MADGRAPH
signal MC sample are given below. The data tables,
\TABS~\ref{tab:sys_uncert_abs} and \ref{tab:sys_uncert_norm}, contain the
following columns:

\begin{description}[noitemsep]

    \item[\phistar Range:] \hfill \\
        The range of \phistar values included in the bin.

    \item[Total Uncertainty (Total):] \hfill \\
        The quadrature sum of the all of the uncertainties.

    \item[Statistical Uncertainty (Stat.):] \hfill \\
        The uncertainty due to the limited number of events in the data, as
        discussed in \SEC~\ref{ssec:stat_uncertainty}.

    \item[Total Systematic Uncertainty (Total Syst.):] \hfill \\
        The quadrature sum of all of the systematic uncertainties including,
        for the absolute distribution, the luminosity uncertainty of
        \LumiUncertainty.

    \item[Monte Carlo Statistical Uncertainty (MC Stat.):] \hfill \\
        The uncertainty due to the limited number of events in the MC samples,
        as discussed in \SEC~\ref{ssec:mc_stat_uncertainty}.

    \item[Pileup Uncertainty (Pileup):] \hfill \\
        The uncertainty due to the pileup reweighting, as discussed in
        \SEC~\ref{ssec:pileup_uncertainty}.

    \item[Scale Factor Uncertainty (SF):] \hfill \\
        The uncertainty due to the scale factors, as discussed in
        \SEC~\ref{scale_factor_uncertainty}.

    \item[\pt Scale Uncertainty (\pt Scale):] \hfill \\
        The uncertainty due to \pt scale, as discussed in
        \SEC~\ref{ssec:pt_scale_uncertainty}.

    \item[Background Subtraction Uncertainty (Bkg.):] \hfill \\
        The uncertainty due to background subtraction, as discussed in
        \SEC~\ref{ssec:background_subtraction_uncertainty}.

\end{description}

The MC tables, \TABS~\ref{tab:madgraph_uncert_abs}, \ref{tab:powheg_uncer_abs},
\ref{tab:madgraph_uncert_norm}, and \ref{tab:powheg_uncer_norm}, have the
following columns:

\begin{description}[noitemsep]

    \item[\phistar Range:] \hfill \\
        The range of \phistar values included in the bin.

    \item[Total Uncertainty (Total):] \hfill \\
        The quadrature sum of the all of the uncertainties.

    \item[Statistical Uncertainty (Stat.):] \hfill \\
        The uncertainty due to the limited number of events in the MC sample.

    \item[Parton Density Function (PDF):] \hfill \\
        The uncertainty due to choice of PDF used to generate the \POWHEG MC,
        as discussed in \SEC~\ref{ssec:pdf_uncertainties}.

    \item[Theoretical Cross Section Uncertainty (Cross Section):] \hfill \\
        The uncertainty in the theoretical cross section of the \MADGRAPH MC,
        as discussed in \SEC~\ref{ssec:pdf_uncertainties}.

    \item[Final State Radiation Uncertainty (FSR):] \hfill \\
        The uncertainty due to the modeling of FSR, as discussed in
        \SEC~\ref{ssec:fsr_uncertainties}.

\end{description}

% Absolute

% tab:sys_uncert_abs
\begin{table}
    \spacerows{1.05}
    \begin{center}
        \resizebox{\columnwidth}{!}{%
            \begin{tabular}{@{}l l l l l l l l l@{}}
                \toprule
                \phistar Range  &  Total  &  Stat.  &  Total Syst.  &  MC Stat.  &  Pileup  &  SF    &  \pt Scale  &  Bkg.  \\
                \midrule
                0.000--0.004    &  2.72   &  0.26   &  2.70         &  0.33      &  0.48    &  0.43  &  0.16       &  0.02  \\
                0.004--0.008    &  2.72   &  0.28   &  2.70         &  0.40      &  0.41    &  0.43  &  0.17       &  0.05  \\
                0.008--0.012    &  2.71   &  0.29   &  2.69         &  0.36      &  0.39    &  0.43  &  0.15       &  0.03  \\
                0.012--0.016    &  2.72   &  0.29   &  2.71         &  0.40      &  0.43    &  0.43  &  0.16       &  0.03  \\
                0.016--0.020    &  2.75   &  0.29   &  2.73         &  0.39      &  0.58    &  0.43  &  0.15       &  0.03  \\
                0.020--0.024    &  2.71   &  0.30   &  2.69         &  0.39      &  0.33    &  0.43  &  0.17       &  0.04  \\
                0.024--0.029    &  2.72   &  0.27   &  2.70         &  0.34      &  0.47    &  0.44  &  0.15       &  0.04  \\
                0.029--0.034    &  2.72   &  0.28   &  2.71         &  0.36      &  0.48    &  0.43  &  0.14       &  0.03  \\
                0.034--0.039    &  2.72   &  0.29   &  2.71         &  0.37      &  0.48    &  0.43  &  0.17       &  0.03  \\
                0.039--0.045    &  2.71   &  0.27   &  2.70         &  0.33      &  0.45    &  0.44  &  0.15       &  0.03  \\
                0.045--0.052    &  2.71   &  0.25   &  2.70         &  0.33      &  0.45    &  0.44  &  0.13       &  0.03  \\
                0.052--0.057    &  2.73   &  0.32   &  2.71         &  0.40      &  0.45    &  0.43  &  0.15       &  0.03  \\
                0.057--0.064    &  2.71   &  0.28   &  2.70         &  0.35      &  0.41    &  0.44  &  0.16       &  0.03  \\
                0.064--0.072    &  2.73   &  0.27   &  2.72         &  0.34      &  0.56    &  0.44  &  0.16       &  0.02  \\
                0.072--0.081    &  2.71   &  0.26   &  2.70         &  0.33      &  0.45    &  0.44  &  0.16       &  0.04  \\
                0.081--0.091    &  2.71   &  0.27   &  2.70         &  0.32      &  0.46    &  0.44  &  0.14       &  0.03  \\
                0.091--0.102    &  2.72   &  0.27   &  2.71         &  0.33      &  0.51    &  0.44  &  0.15       &  0.03  \\
                0.102--0.114    &  2.72   &  0.27   &  2.71         &  0.34      &  0.49    &  0.43  &  0.14       &  0.04  \\
                0.114--0.128    &  2.72   &  0.27   &  2.70         &  0.33      &  0.47    &  0.44  &  0.15       &  0.05  \\
                0.128--0.145    &  2.70   &  0.26   &  2.69         &  0.32      &  0.37    &  0.43  &  0.16       &  0.07  \\
                0.145--0.165    &  2.70   &  0.26   &  2.69         &  0.32      &  0.41    &  0.43  &  0.15       &  0.05  \\
                0.165--0.189    &  2.70   &  0.26   &  2.69         &  0.32      &  0.39    &  0.43  &  0.14       &  0.05  \\
                0.189--0.219    &  2.70   &  0.26   &  2.69         &  0.32      &  0.41    &  0.43  &  0.14       &  0.10  \\
                0.219--0.258    &  2.72   &  0.26   &  2.70         &  0.32      &  0.48    &  0.43  &  0.16       &  0.09  \\
                0.258--0.312    &  2.71   &  0.26   &  2.69         &  0.31      &  0.43    &  0.42  &  0.15       &  0.10  \\
                0.312--0.391    &  2.70   &  0.26   &  2.69         &  0.31      &  0.42    &  0.42  &  0.15       &  0.12  \\
                0.391--0.524    &  2.71   &  0.26   &  2.69         &  0.32      &  0.42    &  0.41  &  0.14       &  0.17  \\
                0.524--0.695    &  2.73   &  0.32   &  2.72         &  0.38      &  0.49    &  0.41  &  0.12       &  0.23  \\
                0.695--0.918    &  2.73   &  0.39   &  2.70         &  0.47      &  0.22    &  0.41  &  0.11       &  0.32  \\
                0.918--1.153    &  2.81   &  0.54   &  2.76         &  0.65      &  0.33    &  0.42  &  0.11       &  0.38  \\
                1.153--1.496    &  2.85   &  0.61   &  2.79         &  0.71      &  0.32    &  0.43  &  0.07       &  0.44  \\
                1.496--1.947    &  2.95   &  0.77   &  2.84         &  0.85      &  0.39    &  0.44  &  0.15       &  0.50  \\
                1.947--2.522    &  3.09   &  0.99   &  2.93         &  1.11      &  0.21    &  0.44  &  0.15       &  0.59  \\
                2.522--3.277    &  3.27   &  1.21   &  3.04         &  1.36      &  0.13    &  0.44  &  0.08       &  0.64  \\
                \bottomrule
            \end{tabular}
        }
    \end{center}
    \caption[
        The uncertainties for the absolute cross section measurement made with
        data unfolded with \MADGRAPH.
    ]{
        The uncertainties (in \%) for the absolute cross section measurement
        made with data unfolded with \MADGRAPH. The total value and the total
        systematic value includes the uncertainty of \LumiUncertainty due to
        the luminosity.
    }
    \label{tab:sys_uncert_abs}
\end{table}

% tab:madgraph_uncert_abs
\begin{table}
    \spacerows{1.05}
    \begin{center}
        \begin{tabular}{@{}l l l l l@{}}
            \toprule
            \phistar Range & Total & Stat. & Cross Section & FSR \\
            \midrule
            0.000--0.004 & 3.32 & 0.26 & 3.30 & 0.27  \\
            0.004--0.008 & 3.32 & 0.27 & 3.30 & 0.27  \\
            0.008--0.012 & 3.32 & 0.27 & 3.30 & 0.27  \\
            0.012--0.016 & 3.32 & 0.27 & 3.30 & 0.27  \\
            0.016--0.020 & 3.32 & 0.27 & 3.30 & 0.27  \\
            0.020--0.024 & 3.32 & 0.28 & 3.30 & 0.27  \\
            0.024--0.029 & 3.32 & 0.25 & 3.30 & 0.27  \\
            0.029--0.034 & 3.32 & 0.26 & 3.30 & 0.28  \\
            0.034--0.039 & 3.32 & 0.27 & 3.30 & 0.28  \\
            0.039--0.045 & 3.32 & 0.25 & 3.30 & 0.28  \\
            0.045--0.052 & 3.32 & 0.25 & 3.30 & 0.29  \\
            0.052--0.057 & 3.33 & 0.30 & 3.30 & 0.29  \\
            0.057--0.064 & 3.32 & 0.27 & 3.30 & 0.29  \\
            0.064--0.072 & 3.32 & 0.26 & 3.30 & 0.29  \\
            0.072--0.081 & 3.32 & 0.26 & 3.30 & 0.30  \\
            0.081--0.091 & 3.32 & 0.26 & 3.30 & 0.30  \\
            0.091--0.102 & 3.32 & 0.27 & 3.30 & 0.30  \\
            0.102--0.114 & 3.33 & 0.27 & 3.30 & 0.31  \\
            0.114--0.128 & 3.33 & 0.27 & 3.30 & 0.31  \\
            0.128--0.145 & 3.33 & 0.27 & 3.30 & 0.32  \\
            0.145--0.165 & 3.33 & 0.27 & 3.30 & 0.32  \\
            0.165--0.189 & 3.33 & 0.27 & 3.30 & 0.32  \\
            0.189--0.219 & 3.33 & 0.27 & 3.30 & 0.32  \\
            0.219--0.258 & 3.33 & 0.27 & 3.30 & 0.32  \\
            0.258--0.312 & 3.33 & 0.27 & 3.30 & 0.32  \\
            0.312--0.391 & 3.33 & 0.28 & 3.30 & 0.32  \\
            0.391--0.524 & 3.33 & 0.28 & 3.30 & 0.33  \\
            0.524--0.695 & 3.33 & 0.34 & 3.30 & 0.32  \\
            0.695--0.918 & 3.34 & 0.42 & 3.30 & 0.32  \\
            0.918--1.153 & 3.36 & 0.57 & 3.30 & 0.33  \\
            1.153--1.496 & 3.38 & 0.65 & 3.30 & 0.33  \\
            1.496--1.947 & 3.41 & 0.80 & 3.30 & 0.33  \\
            1.947--2.522 & 3.47 & 1.02 & 3.30 & 0.34  \\
            2.522--3.277 & 3.55 & 1.26 & 3.30 & 0.34  \\
            \bottomrule
        \end{tabular}
    \end{center}
    \caption[
        The uncertainties for the absolute cross section from the \MADGRAPH MC
        sample.
    ]{
        The uncertainties (in \%) for the absolute cross section from the
        \MADGRAPH MC sample.
    }
    \label{tab:madgraph_uncert_abs}
\end{table}

% tab:powheg_uncert_abs
\begin{table}
    \spacerows{1.05}
    \begin{center}
        \begin{tabular}{@{}l l l l l@{}}
            \toprule
            \phistar Range & Total & Stat. & PDF & FSR \\
            \midrule
            0.000--0.004 & 2.72 & 0.62 & 2.63 & 0.27  \\
            0.004--0.008 & 2.71 & 0.62 & 2.63 & 0.27  \\
            0.008--0.012 & 2.68 & 0.63 & 2.59 & 0.27  \\
            0.012--0.016 & 2.75 & 0.63 & 2.66 & 0.28  \\
            0.016--0.020 & 2.71 & 0.64 & 2.62 & 0.28  \\
            0.020--0.024 & 2.74 & 0.64 & 2.65 & 0.28  \\
            0.024--0.029 & 2.73 & 0.58 & 2.65 & 0.27  \\
            0.029--0.034 & 2.70 & 0.59 & 2.62 & 0.27  \\
            0.034--0.039 & 2.72 & 0.61 & 2.64 & 0.28  \\
            0.039--0.045 & 2.69 & 0.57 & 2.61 & 0.28  \\
            0.045--0.052 & 2.70 & 0.54 & 2.63 & 0.28  \\
            0.052--0.057 & 2.75 & 0.67 & 2.66 & 0.28  \\
            0.057--0.064 & 2.74 & 0.59 & 2.66 & 0.29  \\
            0.064--0.072 & 2.75 & 0.57 & 2.67 & 0.29  \\
            0.072--0.081 & 2.71 & 0.57 & 2.63 & 0.30  \\
            0.081--0.091 & 2.70 & 0.57 & 2.62 & 0.30  \\
            0.091--0.102 & 2.75 & 0.58 & 2.67 & 0.30  \\
            0.102--0.114 & 2.72 & 0.60 & 2.63 & 0.31  \\
            0.114--0.128 & 2.72 & 0.60 & 2.64 & 0.32  \\
            0.128--0.145 & 2.71 & 0.59 & 2.63 & 0.31  \\
            0.145--0.165 & 2.71 & 0.60 & 2.62 & 0.31  \\
            0.165--0.189 & 2.73 & 0.61 & 2.65 & 0.31  \\
            0.189--0.219 & 2.73 & 0.61 & 2.64 & 0.33  \\
            0.219--0.258 & 2.67 & 0.61 & 2.58 & 0.32  \\
            0.258--0.312 & 2.66 & 0.61 & 2.57 & 0.32  \\
            0.312--0.391 & 2.61 & 0.62 & 2.52 & 0.33  \\
            0.391--0.524 & 2.57 & 0.63 & 2.47 & 0.32  \\
            0.524--0.695 & 2.65 & 0.78 & 2.51 & 0.32  \\
            0.695--0.918 & 2.65 & 0.97 & 2.44 & 0.33  \\
            0.918--1.153 & 2.79 & 1.34 & 2.42 & 0.35  \\
            1.153--1.496 & 2.91 & 1.57 & 2.43 & 0.33  \\
            1.496--1.947 & 3.20 & 2.00 & 2.47 & 0.33  \\
            1.947--2.522 & 3.62 & 2.55 & 2.55 & 0.35  \\
            2.522--3.277 & 4.03 & 3.04 & 2.62 & 0.33  \\
            \bottomrule
        \end{tabular}
    \end{center}
    \caption{
        Fractional errors (in \%) for the absolute cross section from the
        \POWHEG MC sample.
    }
    \label{tab:powheg_uncert_abs}
\end{table}



% Normalized

% tab:sys_uncert_norm
\begin{table}
    \spacerows{1.05}
    \begin{center}
        \begin{tabular}{@{}l l l l l l l l l@{}}
            \toprule
            \phistar Range  &  Total  &  Stat.  &  Total Syst.  &  MC Stat.  &  Pileup  &  SF    &  \pt Scale  &  Bkg.  \\
            \midrule
            0.000--0.004    &  0.43   &  0.26   &  0.34         &  0.33      &  0.04    &  0.07  &  0.01       &  0.05  \\
            0.004--0.008    &  0.50   &  0.28   &  0.41         &  0.40      &  0.03    &  0.07  &  0.02       &  0.05  \\
            0.008--0.012    &  0.47   &  0.29   &  0.37         &  0.36      &  0.02    &  0.07  &  0.00       &  0.04  \\
            0.012--0.016    &  0.50   &  0.29   &  0.41         &  0.40      &  0.02    &  0.07  &  0.01       &  0.04  \\
            0.016--0.020    &  0.51   &  0.29   &  0.42         &  0.39      &  0.14    &  0.07  &  0.00       &  0.04  \\
            0.020--0.024    &  0.51   &  0.30   &  0.41         &  0.39      &  0.10    &  0.07  &  0.02       &  0.04  \\
            0.024--0.029    &  0.44   &  0.27   &  0.35         &  0.33      &  0.03    &  0.07  &  0.01       &  0.04  \\
            0.029--0.034    &  0.46   &  0.28   &  0.37         &  0.36      &  0.04    &  0.07  &  0.01       &  0.04  \\
            0.034--0.039    &  0.48   &  0.28   &  0.38         &  0.37      &  0.04    &  0.06  &  0.02       &  0.04  \\
            0.039--0.045    &  0.43   &  0.27   &  0.34         &  0.33      &  0.01    &  0.06  &  0.00       &  0.05  \\
            0.045--0.052    &  0.42   &  0.25   &  0.34         &  0.33      &  0.01    &  0.06  &  0.00       &  0.04  \\
            0.052--0.057    &  0.52   &  0.32   &  0.40         &  0.40      &  0.01    &  0.05  &  0.01       &  0.05  \\
            0.057--0.064    &  0.45   &  0.28   &  0.35         &  0.35      &  0.02    &  0.05  &  0.01       &  0.04  \\
            0.064--0.072    &  0.45   &  0.27   &  0.36         &  0.33      &  0.11    &  0.05  &  0.01       &  0.04  \\
            0.072--0.081    &  0.43   &  0.26   &  0.34         &  0.33      &  0.01    &  0.04  &  0.02       &  0.04  \\
            0.081--0.091    &  0.42   &  0.26   &  0.33         &  0.32      &  0.02    &  0.03  &  0.00       &  0.04  \\
            0.091--0.102    &  0.43   &  0.27   &  0.34         &  0.33      &  0.07    &  0.03  &  0.00       &  0.04  \\
            0.102--0.114    &  0.44   &  0.27   &  0.35         &  0.34      &  0.05    &  0.02  &  0.00       &  0.04  \\
            0.114--0.128    &  0.43   &  0.27   &  0.33         &  0.33      &  0.03    &  0.03  &  0.01       &  0.03  \\
            0.128--0.145    &  0.43   &  0.26   &  0.34         &  0.32      &  0.10    &  0.03  &  0.01       &  0.03  \\
            0.145--0.165    &  0.42   &  0.26   &  0.33         &  0.32      &  0.04    &  0.04  &  0.00       &  0.03  \\
            0.165--0.189    &  0.42   &  0.26   &  0.33         &  0.32      &  0.05    &  0.05  &  0.02       &  0.03  \\
            0.189--0.219    &  0.43   &  0.26   &  0.33         &  0.32      &  0.04    &  0.06  &  0.01       &  0.06  \\
            0.219--0.258    &  0.42   &  0.26   &  0.33         &  0.32      &  0.04    &  0.08  &  0.01       &  0.04  \\
            0.258--0.312    &  0.42   &  0.26   &  0.33         &  0.31      &  0.03    &  0.10  &  0.01       &  0.05  \\
            0.312--0.391    &  0.43   &  0.26   &  0.35         &  0.31      &  0.04    &  0.13  &  0.00       &  0.07  \\
            0.391--0.524    &  0.45   &  0.26   &  0.37         &  0.31      &  0.00    &  0.15  &  0.00       &  0.12  \\
            0.524--0.695    &  0.56   &  0.32   &  0.46         &  0.38      &  0.04    &  0.19  &  0.02       &  0.18  \\
            0.695--0.918    &  0.73   &  0.39   &  0.62         &  0.47      &  0.20    &  0.23  &  0.01       &  0.27  \\
            0.918--1.153    &  0.95   &  0.53   &  0.79         &  0.65      &  0.10    &  0.27  &  0.05       &  0.33  \\
            1.153--1.496    &  1.07   &  0.61   &  0.87         &  0.71      &  0.08    &  0.31  &  0.06       &  0.39  \\
            1.496--1.947    &  1.27   &  0.77   &  1.02         &  0.85      &  0.00    &  0.33  &  0.05       &  0.45  \\
            1.947--2.522    &  1.73   &  0.98   &  1.43         &  1.10      &  0.64    &  0.35  &  0.00       &  0.54  \\
            2.522--3.277    &  2.02   &  1.21   &  1.62         &  1.36      &  0.56    &  0.34  &  0.10       &  0.59  \\
            \bottomrule
        \end{tabular}
    \end{center}
    \caption[
        Fractional errors for the normalized cross section measurement
        made with data unfolded with \MADGRAPH.
    ]{
        Fractional errors (in \%) for the normalized cross section measurement
        made with data unfolded with \MADGRAPH.
    }
    \label{tab:sys_uncert_norm}
\end{table}

% tab:madgraph_uncert_norm
\begin{table}
    \spacerows{1.05}
    \begin{center}
        \begin{tabular}{@{}l l l l@{}}
            \toprule
            \phistar Range & Total & Stat. & FSR \\
            \midrule
            0.000--0.004 & 0.27 & 0.26 & 0.03  \\
            0.004--0.008 & 0.27 & 0.27 & 0.03  \\
            0.008--0.012 & 0.27 & 0.27 & 0.03  \\
            0.012--0.016 & 0.27 & 0.27 & 0.03  \\
            0.016--0.020 & 0.27 & 0.27 & 0.03  \\
            0.020--0.024 & 0.28 & 0.28 & 0.02  \\
            0.024--0.029 & 0.26 & 0.25 & 0.03  \\
            0.029--0.034 & 0.26 & 0.26 & 0.02  \\
            0.034--0.039 & 0.27 & 0.27 & 0.02  \\
            0.039--0.045 & 0.25 & 0.25 & 0.02  \\
            0.045--0.052 & 0.25 & 0.25 & 0.01  \\
            0.052--0.057 & 0.30 & 0.30 & 0.01  \\
            0.057--0.064 & 0.27 & 0.27 & 0.01  \\
            0.064--0.072 & 0.26 & 0.26 & 0.00  \\
            0.072--0.081 & 0.26 & 0.26 & 0.00  \\
            0.081--0.091 & 0.26 & 0.26 & 0.01  \\
            0.091--0.102 & 0.27 & 0.27 & 0.01  \\
            0.102--0.114 & 0.27 & 0.27 & 0.01  \\
            0.114--0.128 & 0.27 & 0.27 & 0.02  \\
            0.128--0.145 & 0.27 & 0.27 & 0.02  \\
            0.145--0.165 & 0.27 & 0.27 & 0.02  \\
            0.165--0.189 & 0.27 & 0.27 & 0.02  \\
            0.189--0.219 & 0.27 & 0.27 & 0.02  \\
            0.219--0.258 & 0.28 & 0.27 & 0.03  \\
            0.258--0.312 & 0.28 & 0.27 & 0.03  \\
            0.312--0.391 & 0.28 & 0.28 & 0.03  \\
            0.391--0.524 & 0.28 & 0.28 & 0.03  \\
            0.524--0.695 & 0.34 & 0.34 & 0.03  \\
            0.695--0.918 & 0.42 & 0.42 & 0.03  \\
            0.918--1.153 & 0.57 & 0.57 & 0.03  \\
            1.153--1.496 & 0.65 & 0.65 & 0.03  \\
            1.496--1.947 & 0.80 & 0.80 & 0.03  \\
            1.947--2.522 & 1.02 & 1.02 & 0.04  \\
            2.522--3.277 & 1.26 & 1.26 & 0.04  \\
            \bottomrule
        \end{tabular}
    \end{center}
    \caption[
        The uncertainties for the normalized cross section from the \MADGRAPH
        MC sample.
    ]{
        The uncertainties (in \%) for the normalized cross section from the
        \MADGRAPH MC sample.
    }
    \label{tab:madgraph_uncert_norm}
\end{table}

% tab:powheg_uncert_norm
\begin{table}
    \spacerows{1.05}
    \begin{center}
        \begin{tabular}{@{}l l l l l@{}}
            \toprule
            \phistar Range & Total & Stat. & PDF & FSR \\
            \midrule
            0.000-0.004 & 0.63 & 0.62 & 0.13 & 0.02  \\
            0.004-0.008 & 0.63 & 0.62 & 0.13 & 0.02  \\
            0.008-0.012 & 0.65 & 0.63 & 0.15 & 0.02  \\
            0.012-0.016 & 0.65 & 0.63 & 0.15 & 0.02  \\
            0.016-0.020 & 0.65 & 0.64 & 0.13 & 0.02  \\
            0.020-0.024 & 0.65 & 0.64 & 0.13 & 0.02  \\
            0.024-0.029 & 0.60 & 0.58 & 0.12 & 0.03  \\
            0.029-0.034 & 0.60 & 0.59 & 0.09 & 0.02  \\
            0.034-0.039 & 0.62 & 0.61 & 0.10 & 0.01  \\
            0.039-0.045 & 0.58 & 0.57 & 0.09 & 0.02  \\
            0.045-0.052 & 0.55 & 0.54 & 0.08 & 0.02  \\
            0.052-0.057 & 0.68 & 0.67 & 0.11 & 0.02  \\
            0.057-0.064 & 0.60 & 0.59 & 0.11 & 0.01  \\
            0.064-0.072 & 0.59 & 0.57 & 0.14 & 0.01  \\
            0.072-0.081 & 0.57 & 0.57 & 0.06 & 0.00  \\
            0.081-0.091 & 0.57 & 0.57 & 0.04 & 0.00  \\
            0.091-0.102 & 0.61 & 0.58 & 0.18 & 0.00  \\
            0.102-0.114 & 0.60 & 0.60 & 0.09 & 0.01  \\
            0.114-0.128 & 0.61 & 0.60 & 0.12 & 0.02  \\
            0.128-0.145 & 0.60 & 0.59 & 0.11 & 0.02  \\
            0.145-0.165 & 0.62 & 0.60 & 0.16 & 0.02  \\
            0.165-0.189 & 0.64 & 0.61 & 0.19 & 0.01  \\
            0.189-0.219 & 0.65 & 0.61 & 0.23 & 0.03  \\
            0.219-0.258 & 0.63 & 0.61 & 0.16 & 0.02  \\
            0.258-0.312 & 0.63 & 0.61 & 0.15 & 0.03  \\
            0.312-0.391 & 0.65 & 0.62 & 0.20 & 0.03  \\
            0.391-0.524 & 0.67 & 0.63 & 0.24 & 0.03  \\
            0.524-0.695 & 0.81 & 0.78 & 0.23 & 0.02  \\
            0.695-0.918 & 1.05 & 0.97 & 0.41 & 0.03  \\
            0.918-1.153 & 1.47 & 1.34 & 0.60 & 0.05  \\
            1.153-1.496 & 1.74 & 1.57 & 0.76 & 0.03  \\
            1.496-1.947 & 2.15 & 2.00 & 0.78 & 0.04  \\
            1.947-2.522 & 2.76 & 2.55 & 1.07 & 0.06  \\
            2.522-3.277 & 3.26 & 3.04 & 1.17 & 0.03  \\
            \bottomrule
        \end{tabular}
    \end{center}
    \caption{
        Total errors (in \%) for the normalized cross section from the
        \POWHEG MC sample.
    }
    \label{tab:powheg_uncert_norm}
\end{table}



% Bibliography
\bibliography{thesis}

% Appendices
\appendix
\chapter{Other Measurements Using \Dressed Electrons}
\label{app:dressed_measurements}

\section{Uncertainty Figures}

The values of the various uncertainty in each \phistar bin are presented in the
figures that follow. \Cref{fig:sys_uncert_norm} shows the fractional
uncertainty in the normalized \phistar cross section in data unfolded with
\MADGRAPH while \cref{fig:sys_uncert_norm_powheg} shows the fractional
uncertainty in the data unfolded with \POWHEG. The uncertainty on the
normalized \phistar cross section in \MADGRAPH and \POWHEG are shown in
\cref{fig:madgraph_uncert_norm,fig:powheg_uncert_norm},
respectively. All of these values are presented in tables in
\cref{app:uncertainty_tables}.

% Absolute

% fig:sys_uncert_abs
\begin{figure}[!p]
    \centering
    \includegraphics[width=\textwidth]{figures/data_uncertainty_absolute.pdf}
    \caption[
        Fractional errors for the absolute cross section measurement
        made with data unfolded with \MADGRAPH.
    ]{
        Fractional errors (in \%) for the absolute cross section measurement
        made with data unfolded with \MADGRAPH. The total value is the sum in
        quadrature of all the other values. These uncertainties are also
        presented in tabular form in \cref{tab:sys_uncert_abs}.
    }
    \label{fig:sys_uncert_abs}
\end{figure}


% fig:sys_uncert_abs_powheg
\begin{figure}[!p]
    \centering
    \includegraphics[width=\textwidth]{figures/data_uncertainty_absolute_powheg_unfolded.pdf}
    \caption[
        Fractional errors (in \%) for the absolute cross section measurement.
    ]{
        Fractional errors (in \%) for the absolute cross section measurement in
        different \phistar bins due to various sources. The total value is the
        sum in quadrature of all the other values. The unfolding of the data
        was done with \POWHEG. These uncertainties are also presented in
        tabular form in \TAB~\ref{tab:sys_uncert_abs_powheg}.
    }
    \label{fig:sys_uncert_abs_powheg}
\end{figure}


% fig:madgraph_uncert_abs
\begin{figure}[!p]
    \centering
    \includegraphics[width=\textwidth]{figures/madgraph_uncertainty_absolute.pdf}
    \caption[
        Fractional errors for the absolute cross section from the
        \MADGRAPH MC sample.
    ]{
        Fractional errors (in \%) for the absolute cross section from the
        \MADGRAPH MC sample. These uncertainties are also presented in tabular
        form in \cref{tab:madgraph_uncert_abs}.
    }
    \label{fig:madgraph_uncert_abs}
\end{figure}


% fig:powheg_uncert_abs
\begin{figure}[!p]
    \centering
    \includegraphics[width=\textwidth]{figures/powheg_uncertainty_absolute.pdf}
    \caption[
        The uncertainties for the absolute cross section from the \POWHEG MC
        sample.
    ]{
        The uncertainties (in \%) for the absolute cross section from the
        \POWHEG MC sample. These uncertainties are also presented in tabular
        form in \cref{tab:powheg_uncert_abs}.
    }
    \label{fig:powheg_uncert_abs}
\end{figure}



\section{Absolute Differential Cross Section}
\label{sec:results_abs}

The absolute differential cross section measurement using data unfolded with
\MADGRAPH is shown in \cref{fig:results_abs} and given in tabular form in
\cref{tab:results_abs}. The lower plot in \cref{fig:results_abs} is
shown in more detail in \cref{fig:results_ratio_abs}. As previously
discussed in this chapter, the primary uncertainty on the data distribution is
from the integrated luminosity. The primary uncertainty for the
\MADGRAPH sample is the \FEWZ calculated overall cross section used to scale
the distribution, while the primary uncertainty for the \POWHEG sample is the
uncertainty calculated by varying the \CTten PDF weights.

% fig:results_abs and fig:results_ratio_abs
\begin{figure}[!htbp]
    \centering
    \includegraphics[width=\textwidth]{figures/ZShape_elec_Abs_Dressed.pdf}
    \caption[
        The absolute differential cross section with respects to \phistar for
        \Ztoee events in our fiducial region from data and \MADGRAPH and
        \POWHEG.
    ]{
        The absolute differential cross section with respects to \phistar for
        \Ztoee events in our fiducial region from data and \MADGRAPH and
        \POWHEG. A close up of the lower plot is shown in
        \FIG~\ref{fig:result_ratio_abs}.
    }
    \label{fig:result_abs}
\end{figure}

\begin{figure}[!htbp]
    \centering
    \includegraphics[width=\textwidth]{figures/ZShape_Ratioelec_Abs_Dressed.pdf}
    \caption[
        Close up of the ratio plot from \FIG~\ref{fig:result_abs} for the
        absolute cross section measurement.
    ]{
        Close up of the ratio plot from \FIG~\ref{fig:result_abs} for the
        absolute cross section measurement. The error band indicates the
        uncertainty in the data, while the square points show the ratio of
        \MADGRAPH over data, and the triangle points show the ratio of \POWHEG
        over data.
    }
    \label{fig:result_ratio_abs}
\end{figure}

% tab:results_abs
\begin{table}
    \spacerows{1.05}
    \begin{center}
        \begin{tabular}{@{}l r r@{}}
            \toprule
            \phistar range & Data $(\pb)$ & \MADGRAPH $(\pb)$ \\
            \midrule
            0.000-0.004  &  $4296  \pm  115$   &  $4155  \pm  138$   \\
            0.004-0.008  &  $4303  \pm  115$   &  $4083  \pm  136$   \\
            0.008-0.012  &  $4192  \pm  112$   &  $4037  \pm  134$   \\
            0.012-0.016  &  $4135  \pm  110$   &  $3969  \pm  132$   \\
            0.016-0.020  &  $3950  \pm  107$   &  $3879  \pm  129$   \\
            0.020-0.024  &  $3815  \pm  101$   &  $3734  \pm  124$   \\
            0.024-0.029  &  $3633  \pm  97$    &  $3586  \pm  119$   \\
            0.029-0.034  &  $3402  \pm  91$    &  $3396  \pm  113$   \\
            0.034-0.039  &  $3180  \pm  85$    &  $3192  \pm  106$   \\
            0.039-0.045  &  $2972  \pm  79$    &  $2987  \pm  99$    \\
            0.045-0.052  &  $2726  \pm  73$    &  $2750  \pm  91$    \\
            0.052-0.057  &  $2519  \pm  67$    &  $2514  \pm  84$    \\
            0.057-0.064  &  $2335  \pm  62$    &  $2335  \pm  78$    \\
            0.064-0.072  &  $2100  \pm  56$    &  $2104  \pm  70$    \\
            0.072-0.081  &  $1904  \pm  51$    &  $1883  \pm  63$    \\
            0.081-0.091  &  $1694  \pm  45$    &  $1678  \pm  56$    \\
            0.091-0.102  &  $1499  \pm  40$    &  $1471  \pm  49$    \\
            0.102-0.114  &  $1319  \pm  35$    &  $1288  \pm  43$    \\
            0.114-0.128  &  $1152  \pm  31$    &  $1118  \pm  37$    \\
            0.128-0.145  &  $985   \pm  26$    &  $958   \pm  32$    \\
            0.145-0.165  &  $826   \pm  22$    &  $797   \pm  27$    \\
            0.165-0.189  &  $678   \pm  18$    &  $648   \pm  22$    \\
            0.189-0.219  &  $537   \pm  14$    &  $517   \pm  17$    \\
            0.219-0.258  &  $415   \pm  11$    &  $394   \pm  13$    \\
            0.258-0.312  &  $304   \pm  8$     &  $287   \pm  10$    \\
            0.312-0.391  &  $204   \pm  5$     &  $192   \pm  6$     \\
            0.391-0.524  &  $120   \pm  3$     &  $112   \pm  4$     \\
            0.524-0.695  &  $62    \pm  2$     &  $59    \pm  2$     \\
            0.695-0.918  &  $31.6  \pm  0.9$   &  $29.3  \pm  1.0$   \\
            0.918-1.153  &  $16.1  \pm  0.4$   &  $15.2  \pm  0.5$   \\
            1.153-1.496  &  $8.5   \pm  0.2$   &  $8.0   \pm  0.3$   \\
            1.496-1.947  &  $4.0   \pm  0.1$   &  $4.0   \pm  0.1$   \\
            1.947-2.522  &  $1.96  \pm  0.06$  &  $1.93  \pm  0.07$  \\
            2.522-3.277  &  $1.00  \pm  0.03$  &  $0.97  \pm  0.03$  \\
            \bottomrule
        \end{tabular}
    \end{center}
    \caption[
        Differential cross-section in \pb with respect to \phistar of \Ztoee.
    ]{
        Differential cross-section in \pb with respect to \phistar of \Ztoee in our
        fiducial region in data and generated by \MADGRAPH. These results are shown
        graphically in figure~\ref{fig:result_abs}.
    }
    \label{tab:results_abs}
\end{table}


The absolute differential cross section measurement using data unfolded with
\POWHEG is shown in \cref{fig:results_abs_powheg} and given in tabular form
in \cref{tab:results_abs_powheg}. The lower plot in
\cref{fig:results_abs_powheg} is shown in more detail in
\cref{fig:results_ratio_abs_powheg}. The primary uncertainty on the data
distribution is still from the integrated luminosity, although the MC
statistical uncertainty is larger in the highest \phistar bins.

% fig:results_abs_powheg and fig:results_ratio_abs_powheg
\begin{figure}[!p]
    \centering
    \includegraphics[width=\textwidth]{figures/ZShape_elec_PH_Abs_Dressed.pdf}
    \caption[
        The absolute differential cross section with respects to \phistar for
        \Ztoee events in our fiducial region from data unfolded with \POWHEG,
        and the same distributions in \MADGRAPH and \POWHEG.
    ]{
        The absolute differential cross section with respects to \phistar for
        \Ztoee events in our fiducial region from data unfolded with \POWHEG,
        and the same distributions in \MADGRAPH and \POWHEG. A close up of the
        lower plot is shown in \cref{fig:results_ratio_abs_powheg}.
    }
    \label{fig:results_abs_powheg}
\end{figure}

\begin{figure}[!p]
    \centering
    \includegraphics[width=\textwidth]{figures/ZShape_Ratioelec_PH_Abs_Dressed.pdf}
    \caption[
        Close up of the ratio plot from \cref{fig:results_abs_powheg} for
        the absolute cross section measurement unfolded with \POWHEG.
    ]{
        Close up of the ratio plot from \cref{fig:results_abs_powheg} for
        the absolute cross section measurement unfolded with \POWHEG. The error
        band indicates the uncertainty in the data, while the square points
        show the ratio of \MADGRAPH over data, and the triangle points show the
        ratio of \POWHEG over data.
    }
    \label{fig:results_ratio_abs_powheg}
\end{figure}

% tab:results_abs_powheg
\begin{table}
    \spacerows{1.05}
    \begin{center}
        \begin{tabular}{@{}l r r r@{}}
            \toprule
            \phistar range & Data $(\pb)$ & \MADGRAPH $(\pb)$ & \POWHEG $(\pm)$ \\
            \midrule
            0.000--0.004  &  $4381  \pm  122$   &  $4155  \pm  138$   &  $3871  \pm  105$   \\
            0.004--0.008  &  $4302  \pm  122$   &  $4083  \pm  136$   &  $3881  \pm  105$   \\
            0.008--0.012  &  $4191  \pm  121$   &  $4037  \pm  134$   &  $3806  \pm  102$   \\
            0.012--0.016  &  $4096  \pm  118$   &  $3969  \pm  132$   &  $3746  \pm  103$   \\
            0.016--0.020  &  $4016  \pm  113$   &  $3879  \pm  129$   &  $3660  \pm  99$    \\
            0.020--0.024  &  $3751  \pm  108$   &  $3734  \pm  124$   &  $3642  \pm  100$   \\
            0.024--0.029  &  $3654  \pm  102$   &  $3586  \pm  119$   &  $3518  \pm  96$    \\
            0.029--0.034  &  $3412  \pm  95$    &  $3396  \pm  113$   &  $3393  \pm  92$    \\
            0.034--0.039  &  $3213  \pm  91$    &  $3192  \pm  106$   &  $3229  \pm  88$    \\
            0.039--0.045  &  $2985  \pm  83$    &  $2987  \pm  99$    &  $3071  \pm  83$    \\
            0.045--0.052  &  $2734  \pm  77$    &  $2750  \pm  91$    &  $2878  \pm  78$    \\
            0.052--0.057  &  $2509  \pm  71$    &  $2514  \pm  84$    &  $2662  \pm  73$    \\
            0.057--0.064  &  $2295  \pm  65$    &  $2335  \pm  78$    &  $2479  \pm  68$    \\
            0.064--0.072  &  $2107  \pm  59$    &  $2104  \pm  70$    &  $2257  \pm  62$    \\
            0.072--0.081  &  $1891  \pm  53$    &  $1883  \pm  63$    &  $2057  \pm  56$    \\
            0.081--0.091  &  $1676  \pm  47$    &  $1678  \pm  56$    &  $1831  \pm  49$    \\
            0.091--0.102  &  $1488  \pm  41$    &  $1471  \pm  49$    &  $1589  \pm  44$    \\
            0.102--0.114  &  $1327  \pm  37$    &  $1288  \pm  43$    &  $1392  \pm  38$    \\
            0.114--0.128  &  $1135  \pm  32$    &  $1118  \pm  37$    &  $1198  \pm  33$    \\
            0.128--0.145  &  $985   \pm  28$    &  $958   \pm  32$    &  $1008  \pm  27$    \\
            0.145--0.165  &  $824   \pm  23$    &  $797   \pm  27$    &  $824   \pm  22$    \\
            0.165--0.189  &  $671   \pm  19$    &  $648   \pm  22$    &  $676   \pm  18$    \\
            0.189--0.219  &  $538   \pm  15$    &  $517   \pm  17$    &  $536   \pm  15$    \\
            0.219--0.258  &  $412   \pm  11$    &  $394   \pm  13$    &  $412   \pm  11$    \\
            0.258--0.312  &  $298   \pm  8$     &  $287   \pm  10$    &  $298   \pm  8$     \\
            0.312--0.391  &  $202   \pm  6$     &  $192   \pm  6$     &  $197   \pm  5$     \\
            0.391--0.524  &  $116   \pm  3$     &  $112   \pm  4$     &  $114   \pm  3$     \\
            0.524--0.695  &  $61    \pm  2$     &  $59    \pm  2$     &  $58    \pm  2$     \\
            0.695--0.918  &  $31.3  \pm  0.9$   &  $29.3  \pm  1.0$   &  $28.3  \pm  0.7$   \\
            0.918--1.153  &  $16.4  \pm  0.5$   &  $15.2  \pm  0.5$   &  $14.1  \pm  0.4$   \\
            1.153--1.496  &  $8.4   \pm  0.3$   &  $8.0   \pm  0.3$   &  $7.1   \pm  0.2$   \\
            1.496--1.947  &  $3.9   \pm  0.1$   &  $4.0   \pm  0.1$   &  $3.3   \pm  0.1$   \\
            1.947--2.522  &  $1.95  \pm  0.08$  &  $1.93  \pm  0.07$  &  $1.60  \pm  0.06$  \\
            2.522--3.277  &  $0.97  \pm  0.04$  &  $0.97  \pm  0.03$  &  $0.85  \pm  0.03$  \\
            \bottomrule
        \end{tabular}
    \end{center}
    \caption[
        The absolute differential cross section in \pb with respects to
        \phistar for \Ztoee events in our fiducial region from data unfolded
        with \POWHEG.
    ]{
        The absolute differential cross section in \pb with respects to
        \phistar for \Ztoee events in our fiducial region from data unfolded
        with \POWHEG, and the same distributions in \MADGRAPH and \POWHEG.
        These results are shown graphically in \cref{fig:results_abs_powheg}.
    }
    \label{tab:results_abs_powheg}
\end{table}


\chapter{Uncertainty Tables}
\label{app:uncertainty_tables}

The fractional uncertainty on the normalized cross section measurement is shown
in
\cref{fig:sys_uncert_norm,fig:sys_uncert_norm_powheg,fig:madgraph_uncert_norm,fig:powheg_uncert_norm},
and for the absolute cross section measurement is shown in
\cref{fig:sys_uncert_abs,fig:sys_uncert_abs_powheg,fig:madgraph_uncert_abs,fig:powheg_uncert_abs}.
In this appendix, the values of these uncertainties are presented in tabular
form.

\section{Explanation of the Columns}

The tables with information about the uncertainty on the data,
\cref{tab:sys_uncert_norm,tab:sys_uncert_norm_powheg,tab:sys_uncert_abs,tab:sys_uncert_abs_powheg},
contain the following columns:

\begin{description}[noitemsep]

    \item[\phistar Range:] \hfill \\
        The range of \phistar values included in the bin.

    \item[Total Uncertainty (Total):] \hfill \\
        The sum in quadrature of the all of the uncertainties.

    \item[Statistical Uncertainty (Stat.):] \hfill \\
        The uncertainty due to the limited number of events in the data, as
        discussed in \cref{ssec:stat_uncertainty}.

    \item[Total Systematic Uncertainty (Total Syst.):] \hfill \\
        The sum in quadrature of all of the systematic uncertainties including,
        for the absolute distribution, the luminosity uncertainty of
        \LumiUncertainty.

    \item[Monte Carlo Statistical Uncertainty (MC Stat.):] \hfill \\
        The uncertainty due to the limited number of events in the MC samples,
        as discussed in \cref{ssec:mc_stat_uncertainty}.

    \item[Pileup Uncertainty (Pileup):] \hfill \\
        The uncertainty due to the pileup reweighting, as discussed in
        \cref{ssec:pileup_uncertainty}.

    \item[Scale Factor Uncertainty (SF):] \hfill \\
        The uncertainty due to the scale factors, as discussed in
        \cref{scale_factor_uncertainty}.

    \item[\pt Scale Uncertainty (\pt Scale):] \hfill \\
        The uncertainty due to \pt scale, as discussed in
        \cref{ssec:pt_scale_uncertainty}.

    \item[Background Subtraction Uncertainty (Bkg.):] \hfill \\
        The uncertainty due to background subtraction, as discussed in
        \cref{ssec:background_subtraction_uncertainty}.

\end{description}

The tables with information related to the uncertainty on the MC distributions,
\cref{tab:madgraph_uncert_norm,tab:powheg_uncert_norm,tab:madgraph_uncert_abs,tab:powheg_uncert_abs},
have the following columns:

\begin{description}[noitemsep]

    \item[\phistar Range:] \hfill \\
        The range of \phistar values included in the bin.

    \item[Total Uncertainty (Total):] \hfill \\
        The sum in quadrature of the all of the uncertainties.

    \item[Statistical Uncertainty (Stat.):] \hfill \\
        The uncertainty due to the limited number of events in the MC sample.

    \item[Parton Density Function (PDF):] \hfill \\
        The uncertainty due to choice of PDF used to generate the \POWHEG MC,
        as discussed in \cref{ssec:pdf_uncertainties}.

    \item[Theoretical Cross Section Uncertainty (Cross Section):] \hfill \\
        The uncertainty in the theoretical cross section of the \MADGRAPH MC,
        as discussed in \cref{ssec:pdf_uncertainties}.

    \item[Final State Radiation Uncertainty (FSR):] \hfill \\
        The uncertainty due to the modeling of FSR, as discussed in
        \cref{ssec:fsr_uncertainties}.

\end{description}

\section{Tables}

The values of the various uncertainty in each \phistar bin are presented in the
tables that follow. \Cref{tab:sys_uncert_norm,tab:sys_uncert_norm_powheg}
present the fractional uncertainty in the normalized \phistar cross section in
data unfolded with \MADGRAPH and with \POWHEG.
\Cref{tab:madgraph_uncert_norm,tab:powheg_uncert_norm} present the uncertainty
on the normalized \phistar cross section in \MADGRAPH and \POWHEG MC,
respectively. \Cref{tab:sys_uncert_abs,tab:sys_uncert_abs_powheg} presents the
fractional uncertainty in the absolute \phistar cross section in data unfolded
with \MADGRAPH and with \POWHEG.
\Cref{tab:madgraph_uncert_abs,tab:powheg_uncert_abs} present the uncertainty on
the absolute \phistar cross section in \MADGRAPH and \POWHEG MC, respectively.

% Normalized

% tab:sys_uncert_norm
\begin{table}
    \spacerows{1.05}
    \begin{center}
        \begin{tabular}{@{}l l l l l l l l l@{}}
            \toprule
            \phistar Range  &  Total  &  Stat.  &  Total Syst.  &  MC Stat.  &  Pileup  &  SF    &  \pt Scale  &  Bkg.  \\
            \midrule
            0.000--0.004    &  0.43   &  0.26   &  0.34         &  0.33      &  0.04    &  0.07  &  0.01       &  0.05  \\
            0.004--0.008    &  0.50   &  0.28   &  0.41         &  0.40      &  0.03    &  0.07  &  0.02       &  0.05  \\
            0.008--0.012    &  0.47   &  0.29   &  0.37         &  0.36      &  0.02    &  0.07  &  0.00       &  0.04  \\
            0.012--0.016    &  0.50   &  0.29   &  0.41         &  0.40      &  0.02    &  0.07  &  0.01       &  0.04  \\
            0.016--0.020    &  0.51   &  0.29   &  0.42         &  0.39      &  0.14    &  0.07  &  0.00       &  0.04  \\
            0.020--0.024    &  0.51   &  0.30   &  0.41         &  0.39      &  0.10    &  0.07  &  0.02       &  0.04  \\
            0.024--0.029    &  0.44   &  0.27   &  0.35         &  0.33      &  0.03    &  0.07  &  0.01       &  0.04  \\
            0.029--0.034    &  0.46   &  0.28   &  0.37         &  0.36      &  0.04    &  0.07  &  0.01       &  0.04  \\
            0.034--0.039    &  0.48   &  0.28   &  0.38         &  0.37      &  0.04    &  0.06  &  0.02       &  0.04  \\
            0.039--0.045    &  0.43   &  0.27   &  0.34         &  0.33      &  0.01    &  0.06  &  0.00       &  0.05  \\
            0.045--0.052    &  0.42   &  0.25   &  0.34         &  0.33      &  0.01    &  0.06  &  0.00       &  0.04  \\
            0.052--0.057    &  0.52   &  0.32   &  0.40         &  0.40      &  0.01    &  0.05  &  0.01       &  0.05  \\
            0.057--0.064    &  0.45   &  0.28   &  0.35         &  0.35      &  0.02    &  0.05  &  0.01       &  0.04  \\
            0.064--0.072    &  0.45   &  0.27   &  0.36         &  0.33      &  0.11    &  0.05  &  0.01       &  0.04  \\
            0.072--0.081    &  0.43   &  0.26   &  0.34         &  0.33      &  0.01    &  0.04  &  0.02       &  0.04  \\
            0.081--0.091    &  0.42   &  0.26   &  0.33         &  0.32      &  0.02    &  0.03  &  0.00       &  0.04  \\
            0.091--0.102    &  0.43   &  0.27   &  0.34         &  0.33      &  0.07    &  0.03  &  0.00       &  0.04  \\
            0.102--0.114    &  0.44   &  0.27   &  0.35         &  0.34      &  0.05    &  0.02  &  0.00       &  0.04  \\
            0.114--0.128    &  0.43   &  0.27   &  0.33         &  0.33      &  0.03    &  0.03  &  0.01       &  0.03  \\
            0.128--0.145    &  0.43   &  0.26   &  0.34         &  0.32      &  0.10    &  0.03  &  0.01       &  0.03  \\
            0.145--0.165    &  0.42   &  0.26   &  0.33         &  0.32      &  0.04    &  0.04  &  0.00       &  0.03  \\
            0.165--0.189    &  0.42   &  0.26   &  0.33         &  0.32      &  0.05    &  0.05  &  0.02       &  0.03  \\
            0.189--0.219    &  0.43   &  0.26   &  0.33         &  0.32      &  0.04    &  0.06  &  0.01       &  0.06  \\
            0.219--0.258    &  0.42   &  0.26   &  0.33         &  0.32      &  0.04    &  0.08  &  0.01       &  0.04  \\
            0.258--0.312    &  0.42   &  0.26   &  0.33         &  0.31      &  0.03    &  0.10  &  0.01       &  0.05  \\
            0.312--0.391    &  0.43   &  0.26   &  0.35         &  0.31      &  0.04    &  0.13  &  0.00       &  0.07  \\
            0.391--0.524    &  0.45   &  0.26   &  0.37         &  0.31      &  0.00    &  0.15  &  0.00       &  0.12  \\
            0.524--0.695    &  0.56   &  0.32   &  0.46         &  0.38      &  0.04    &  0.19  &  0.02       &  0.18  \\
            0.695--0.918    &  0.73   &  0.39   &  0.62         &  0.47      &  0.20    &  0.23  &  0.01       &  0.27  \\
            0.918--1.153    &  0.95   &  0.53   &  0.79         &  0.65      &  0.10    &  0.27  &  0.05       &  0.33  \\
            1.153--1.496    &  1.07   &  0.61   &  0.87         &  0.71      &  0.08    &  0.31  &  0.06       &  0.39  \\
            1.496--1.947    &  1.27   &  0.77   &  1.02         &  0.85      &  0.00    &  0.33  &  0.05       &  0.45  \\
            1.947--2.522    &  1.73   &  0.98   &  1.43         &  1.10      &  0.64    &  0.35  &  0.00       &  0.54  \\
            2.522--3.277    &  2.02   &  1.21   &  1.62         &  1.36      &  0.56    &  0.34  &  0.10       &  0.59  \\
            \bottomrule
        \end{tabular}
    \end{center}
    \caption[
        Fractional errors for the normalized cross section measurement
        made with data unfolded with \MADGRAPH.
    ]{
        Fractional errors (in \%) for the normalized cross section measurement
        made with data unfolded with \MADGRAPH.
    }
    \label{tab:sys_uncert_norm}
\end{table}


% tab:sys_uncert_norm_powheg
\begin{table}
    \spacerows{1.05}
    \begin{center}
        \begin{tabular}{@{}l l l l l l l l l l@{}}
            \toprule
            \phistar Range  &  Total  &  Stat.  &  Total Syst.  &  MC Stat.  &  Pileup  &  SF    &  \pt Scale  &  Bkg.  &  PDF   \\
            \midrule
            0.000--0.004     &  0.82   &  0.26   &  0.77         &  0.77      &  0.05    &  0.07  &  0.01       &  0.05  &  0.01  \\
            0.004--0.008     &  0.93   &  0.28   &  0.88         &  0.87      &  0.03    &  0.07  &  0.02       &  0.05  &  0.11  \\
            0.008--0.012     &  0.99   &  0.28   &  0.94         &  0.91      &  0.21    &  0.07  &  0.00       &  0.04  &  0.08  \\
            0.012--0.016     &  1.03   &  0.29   &  0.99         &  0.98      &  0.08    &  0.07  &  0.01       &  0.04  &  0.10  \\
            0.016--0.020     &  0.89   &  0.29   &  0.84         &  0.83      &  0.03    &  0.07  &  0.00       &  0.04  &  0.03  \\
            0.020--0.024     &  0.96   &  0.30   &  0.91         &  0.90      &  0.14    &  0.06  &  0.02       &  0.04  &  0.05  \\
            0.024--0.029     &  0.91   &  0.27   &  0.87         &  0.83      &  0.23    &  0.07  &  0.01       &  0.04  &  0.03  \\
            0.029--0.034     &  0.84   &  0.27   &  0.80         &  0.78      &  0.10    &  0.06  &  0.01       &  0.04  &  0.08  \\
            0.034--0.039     &  0.88   &  0.29   &  0.83         &  0.82      &  0.08    &  0.06  &  0.02       &  0.04  &  0.09  \\
            0.039--0.045     &  0.77   &  0.26   &  0.72         &  0.72      &  0.02    &  0.06  &  0.00       &  0.04  &  0.01  \\
            0.045--0.052     &  0.77   &  0.25   &  0.73         &  0.69      &  0.20    &  0.06  &  0.00       &  0.04  &  0.13  \\
            0.052--0.057     &  0.99   &  0.32   &  0.94         &  0.93      &  0.05    &  0.05  &  0.01       &  0.05  &  0.03  \\
            0.057--0.064     &  0.81   &  0.28   &  0.77         &  0.75      &  0.08    &  0.05  &  0.01       &  0.04  &  0.08  \\
            0.064--0.072     &  0.78   &  0.27   &  0.74         &  0.72      &  0.12    &  0.05  &  0.01       &  0.04  &  0.06  \\
            0.072--0.081     &  0.77   &  0.26   &  0.73         &  0.72      &  0.02    &  0.04  &  0.02       &  0.04  &  0.06  \\
            0.081--0.091     &  0.75   &  0.26   &  0.70         &  0.69      &  0.05    &  0.04  &  0.00       &  0.04  &  0.05  \\
            0.091--0.102     &  0.76   &  0.27   &  0.71         &  0.71      &  0.03    &  0.03  &  0.00       &  0.04  &  0.04  \\
            0.102--0.114     &  0.78   &  0.27   &  0.73         &  0.72      &  0.07    &  0.03  &  0.00       &  0.04  &  0.05  \\
            0.114--0.128     &  0.76   &  0.27   &  0.71         &  0.70      &  0.02    &  0.03  &  0.01       &  0.03  &  0.03  \\
            0.128--0.145     &  0.76   &  0.26   &  0.71         &  0.71      &  0.08    &  0.03  &  0.01       &  0.03  &  0.01  \\
            0.145--0.165     &  0.81   &  0.26   &  0.76         &  0.73      &  0.21    &  0.04  &  0.00       &  0.03  &  0.02  \\
            0.165--0.189     &  0.76   &  0.26   &  0.72         &  0.71      &  0.06    &  0.05  &  0.02       &  0.03  &  0.04  \\
            0.189--0.219     &  0.76   &  0.26   &  0.71         &  0.70      &  0.03    &  0.07  &  0.01       &  0.06  &  0.03  \\
            0.219--0.258     &  0.76   &  0.26   &  0.71         &  0.70      &  0.07    &  0.08  &  0.01       &  0.04  &  0.03  \\
            0.258--0.312     &  0.74   &  0.26   &  0.70         &  0.69      &  0.03    &  0.10  &  0.01       &  0.05  &  0.01  \\
            0.312--0.391     &  0.75   &  0.26   &  0.71         &  0.69      &  0.03    &  0.12  &  0.00       &  0.08  &  0.04  \\
            0.391--0.524     &  0.81   &  0.26   &  0.77         &  0.75      &  0.02    &  0.15  &  0.00       &  0.12  &  0.03  \\
            0.524--0.695     &  0.96   &  0.32   &  0.90         &  0.85      &  0.16    &  0.19  &  0.02       &  0.18  &  0.03  \\
            0.695--0.918     &  1.22   &  0.39   &  1.15         &  1.06      &  0.27    &  0.23  &  0.01       &  0.27  &  0.04  \\
            0.918--1.153     &  1.66   &  0.53   &  1.57         &  1.47      &  0.20    &  0.26  &  0.05       &  0.33  &  0.28  \\
            1.153--1.496     &  1.88   &  0.61   &  1.78         &  1.70      &  0.08    &  0.30  &  0.06       &  0.39  &  0.17  \\
            1.496--1.947     &  2.34   &  0.77   &  2.21         &  2.11      &  0.28    &  0.33  &  0.05       &  0.45  &  0.15  \\
            1.947--2.522     &  2.99   &  0.98   &  2.82         &  2.75      &  0.03    &  0.33  &  0.00       &  0.54  &  0.18  \\
            2.522--3.277     &  3.64   &  1.20   &  3.43         &  3.17      &  1.12    &  0.33  &  0.10       &  0.58  &  0.18  \\
            \bottomrule
        \end{tabular}
    \end{center}
    \caption[
        Total errors (in \%) for the normalized cross section normalized
        measurement with \POWHEG unfolding.
    ]{
        Total errors (in \%) for the normalized cross section normalized
        measurement in different \phistar bins due to various sources. The
        unfolding was done with \POWHEG.
    }
    \label{tab:sys_uncert_norm_powheg}
\end{table}


% tab:madgraph_uncert_norm
\begin{table}
    \spacerows{1.05}
    \begin{center}
        \begin{tabular}{@{}l l l l@{}}
            \toprule
            \phistar Range & Total & Stat. & FSR \\
            \midrule
            0.000--0.004 & 0.27 & 0.26 & 0.03  \\
            0.004--0.008 & 0.27 & 0.27 & 0.03  \\
            0.008--0.012 & 0.27 & 0.27 & 0.03  \\
            0.012--0.016 & 0.27 & 0.27 & 0.03  \\
            0.016--0.020 & 0.27 & 0.27 & 0.03  \\
            0.020--0.024 & 0.28 & 0.28 & 0.02  \\
            0.024--0.029 & 0.26 & 0.25 & 0.03  \\
            0.029--0.034 & 0.26 & 0.26 & 0.02  \\
            0.034--0.039 & 0.27 & 0.27 & 0.02  \\
            0.039--0.045 & 0.25 & 0.25 & 0.02  \\
            0.045--0.052 & 0.25 & 0.25 & 0.01  \\
            0.052--0.057 & 0.30 & 0.30 & 0.01  \\
            0.057--0.064 & 0.27 & 0.27 & 0.01  \\
            0.064--0.072 & 0.26 & 0.26 & 0.00  \\
            0.072--0.081 & 0.26 & 0.26 & 0.00  \\
            0.081--0.091 & 0.26 & 0.26 & 0.01  \\
            0.091--0.102 & 0.27 & 0.27 & 0.01  \\
            0.102--0.114 & 0.27 & 0.27 & 0.01  \\
            0.114--0.128 & 0.27 & 0.27 & 0.02  \\
            0.128--0.145 & 0.27 & 0.27 & 0.02  \\
            0.145--0.165 & 0.27 & 0.27 & 0.02  \\
            0.165--0.189 & 0.27 & 0.27 & 0.02  \\
            0.189--0.219 & 0.27 & 0.27 & 0.02  \\
            0.219--0.258 & 0.28 & 0.27 & 0.03  \\
            0.258--0.312 & 0.28 & 0.27 & 0.03  \\
            0.312--0.391 & 0.28 & 0.28 & 0.03  \\
            0.391--0.524 & 0.28 & 0.28 & 0.03  \\
            0.524--0.695 & 0.34 & 0.34 & 0.03  \\
            0.695--0.918 & 0.42 & 0.42 & 0.03  \\
            0.918--1.153 & 0.57 & 0.57 & 0.03  \\
            1.153--1.496 & 0.65 & 0.65 & 0.03  \\
            1.496--1.947 & 0.80 & 0.80 & 0.03  \\
            1.947--2.522 & 1.02 & 1.02 & 0.04  \\
            2.522--3.277 & 1.26 & 1.26 & 0.04  \\
            \bottomrule
        \end{tabular}
    \end{center}
    \caption[
        The uncertainties for the normalized cross section from the \MADGRAPH
        MC sample.
    ]{
        The uncertainties (in \%) for the normalized cross section from the
        \MADGRAPH MC sample.
    }
    \label{tab:madgraph_uncert_norm}
\end{table}


% tab:powheg_uncert_norm
\begin{table}
    \spacerows{1.05}
    \begin{center}
        \begin{tabular}{@{}l l l l l@{}}
            \toprule
            \phistar Range & Total & Stat. & PDF & FSR \\
            \midrule
            0.000-0.004 & 0.63 & 0.62 & 0.13 & 0.02  \\
            0.004-0.008 & 0.63 & 0.62 & 0.13 & 0.02  \\
            0.008-0.012 & 0.65 & 0.63 & 0.15 & 0.02  \\
            0.012-0.016 & 0.65 & 0.63 & 0.15 & 0.02  \\
            0.016-0.020 & 0.65 & 0.64 & 0.13 & 0.02  \\
            0.020-0.024 & 0.65 & 0.64 & 0.13 & 0.02  \\
            0.024-0.029 & 0.60 & 0.58 & 0.12 & 0.03  \\
            0.029-0.034 & 0.60 & 0.59 & 0.09 & 0.02  \\
            0.034-0.039 & 0.62 & 0.61 & 0.10 & 0.01  \\
            0.039-0.045 & 0.58 & 0.57 & 0.09 & 0.02  \\
            0.045-0.052 & 0.55 & 0.54 & 0.08 & 0.02  \\
            0.052-0.057 & 0.68 & 0.67 & 0.11 & 0.02  \\
            0.057-0.064 & 0.60 & 0.59 & 0.11 & 0.01  \\
            0.064-0.072 & 0.59 & 0.57 & 0.14 & 0.01  \\
            0.072-0.081 & 0.57 & 0.57 & 0.06 & 0.00  \\
            0.081-0.091 & 0.57 & 0.57 & 0.04 & 0.00  \\
            0.091-0.102 & 0.61 & 0.58 & 0.18 & 0.00  \\
            0.102-0.114 & 0.60 & 0.60 & 0.09 & 0.01  \\
            0.114-0.128 & 0.61 & 0.60 & 0.12 & 0.02  \\
            0.128-0.145 & 0.60 & 0.59 & 0.11 & 0.02  \\
            0.145-0.165 & 0.62 & 0.60 & 0.16 & 0.02  \\
            0.165-0.189 & 0.64 & 0.61 & 0.19 & 0.01  \\
            0.189-0.219 & 0.65 & 0.61 & 0.23 & 0.03  \\
            0.219-0.258 & 0.63 & 0.61 & 0.16 & 0.02  \\
            0.258-0.312 & 0.63 & 0.61 & 0.15 & 0.03  \\
            0.312-0.391 & 0.65 & 0.62 & 0.20 & 0.03  \\
            0.391-0.524 & 0.67 & 0.63 & 0.24 & 0.03  \\
            0.524-0.695 & 0.81 & 0.78 & 0.23 & 0.02  \\
            0.695-0.918 & 1.05 & 0.97 & 0.41 & 0.03  \\
            0.918-1.153 & 1.47 & 1.34 & 0.60 & 0.05  \\
            1.153-1.496 & 1.74 & 1.57 & 0.76 & 0.03  \\
            1.496-1.947 & 2.15 & 2.00 & 0.78 & 0.04  \\
            1.947-2.522 & 2.76 & 2.55 & 1.07 & 0.06  \\
            2.522-3.277 & 3.26 & 3.04 & 1.17 & 0.03  \\
            \bottomrule
        \end{tabular}
    \end{center}
    \caption{
        Total errors (in \%) for the normalized cross section from the
        \POWHEG MC sample.
    }
    \label{tab:powheg_uncert_norm}
\end{table}


% Absolute

% tab:sys_uncert_abs
\begin{table}
    \spacerows{1.05}
    \begin{center}
        \resizebox{\columnwidth}{!}{%
            \begin{tabular}{@{}l l l l l l l l l@{}}
                \toprule
                \phistar Range  &  Total  &  Stat.  &  Total Syst.  &  MC Stat.  &  Pileup  &  SF    &  \pt Scale  &  Bkg.  \\
                \midrule
                0.000--0.004    &  2.72   &  0.26   &  2.70         &  0.33      &  0.48    &  0.43  &  0.16       &  0.02  \\
                0.004--0.008    &  2.72   &  0.28   &  2.70         &  0.40      &  0.41    &  0.43  &  0.17       &  0.05  \\
                0.008--0.012    &  2.71   &  0.29   &  2.69         &  0.36      &  0.39    &  0.43  &  0.15       &  0.03  \\
                0.012--0.016    &  2.72   &  0.29   &  2.71         &  0.40      &  0.43    &  0.43  &  0.16       &  0.03  \\
                0.016--0.020    &  2.75   &  0.29   &  2.73         &  0.39      &  0.58    &  0.43  &  0.15       &  0.03  \\
                0.020--0.024    &  2.71   &  0.30   &  2.69         &  0.39      &  0.33    &  0.43  &  0.17       &  0.04  \\
                0.024--0.029    &  2.72   &  0.27   &  2.70         &  0.34      &  0.47    &  0.44  &  0.15       &  0.04  \\
                0.029--0.034    &  2.72   &  0.28   &  2.71         &  0.36      &  0.48    &  0.43  &  0.14       &  0.03  \\
                0.034--0.039    &  2.72   &  0.29   &  2.71         &  0.37      &  0.48    &  0.43  &  0.17       &  0.03  \\
                0.039--0.045    &  2.71   &  0.27   &  2.70         &  0.33      &  0.45    &  0.44  &  0.15       &  0.03  \\
                0.045--0.052    &  2.71   &  0.25   &  2.70         &  0.33      &  0.45    &  0.44  &  0.13       &  0.03  \\
                0.052--0.057    &  2.73   &  0.32   &  2.71         &  0.40      &  0.45    &  0.43  &  0.15       &  0.03  \\
                0.057--0.064    &  2.71   &  0.28   &  2.70         &  0.35      &  0.41    &  0.44  &  0.16       &  0.03  \\
                0.064--0.072    &  2.73   &  0.27   &  2.72         &  0.34      &  0.56    &  0.44  &  0.16       &  0.02  \\
                0.072--0.081    &  2.71   &  0.26   &  2.70         &  0.33      &  0.45    &  0.44  &  0.16       &  0.04  \\
                0.081--0.091    &  2.71   &  0.27   &  2.70         &  0.32      &  0.46    &  0.44  &  0.14       &  0.03  \\
                0.091--0.102    &  2.72   &  0.27   &  2.71         &  0.33      &  0.51    &  0.44  &  0.15       &  0.03  \\
                0.102--0.114    &  2.72   &  0.27   &  2.71         &  0.34      &  0.49    &  0.43  &  0.14       &  0.04  \\
                0.114--0.128    &  2.72   &  0.27   &  2.70         &  0.33      &  0.47    &  0.44  &  0.15       &  0.05  \\
                0.128--0.145    &  2.70   &  0.26   &  2.69         &  0.32      &  0.37    &  0.43  &  0.16       &  0.07  \\
                0.145--0.165    &  2.70   &  0.26   &  2.69         &  0.32      &  0.41    &  0.43  &  0.15       &  0.05  \\
                0.165--0.189    &  2.70   &  0.26   &  2.69         &  0.32      &  0.39    &  0.43  &  0.14       &  0.05  \\
                0.189--0.219    &  2.70   &  0.26   &  2.69         &  0.32      &  0.41    &  0.43  &  0.14       &  0.10  \\
                0.219--0.258    &  2.72   &  0.26   &  2.70         &  0.32      &  0.48    &  0.43  &  0.16       &  0.09  \\
                0.258--0.312    &  2.71   &  0.26   &  2.69         &  0.31      &  0.43    &  0.42  &  0.15       &  0.10  \\
                0.312--0.391    &  2.70   &  0.26   &  2.69         &  0.31      &  0.42    &  0.42  &  0.15       &  0.12  \\
                0.391--0.524    &  2.71   &  0.26   &  2.69         &  0.32      &  0.42    &  0.41  &  0.14       &  0.17  \\
                0.524--0.695    &  2.73   &  0.32   &  2.72         &  0.38      &  0.49    &  0.41  &  0.12       &  0.23  \\
                0.695--0.918    &  2.73   &  0.39   &  2.70         &  0.47      &  0.22    &  0.41  &  0.11       &  0.32  \\
                0.918--1.153    &  2.81   &  0.54   &  2.76         &  0.65      &  0.33    &  0.42  &  0.11       &  0.38  \\
                1.153--1.496    &  2.85   &  0.61   &  2.79         &  0.71      &  0.32    &  0.43  &  0.07       &  0.44  \\
                1.496--1.947    &  2.95   &  0.77   &  2.84         &  0.85      &  0.39    &  0.44  &  0.15       &  0.50  \\
                1.947--2.522    &  3.09   &  0.99   &  2.93         &  1.11      &  0.21    &  0.44  &  0.15       &  0.59  \\
                2.522--3.277    &  3.27   &  1.21   &  3.04         &  1.36      &  0.13    &  0.44  &  0.08       &  0.64  \\
                \bottomrule
            \end{tabular}
        }
    \end{center}
    \caption[
        The uncertainties for the absolute cross section measurement made with
        data unfolded with \MADGRAPH.
    ]{
        The uncertainties (in \%) for the absolute cross section measurement
        made with data unfolded with \MADGRAPH. The total value and the total
        systematic value includes the uncertainty of \LumiUncertainty due to
        the luminosity.
    }
    \label{tab:sys_uncert_abs}
\end{table}


% tab:sys_uncert_abs_powheg
\begin{table}
    \spacerows{1.05}
    \begin{center}
        \begin{tabular}{@{}l l l l l l l l l l@{}}
            \toprule
            \phistar Range  &  Total  &  Stat.  &  Total Syst.  &  MC Stat.  &  Pileup  &  SF    &  \pt Scale  &  Bkg.  &  PDF   \\
            \midrule
            0.000-0.004     &  2.79   &  0.26   &  2.78         &  0.77      &  0.36    &  0.44  &  0.16       &  0.03  &  0.16  \\
            0.004-0.008     &  2.85   &  0.28   &  2.83         &  0.87      &  0.46    &  0.44  &  0.17       &  0.05  &  0.24  \\
            0.008-0.012     &  2.88   &  0.29   &  2.87         &  0.91      &  0.63    &  0.44  &  0.15       &  0.03  &  0.11  \\
            0.012-0.016     &  2.89   &  0.29   &  2.87         &  0.98      &  0.51    &  0.44  &  0.16       &  0.03  &  0.23  \\
            0.016-0.020     &  2.82   &  0.29   &  2.80         &  0.83      &  0.40    &  0.44  &  0.15       &  0.03  &  0.14  \\
            0.020-0.024     &  2.87   &  0.30   &  2.85         &  0.90      &  0.57    &  0.44  &  0.17       &  0.04  &  0.19  \\
            0.024-0.029     &  2.79   &  0.27   &  2.78         &  0.83      &  0.17    &  0.44  &  0.15       &  0.04  &  0.16  \\
            0.029-0.034     &  2.79   &  0.28   &  2.77         &  0.78      &  0.31    &  0.44  &  0.14       &  0.03  &  0.11  \\
            0.034-0.039     &  2.83   &  0.29   &  2.81         &  0.82      &  0.51    &  0.43  &  0.17       &  0.03  &  0.12  \\
            0.039-0.045     &  2.79   &  0.26   &  2.77         &  0.72      &  0.42    &  0.44  &  0.15       &  0.03  &  0.15  \\
            0.045-0.052     &  2.81   &  0.25   &  2.80         &  0.69      &  0.63    &  0.44  &  0.13       &  0.03  &  0.11  \\
            0.052-0.057     &  2.85   &  0.32   &  2.83         &  0.93      &  0.38    &  0.44  &  0.15       &  0.03  &  0.14  \\
            0.057-0.064     &  2.81   &  0.28   &  2.80         &  0.75      &  0.51    &  0.44  &  0.16       &  0.03  &  0.12  \\
            0.064-0.072     &  2.81   &  0.27   &  2.80         &  0.72      &  0.55    &  0.44  &  0.16       &  0.02  &  0.12  \\
            0.072-0.081     &  2.79   &  0.26   &  2.77         &  0.72      &  0.42    &  0.44  &  0.16       &  0.04  &  0.13  \\
            0.081-0.091     &  2.79   &  0.27   &  2.78         &  0.69      &  0.47    &  0.45  &  0.14       &  0.03  &  0.18  \\
            0.091-0.102     &  2.78   &  0.27   &  2.77         &  0.71      &  0.38    &  0.44  &  0.15       &  0.03  &  0.18  \\
            0.102-0.114     &  2.80   &  0.27   &  2.79         &  0.72      &  0.50    &  0.44  &  0.14       &  0.04  &  0.19  \\
            0.114-0.128     &  2.79   &  0.27   &  2.77         &  0.70      &  0.44    &  0.44  &  0.15       &  0.05  &  0.16  \\
            0.128-0.145     &  2.80   &  0.26   &  2.79         &  0.71      &  0.51    &  0.44  &  0.16       &  0.07  &  0.15  \\
            0.145-0.165     &  2.77   &  0.26   &  2.76         &  0.73      &  0.26    &  0.45  &  0.15       &  0.05  &  0.14  \\
            0.165-0.189     &  2.78   &  0.26   &  2.77         &  0.71      &  0.39    &  0.44  &  0.14       &  0.05  &  0.13  \\
            0.189-0.219     &  2.78   &  0.26   &  2.77         &  0.70      &  0.42    &  0.44  &  0.14       &  0.10  &  0.16  \\
            0.219-0.258     &  2.77   &  0.26   &  2.76         &  0.71      &  0.35    &  0.43  &  0.16       &  0.09  &  0.16  \\
            0.258-0.312     &  2.78   &  0.26   &  2.77         &  0.69      &  0.46    &  0.43  &  0.15       &  0.10  &  0.15  \\
            0.312-0.391     &  2.78   &  0.26   &  2.77         &  0.69      &  0.40    &  0.42  &  0.15       &  0.12  &  0.17  \\
            0.391-0.524     &  2.80   &  0.26   &  2.78         &  0.75      &  0.44    &  0.42  &  0.14       &  0.17  &  0.13  \\
            0.524-0.695     &  2.81   &  0.32   &  2.79         &  0.85      &  0.22    &  0.41  &  0.12       &  0.23  &  0.15  \\
            0.695-0.918     &  2.89   &  0.39   &  2.87         &  1.06      &  0.15    &  0.42  &  0.11       &  0.32  &  0.12  \\
            0.918-1.153     &  3.10   &  0.53   &  3.06         &  1.48      &  0.20    &  0.41  &  0.11       &  0.38  &  0.16  \\
            1.153-1.496     &  3.28   &  0.61   &  3.22         &  1.70      &  0.50    &  0.42  &  0.07       &  0.44  &  0.27  \\
            1.496-1.947     &  3.58   &  0.77   &  3.49         &  2.12      &  0.71    &  0.44  &  0.15       &  0.50  &  0.06  \\
            1.947-2.522     &  4.00   &  0.99   &  3.88         &  2.75      &  0.37    &  0.44  &  0.15       &  0.59  &  0.09  \\
            2.522-3.277     &  4.40   &  1.20   &  4.23         &  3.17      &  0.70    &  0.43  &  0.08       &  0.64  &  0.04  \\
            \bottomrule
        \end{tabular}
    \end{center}
    \caption[
        Total errors (in \%) for the absolute cross section measurement with
        \POWHEG unfolding.
    ]{
        Total errors (in \%) for the absolute cross section measurement in
        different \phistar bins due to various sources. The total value and the
        total systematic value includes the uncertainty of 2.6\% due to
        luminosity. The unfolding was done with \POWHEG.
    }
    \label{tab:sys_uncert_abs_powheg}
\end{table}


% tab:madgraph_uncert_abs
\begin{table}
    \spacerows{1.05}
    \begin{center}
        \begin{tabular}{@{}l l l l l@{}}
            \toprule
            \phistar Range & Total & Stat. & Cross Section & FSR \\
            \midrule
            0.000--0.004 & 3.32 & 0.26 & 3.30 & 0.27  \\
            0.004--0.008 & 3.32 & 0.27 & 3.30 & 0.27  \\
            0.008--0.012 & 3.32 & 0.27 & 3.30 & 0.27  \\
            0.012--0.016 & 3.32 & 0.27 & 3.30 & 0.27  \\
            0.016--0.020 & 3.32 & 0.27 & 3.30 & 0.27  \\
            0.020--0.024 & 3.32 & 0.28 & 3.30 & 0.27  \\
            0.024--0.029 & 3.32 & 0.25 & 3.30 & 0.27  \\
            0.029--0.034 & 3.32 & 0.26 & 3.30 & 0.28  \\
            0.034--0.039 & 3.32 & 0.27 & 3.30 & 0.28  \\
            0.039--0.045 & 3.32 & 0.25 & 3.30 & 0.28  \\
            0.045--0.052 & 3.32 & 0.25 & 3.30 & 0.29  \\
            0.052--0.057 & 3.33 & 0.30 & 3.30 & 0.29  \\
            0.057--0.064 & 3.32 & 0.27 & 3.30 & 0.29  \\
            0.064--0.072 & 3.32 & 0.26 & 3.30 & 0.29  \\
            0.072--0.081 & 3.32 & 0.26 & 3.30 & 0.30  \\
            0.081--0.091 & 3.32 & 0.26 & 3.30 & 0.30  \\
            0.091--0.102 & 3.32 & 0.27 & 3.30 & 0.30  \\
            0.102--0.114 & 3.33 & 0.27 & 3.30 & 0.31  \\
            0.114--0.128 & 3.33 & 0.27 & 3.30 & 0.31  \\
            0.128--0.145 & 3.33 & 0.27 & 3.30 & 0.32  \\
            0.145--0.165 & 3.33 & 0.27 & 3.30 & 0.32  \\
            0.165--0.189 & 3.33 & 0.27 & 3.30 & 0.32  \\
            0.189--0.219 & 3.33 & 0.27 & 3.30 & 0.32  \\
            0.219--0.258 & 3.33 & 0.27 & 3.30 & 0.32  \\
            0.258--0.312 & 3.33 & 0.27 & 3.30 & 0.32  \\
            0.312--0.391 & 3.33 & 0.28 & 3.30 & 0.32  \\
            0.391--0.524 & 3.33 & 0.28 & 3.30 & 0.33  \\
            0.524--0.695 & 3.33 & 0.34 & 3.30 & 0.32  \\
            0.695--0.918 & 3.34 & 0.42 & 3.30 & 0.32  \\
            0.918--1.153 & 3.36 & 0.57 & 3.30 & 0.33  \\
            1.153--1.496 & 3.38 & 0.65 & 3.30 & 0.33  \\
            1.496--1.947 & 3.41 & 0.80 & 3.30 & 0.33  \\
            1.947--2.522 & 3.47 & 1.02 & 3.30 & 0.34  \\
            2.522--3.277 & 3.55 & 1.26 & 3.30 & 0.34  \\
            \bottomrule
        \end{tabular}
    \end{center}
    \caption[
        The uncertainties for the absolute cross section from the \MADGRAPH MC
        sample.
    ]{
        The uncertainties (in \%) for the absolute cross section from the
        \MADGRAPH MC sample.
    }
    \label{tab:madgraph_uncert_abs}
\end{table}


% tab:powheg_uncert_abs
\begin{table}
    \spacerows{1.05}
    \begin{center}
        \begin{tabular}{@{}l l l l l@{}}
            \toprule
            \phistar Range & Total & Stat. & PDF & FSR \\
            \midrule
            0.000--0.004 & 2.72 & 0.62 & 2.63 & 0.27  \\
            0.004--0.008 & 2.71 & 0.62 & 2.63 & 0.27  \\
            0.008--0.012 & 2.68 & 0.63 & 2.59 & 0.27  \\
            0.012--0.016 & 2.75 & 0.63 & 2.66 & 0.28  \\
            0.016--0.020 & 2.71 & 0.64 & 2.62 & 0.28  \\
            0.020--0.024 & 2.74 & 0.64 & 2.65 & 0.28  \\
            0.024--0.029 & 2.73 & 0.58 & 2.65 & 0.27  \\
            0.029--0.034 & 2.70 & 0.59 & 2.62 & 0.27  \\
            0.034--0.039 & 2.72 & 0.61 & 2.64 & 0.28  \\
            0.039--0.045 & 2.69 & 0.57 & 2.61 & 0.28  \\
            0.045--0.052 & 2.70 & 0.54 & 2.63 & 0.28  \\
            0.052--0.057 & 2.75 & 0.67 & 2.66 & 0.28  \\
            0.057--0.064 & 2.74 & 0.59 & 2.66 & 0.29  \\
            0.064--0.072 & 2.75 & 0.57 & 2.67 & 0.29  \\
            0.072--0.081 & 2.71 & 0.57 & 2.63 & 0.30  \\
            0.081--0.091 & 2.70 & 0.57 & 2.62 & 0.30  \\
            0.091--0.102 & 2.75 & 0.58 & 2.67 & 0.30  \\
            0.102--0.114 & 2.72 & 0.60 & 2.63 & 0.31  \\
            0.114--0.128 & 2.72 & 0.60 & 2.64 & 0.32  \\
            0.128--0.145 & 2.71 & 0.59 & 2.63 & 0.31  \\
            0.145--0.165 & 2.71 & 0.60 & 2.62 & 0.31  \\
            0.165--0.189 & 2.73 & 0.61 & 2.65 & 0.31  \\
            0.189--0.219 & 2.73 & 0.61 & 2.64 & 0.33  \\
            0.219--0.258 & 2.67 & 0.61 & 2.58 & 0.32  \\
            0.258--0.312 & 2.66 & 0.61 & 2.57 & 0.32  \\
            0.312--0.391 & 2.61 & 0.62 & 2.52 & 0.33  \\
            0.391--0.524 & 2.57 & 0.63 & 2.47 & 0.32  \\
            0.524--0.695 & 2.65 & 0.78 & 2.51 & 0.32  \\
            0.695--0.918 & 2.65 & 0.97 & 2.44 & 0.33  \\
            0.918--1.153 & 2.79 & 1.34 & 2.42 & 0.35  \\
            1.153--1.496 & 2.91 & 1.57 & 2.43 & 0.33  \\
            1.496--1.947 & 3.20 & 2.00 & 2.47 & 0.33  \\
            1.947--2.522 & 3.62 & 2.55 & 2.55 & 0.35  \\
            2.522--3.277 & 4.03 & 3.04 & 2.62 & 0.33  \\
            \bottomrule
        \end{tabular}
    \end{center}
    \caption{
        Fractional errors (in \%) for the absolute cross section from the
        \POWHEG MC sample.
    }
    \label{tab:powheg_uncert_abs}
\end{table}


\chapter{QCD Background Fits}
\label{app:qcd_fits}

\begin{figure}[!htbp]
    \centering
    \begin{subfigure}[b]{0.5\textwidth}
        \includegraphics[width=\linewidth]{figures/qcd_fits/qcd_fit_plot_for_01.pdf}
        \caption{}
        \label{fig:qcd_fit_01}
    \end{subfigure}%
    % The comment right after suppresses white space that would push the images
    % to new lines
    \begin{subfigure}[b]{0.5\textwidth}
        \includegraphics[width=\linewidth]{figures/qcd_fits/qcd_fit_plot_for_02.pdf}
        \caption{}
        \label{fig:qcd_fit_02}
    \end{subfigure}
    % New line
    \begin{subfigure}[b]{0.5\textwidth}
        \includegraphics[width=\linewidth]{figures/qcd_fits/qcd_fit_plot_for_03.pdf}
        \caption{}
        \label{fig:qcd_fit_03}
    \end{subfigure}%
    \begin{subfigure}[b]{0.5\textwidth}
        \includegraphics[width=\linewidth]{figures/qcd_fits/qcd_fit_plot_for_04.pdf}
        \caption{}
        \label{fig:qcd_fit_04}
    \end{subfigure}
    \caption[
       The QCD data-driven background fits for the first set of four \phistar
       bins.
    ]{
       The QCD data-driven background fits for the first set of four \phistar
       bins. The data are shown as points with error bars, MC template as a
       dashed histogram, the analytic QCD background function as the dashed
       line, and the sum of the template and function as a solid histogram.
    }
    \label{fig:qcd_many_1}
\end{figure}

\begin{figure}[!htbp]
    \centering
    \begin{subfigure}[b]{0.5\textwidth}
        \includegraphics[width=\linewidth]{figures/qcd_fits/qcd_fit_plot_for_05.pdf}
        \caption{}
        \label{fig:qcd_fit_05}
    \end{subfigure}%
    % The comment right after suppresses white space that would push the images
    % to new lines
    \begin{subfigure}[b]{0.5\textwidth}
        \includegraphics[width=\linewidth]{figures/qcd_fits/qcd_fit_plot_for_06.pdf}
        \caption{}
        \label{fig:qcd_fit_06}
    \end{subfigure}
    % New line
    \begin{subfigure}[b]{0.5\textwidth}
        \includegraphics[width=\linewidth]{figures/qcd_fits/qcd_fit_plot_for_07.pdf}
        \caption{}
        \label{fig:qcd_fit_07}
    \end{subfigure}%
    \begin{subfigure}[b]{0.5\textwidth}
        \includegraphics[width=\linewidth]{figures/qcd_fits/qcd_fit_plot_for_08.pdf}
        \caption{}
        \label{fig:qcd_fit_08}
    \end{subfigure}
    \caption[
       The QCD data-driven background fits for the second set of four \phistar
       bins.
    ]{
       The QCD data-driven background fits for the second set of four \phistar
       bins. The data are shown as points with error bars, MC template as a
       dashed histogram, the analytic QCD background function as the dashed
       line, and the sum of the template and function as a solid histogram.
    }
    \label{fig:qcd_many_2}
\end{figure}


\begin{figure}[!htbp]
    \centering
    \begin{subfigure}[b]{0.5\textwidth}
        \includegraphics[width=\linewidth]{figures/qcd_fits/qcd_fit_plot_for_09.pdf}
        \caption{}
        \label{fig:qcd_fit_09}
    \end{subfigure}%
    % The comment right after suppresses white space that would push the images
    % to new lines
    \begin{subfigure}[b]{0.5\textwidth}
        \includegraphics[width=\linewidth]{figures/qcd_fits/qcd_fit_plot_for_10.pdf}
        \caption{}
        \label{fig:qcd_fit_10}
    \end{subfigure}
    % New line
    \begin{subfigure}[b]{0.5\textwidth}
        \includegraphics[width=\linewidth]{figures/qcd_fits/qcd_fit_plot_for_11.pdf}
        \caption{}
        \label{fig:qcd_fit_11}
    \end{subfigure}%
    \begin{subfigure}[b]{0.5\textwidth}
        \includegraphics[width=\linewidth]{figures/qcd_fits/qcd_fit_plot_for_12.pdf}
        \caption{}
        \label{fig:qcd_fit_11}
    \end{subfigure}
    \caption[
       The QCD data-driven background fits for the third set of four \phistar
       bins.
    ]{
       The QCD data-driven background fits for the third set of four \phistar
       bins. The data are shown as points with error bars, MC template as a
       dashed histogram, the analytic QCD background function as the dashed
       line, and the sum of the template and function as a solid histogram.
    }
    \label{fig:qcd_many_3}
\end{figure}


\begin{figure}[!htbp]
    \centering
    \begin{subfigure}[b]{0.5\textwidth}
        \includegraphics[width=\linewidth]{figures/qcd_fits/qcd_fit_plot_for_13.pdf}
        \caption{}
        \label{fig:qcd_fit_13}
    \end{subfigure}%
    % The comment right after suppresses white space that would push the images
    % to new lines
    \begin{subfigure}[b]{0.5\textwidth}
        \includegraphics[width=\linewidth]{figures/qcd_fits/qcd_fit_plot_for_14.pdf}
        \caption{}
        \label{fig:qcd_fit_14}
    \end{subfigure}
    % New line
    \begin{subfigure}[b]{0.5\textwidth}
        \includegraphics[width=\linewidth]{figures/qcd_fits/qcd_fit_plot_for_15.pdf}
        \caption{}
        \label{fig:qcd_fit_15}
    \end{subfigure}%
    \begin{subfigure}[b]{0.5\textwidth}
        \includegraphics[width=\linewidth]{figures/qcd_fits/qcd_fit_plot_for_16.pdf}
        \caption{}
        \label{fig:qcd_fit_16}
    \end{subfigure}
    \caption[
       The QCD data-driven background fits for the forth set of four \phistar
       bins.
    ]{
       The QCD data-driven background fits for the forth set of four \phistar
       bins. The data are shown as points with error bars, MC template as a
       dashed histogram, the analytic QCD background function as the dashed
       line, and the sum of the template and function as a solid histogram.
    }
    \label{fig:qcd_many_4}
\end{figure}

\begin{figure}[!htbp]
    \centering
    \begin{subfigure}[b]{0.5\textwidth}
        \includegraphics[width=\linewidth]{figures/qcd_fits/qcd_fit_plot_for_17.pdf}
        \caption{}
        \label{fig:qcd_fit_17}
    \end{subfigure}%
    % The comment right after suppresses white space that would push the images
    % to new lines
    \begin{subfigure}[b]{0.5\textwidth}
        \includegraphics[width=\linewidth]{figures/qcd_fits/qcd_fit_plot_for_18.pdf}
        \caption{}
        \label{fig:qcd_fit_18}
    \end{subfigure}
    % New line
    \begin{subfigure}[b]{0.5\textwidth}
        \includegraphics[width=\linewidth]{figures/qcd_fits/qcd_fit_plot_for_19.pdf}
        \caption{}
        \label{fig:qcd_fit_19}
    \end{subfigure}%
    \begin{subfigure}[b]{0.5\textwidth}
        \includegraphics[width=\linewidth]{figures/qcd_fits/qcd_fit_plot_for_20.pdf}
        \caption{}
        \label{fig:qcd_fit_20}
    \end{subfigure}
    \caption[
       The QCD data-driven background fits for the fifth set of four \phistar
       bins.
    ]{
       The QCD data-driven background fits for the fifth set of four \phistar
       bins. The data are shown as points with error bars, MC template as a
       dashed histogram, the analytic QCD background function as the dashed
       line, and the sum of the template and function as a solid histogram.
    }
    \label{fig:qcd_many_5}
\end{figure}

\begin{figure}[!htbp]
    \centering
    \begin{subfigure}[b]{0.5\textwidth}
        \includegraphics[width=\linewidth]{figures/qcd_fits/qcd_fit_plot_for_21.pdf}
        \caption{}
        \label{fig:qcd_fit_21}
    \end{subfigure}%
    % The comment right after suppresses white space that would push the images
    % to new lines
    \begin{subfigure}[b]{0.5\textwidth}
        \includegraphics[width=\linewidth]{figures/qcd_fits/qcd_fit_plot_for_22.pdf}
        \caption{}
        \label{fig:qcd_fit_22}
    \end{subfigure}
    % New line
    \begin{subfigure}[b]{0.5\textwidth}
        \includegraphics[width=\linewidth]{figures/qcd_fits/qcd_fit_plot_for_23.pdf}
        \caption{}
        \label{fig:qcd_fit_23}
    \end{subfigure}%
    \begin{subfigure}[b]{0.5\textwidth}
        \includegraphics[width=\linewidth]{figures/qcd_fits/qcd_fit_plot_for_24.pdf}
        \caption{}
        \label{fig:qcd_fit_24}
    \end{subfigure}
    \caption[
       The QCD data-driven background fits for the sixth set of four \phistar
       bins.
    ]{
       The QCD data-driven background fits for the sixth set of four \phistar
       bins. The data are shown as points with error bars, MC template as a
       dashed histogram, the analytic QCD background function as the dashed
       line, and the sum of the template and function as a solid histogram.
    }
    \label{fig:qcd_many_6}
\end{figure}

\begin{figure}[!htbp]
    \centering
    \begin{subfigure}[b]{0.5\textwidth}
        \includegraphics[width=\linewidth]{figures/qcd_fits/qcd_fit_plot_for_25.pdf}
        \caption{}
        \label{fig:qcd_fit_25}
    \end{subfigure}%
    % The comment right after suppresses white space that would push the images
    % to new lines
    \begin{subfigure}[b]{0.5\textwidth}
        \includegraphics[width=\linewidth]{figures/qcd_fits/qcd_fit_plot_for_26.pdf}
        \caption{}
        \label{fig:qcd_fit_26}
    \end{subfigure}
    % New line
    \begin{subfigure}[b]{0.5\textwidth}
        \includegraphics[width=\linewidth]{figures/qcd_fits/qcd_fit_plot_for_27.pdf}
        \caption{}
        \label{fig:qcd_fit_27}
    \end{subfigure}%
    \begin{subfigure}[b]{0.5\textwidth}
        \includegraphics[width=\linewidth]{figures/qcd_fits/qcd_fit_plot_for_28.pdf}
        \caption{}
        \label{fig:qcd_fit_28}
    \end{subfigure}
    \caption[
       The QCD data-driven background fits for the seventh set of four \phistar
       bins.
    ]{
       The QCD data-driven background fits for the seventh set of four \phistar
       bins. The data are shown as points with error bars, MC template as a
       dashed histogram, the analytic QCD background function as the dashed
       line, and the sum of the template and function as a solid histogram.
    }
    \label{fig:qcd_many_7}
\end{figure}

\begin{figure}[!htbp]
    \centering
    \begin{subfigure}[b]{0.5\textwidth}
        \includegraphics[width=\linewidth]{figures/qcd_fits/qcd_fit_plot_for_29.pdf}
        \caption{}
        \label{fig:qcd_fit_29}
    \end{subfigure}%
    % The comment right after suppresses white space that would push the images
    % to new lines
    \begin{subfigure}[b]{0.5\textwidth}
        \includegraphics[width=\linewidth]{figures/qcd_fits/qcd_fit_plot_for_30.pdf}
        \caption{}
        \label{fig:qcd_fit_30}
    \end{subfigure}
    % New line
    \begin{subfigure}[b]{0.5\textwidth}
        \includegraphics[width=\linewidth]{figures/qcd_fits/qcd_fit_plot_for_31.pdf}
        \caption{}
        \label{fig:qcd_fit_31}
    \end{subfigure}%
    \begin{subfigure}[b]{0.5\textwidth}
        \includegraphics[width=\linewidth]{figures/qcd_fits/qcd_fit_plot_for_32.pdf}
        \caption{}
        \label{fig:qcd_fit_32}
    \end{subfigure}
    \caption[
       The QCD data-driven background fits for the eighth set of four \phistar
       bins.
    ]{
       The QCD data-driven background fits for the eighth set of four \phistar
       bins. The data are shown as points with error bars, MC template as a
       dashed histogram, the analytic QCD background function as the dashed
       line, and the sum of the template and function as a solid histogram.
    }
    \label{fig:qcd_many_8}
\end{figure}

\begin{figure}[!htbp]
    \centering
    \begin{subfigure}[b]{0.5\textwidth}
        \includegraphics[width=\linewidth]{figures/qcd_fits/qcd_fit_plot_for_33.pdf}
        \caption{}
        \label{fig:qcd_fit_33}
    \end{subfigure}%
    % The comment right after suppresses white space that would push the images
    % to new lines
    \begin{subfigure}[b]{0.5\textwidth}
        \includegraphics[width=\linewidth]{figures/qcd_fits/qcd_fit_plot_for_34.pdf}
        \caption{}
        \label{fig:qcd_fit_34}
    \end{subfigure}
    \caption[
       The QCD data-driven background fits for the last two \phistar bins.
    ]{
       The QCD data-driven background fits for the last two \phistar bins. The
       data are shown as points with error bars, MC template as a dashed
       histogram, the analytic QCD background function as the dashed line, and
       the sum of the template and function as a solid histogram.
    }
    \label{fig:qcd_many_9}
\end{figure}

\chapter{Glossary and Acronyms}
\label{app:glossary}

Every occupation develops its own jargon, and while this aides in communication
between members of the group, it often hinders the understanding of those
unfamiliar with it. Where possible, jargon has been minimized, but as with any
technical publication, some is unavoidable. To aid the understanding of the
reader, commonly used terms and acronyms have been defined below.

% Glossary
\section{Glossary}
\label{sec:gloassary}

\begin{itemize}

\GlossaryEntry{Barrel}{The central region in $\eta$ of the detector.}
\GlossaryEntry{Compact Muon Solenoid (CMS)}{The detector that collected the data used in this thesis. See \SEC~\ref{sec:cms}.}
\GlossaryEntry{Endcap}{The portion of each subdetector that is flat and covers the high $|\eta|$ regions.}
\GlossaryEntry{Generator Level}{The information about an MC event as determined by the MC generator, before the event is passed through a detector simulation.}
\GlossaryEntry{Hadronization}{The process by which color charge is hidden from observation by producing colorless hadrons.}
\GlossaryEntry{Hadron}{Color neutral combinations of three quarks, for example protons and neutrons.}
\GlossaryEntry{Interaction Point}{The region at the center of the detector where proton-proton collisions occur.}
\GlossaryEntry{Jet}{A spray of high energy particles that originate from a colored object as it tries to maintain its colorless state.}
\GlossaryEntry{Large Hadron Collider (LHC)}{The collider used to produce the data used in this thesis. See \SEC~\ref{sec:lhc}.}
\GlossaryEntry{\Moliere Radius}{The radius in which 90\% of the energy of an electromagnetic shower is contained within for a given material.}
\GlossaryEntry{Monte Calro (MC)}{Simulated data. See \SEC~\ref{sec:mc}.}
\GlossaryEntry{Particle Flow}{An algorithm for reconstruction particles using information from multiple subdetectors. See \SEC~\ref{ssec:iso}.}
\GlossaryEntry{Parton}{The individual constituents of a proton including the valence quarks, gluons, and sea quarks. See \SEC~\ref{ssec:parton_model}.}
\GlossaryEntry{Pileup}{Additional proton-proton interactions which occur during an event.}
\GlossaryEntry{Prescaled}{To reduce the rate of a trigger by randomly throwing out events the trigger accepted.}
\GlossaryEntry{Primary Vertex}{The reconstructed location of the proton-proton interaction.}
\GlossaryEntry{Reconstructed Level}{The information about an event determined from the detector (for data) or the simulation of the detector response (in MC).}
\GlossaryEntry{Reconstruction}{The process of taking raw data from the detector and creating objects useful for physics. See \CHP~\ref{chapter:reconstruction}.}
\GlossaryEntry{Sea Quarks}{Pairs of quarks and antiquarks from gluon splitting that exist within each hadron.}
\GlossaryEntry{Simulation}{See \textit{Monte Carlo}.}
\GlossaryEntry{Transverse Momentum}{Momentum transverse to the beamline. In general denoted \pt, but \bosonpt is used specifically to mean the transverse momentum of the \Z or \W boson.}
\GlossaryEntry{Trigger}{A system that analyses events as they are happening and decides which ones to keep. See \SEC~\ref{ssec:trigger}.}
\GlossaryEntry{Truth Level}{See \textit{Generator Level}.}
\GlossaryEntry{Valence Quark}{The quarks which give rise to the quantum numbers of the proton, specifically the two up quarks and the down quark.}

\end{itemize}

% Acronyms
\section{Acronyms}
\label{sec:acronyms}

% Table formatting

% Heading for the first page
\begin{longtable}{p{0.25\textwidth} p{0.75\textwidth}}
\caption{Acronyms} \label{tab:acronyms} \\

\toprule
Acronym & Meaning \\
\midrule
\endfirsthead

% Heading for all subsequent pages
\multicolumn{2}{l}{\textit{\tablename\ \thetable{} -- Continued from previous page}} \\
\toprule
Acronym & Meaning \\
\midrule
\endhead

% Footer for each page that wraps over to the next
\multicolumn{2}{r}{\textit{Continued on next page}} \\
\bottomrule
\endfoot

% Footer for the end of the table
\bottomrule
\endlastfoot

% End table formatting

\AcronymEntry{CB}{Crystal ball}
\AcronymEntry{CERN}{Originally from \textit{Conseil Europ\'{e}en pour la Recherche Nucl\'{e}aire}, now the European Organization for Nuclear Research}
\AcronymEntry{CMS}{Compact Muon Solenoid}
\AcronymEntry{CSC}{Cathode stripe chambers}
\AcronymEntry{DAQ}{Data acquisition}
\AcronymEntry{DT}{Drift tubes}
\AcronymEntry{EB}{Electromagnetic calorimeter barrel}
\AcronymEntry{ECAL}{Electromagnetic calorimeter}
\AcronymEntry{EE}{Electromagnetic calorimeter endcap}
\AcronymEntry{ES}{Electromagnetic calorimeter preshower}
\AcronymEntry{FNAL}{Fermi National Accelerator Laboratory}
\AcronymEntry{FSR}{Final state radiation}
\AcronymEntry{GSF}{Gaussian-sum filter}
\AcronymEntry{HB}{Hadronic calorimeter barrel}
\AcronymEntry{HB}{Hadronic calorimeter endcap}
\AcronymEntry{HCAL}{Hadronic calorimeter}
\AcronymEntry{HF}{Forward hadronic calorimeter}
\AcronymEntry{HLT}{High-level trigger}
\AcronymEntry{HO}{Hadronic calorimeter outer}
\AcronymEntry{ID}{Electron identification}
\AcronymEntry{ISR}{Initial state radiation}
\AcronymEntry{L1}{\Lone trigger}
\AcronymEntry{LHC}{Large Hadron Collider}
\AcronymEntry{LO}{Leading order}
\AcronymEntry{MC}{Monte Carlo}
\AcronymEntry{NLO}{Next-to-leading order}
\AcronymEntry{NNLO}{Next-to-next-to-leading order}
\AcronymEntry{PDF}{Parton distribution function}
\AcronymEntry{PSB}{Proton synchrotron booster}
\AcronymEntry{PS}{Proton synchrotron}
\AcronymEntry{QCD}{Quantum Chromodynamics}
\AcronymEntry{QED}{Quantum Electrodynamics}
\AcronymEntry{RPC}{Resistive place chambers}
\AcronymEntry{SPS}{Super Proton Synchrotron}
\AcronymEntry{TEC}{Tracker endcap}
\AcronymEntry{TIB}{Tracker inner barrel}
\AcronymEntry{TID}{Tracker inner disk}
\AcronymEntry{TOB}{Tracker outer barrel}

\end{longtable}


% End of the Document
\end{document}
