No scientific work is created in isolation and this thesis is no exception.
Many people were instrumental in ensuring its completion, either directly
through their work, or indirectly through their influences on my life.

To begin, I would like to offer thanks to my parents, Hans and Laurel Gude, who
always encouraged my love of science. Whether it was when I wanted to study
dinosaurs, or planets and stars, or whales, or finally supernovae and
sub-atomic particles, they were always there cheering me on.

Thanks to my science and math teachers; they saw something in me when I was
young and helped to nurture the skills and interests I would need to be a
scientist, and for that I am forever in their debt.

A big thanks to the members of the Supernova Cosmology Project, especially Saul
Perlmutter, Anthony Spadafora, Nao Suzuki, and David Rubin. They made me part
of their group when I was a student at Berkeley and taught me how to do
precision physics. Working with these world-class scientists, and being able to
contribute, gave me the confidence I needed when I otherwise felt like I was
inadequate as a scientist.

A special thanks to the graduate students at the University of Minnesota:
Michael Albright, Dan Endean, Charles McEachern, Roxanne Radpour, Dominic
Rocco, Allan and Mandy Straub, and Jan Zirnstein. They offered their friendship
to a Californian sorely out of place in the Midwest and so filled the last six
years with joy.

I would especially like to thank the members of the CMS group at the University
of Minnesota: Roger Rusack, Yuichi Kubota, Bryan Dahmes, Nathaniel Joseph
Pastika, and Jared Turkewitz. They offered invaluable advice about difficulties
I encountered when I was just starting out in high energy physics. Thanks to
Kevin Klapoetke, who worked on this measurement before me, to Alexey Finkel,
who assisted me with this measurement, and Zachary Lesko, who also assisted me
and will take over now that I have graduated. A special thanks to my adviser,
Jeremiah Mans, who guided me through my formative graduate school years and who
made me the scientist I am today. Also thanks to Nicole Ruckstuhl, who worked
side-by-side on the analysis with me and without whom I would have been
completely lost. If this work is of any value, it is because of the
contributions from those two.

Of course, none of this work would have been possible without the thousands of
scientists, engineers, and other people working hard at CERN, the LHC, and CMS.
\textit{Merci beaucoup pour votre travail acharn\'{e}}!

Thanks to Alan Wu for his extensive edits and for letting me bounce ideas off
him when I was not sure how to phrase things. Without his help, this work would
have been even more impenetrable than it already is.

Finally, I would like to thank my wife, Connie Lam. She offered me her love and
support through the entire process, and never wavered even when the thesis
began to take over all of my free time. I love you!
