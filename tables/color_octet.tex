\begin{table}[h]
    \centering
    \spacerows{1.2}
    \begin{center}
        \begin{tabular}{c  c}
            %\toprule
            $\left( \xxbar{\red}{\blue} + \xxbar{\blue}{\red} \right) / \sqrt{2}$ &
            $-i \left( \xxbar{\red}{\blue} - \xxbar{\blue}{\red} \right) / \sqrt{2}$ \\
            $\left( \xxbar{\red}{\green} + \xxbar{\green}{\red} \right) / \sqrt{2}$ &
            $-i \left( \xxbar{\red}{\green} - \xxbar{\green}{\red} \right) / \sqrt{2}$ \\
            $\left( \xxbar{\blue}{\green} + \xxbar{\green}{\blue} \right) / \sqrt{2}$ &
            $-i \left( \xxbar{\blue}{\green} - \xxbar{\green}{\blue} \right) / \sqrt{2}$ \\
            $\left( \xxbar{\red}{\red} - \xxbar{\blue}{\blue} \right) / \sqrt{2}$ &
            $\left( \xxbar{\red}{\red} - 2\xxbar{\blue}{\blue} + \xxbar{\green}{\green} \right) / \sqrt{6}$ \\
            %\bottomrule
        \end{tabular}
        \caption[
            One possible QCD color-octet.
        ]{
            One of the possible color-octets. The colors are red (\red), blue
            (\blue), green (\green), and their anticolors ($\overline{\red}$,
            $\overline{\blue}$, and $\overline{\green}$) .
        }
        \label{table:gluon_color}
    \end{center}
\end{table}
