\begin{table}[h]
    \centering
    \spacerows{1.2}
    \begin{center}
        \begin{tabular}{c  c}
            %\toprule
            $\left( \xxbar{r}{b} + \xxbar{b}{r} \right) / \sqrt{2}$ &
            $-i \left( \xxbar{r}{b} - \xxbar{b}{r} \right) / \sqrt{2}$ \\
            $\left( \xxbar{r}{g} + \xxbar{g}{r} \right) / \sqrt{2}$ &
            $-i \left( \xxbar{r}{g} - \xxbar{g}{r} \right) / \sqrt{2}$ \\
            $\left( \xxbar{b}{g} + \xxbar{g}{b} \right) / \sqrt{2}$ &
            $-i \left( \xxbar{b}{g} - \xxbar{g}{b} \right) / \sqrt{2}$ \\
            $\left( \xxbar{r}{r} - \xxbar{b}{b} \right) / \sqrt{2}$ &
            $\left( \xxbar{r}{r} - 2\xxbar{b}{b} + \xxbar{g}{g} \right) / \sqrt{6}$ \\
            %\bottomrule
        \end{tabular}
        \caption[
            One possible QCD color-octet.
        ]{
            One of the possible color-octets. The colors are red ($r$), blue ($b$),
            green ($g$), and their anti-colors ($\overline{r}$, $\overline{b}$, and
            $\overline{g}$) .
        }
        \label{table:gluon_color}
    \end{center}
\end{table}
